\chapter{【長文】其實月夕行動好反映到我地目前嘅抗爭陰暗面}

\#1阿嫲飲可樂•4 年前

唔知大家係以一個咩expectation入黎睇呢篇文呢?自己對於標題有無答案?

小弟12年開始接觸社運(國教),去到今時今日,其實香港嘅抗爭變化好大

以前睇抗爭 只係會二分法 公民抗命v.s.武力抗爭 呢種諗法其實都根深蒂固落我地每一個人到

導致我地宜家會覺得“進化”、“跟上” 就係將我地認爲無乜用嘅公民抗命 轉化成 更激烈武力抗爭

然而,我地過去一年經歷話我地知,武力抗爭帶黎嘅推進其實同公民抗命,其實老實講真係相差唔多

所以我都可以大膽引用三步中出西門巴嘅說法“一般國家推翻政權既方法已經證實左冇用” 【附錄1】

面對目前好絕望,唔係幾覺得有未來嘅社會,我地嘅抗爭其實都慢慢將香港人陰暗面呈現出黎

以下講幾個見到嘅:

經歷多次嘅慘淡收場,我地係咁比藉口自己逃避,以下幾個,恕我唔能夠一一盡錄

“CCP咁強大” “班黃絲港豬唔值得贏” “2046仲街頭抗爭” 上面嘅statement 其實錯唔嗮

但我地如果一路係咁諗,其實只會固化思維,令我地走唔出 “我地贏唔到”嘅局面

自己比藉口,然後相信左藉口,沉迷左藉口,導致心態上已經輸左,呢個係第一點

第二點就係習慣性嘅揶揄同批評,而依兩樣野嘅背後係我地嘅自大

“我食鹽多過你食米”,依句我地咁討厭嘅句子其實都幾能描述我地嘅心態

月夕行動一出,已經有人話起名好On9,亦都有人話set rb/遊行 已經無意思

面臨呢個情況,搞手選擇批評批評佢嘅人,唔願意改善

之後見無料到,其他人就笑搞手On9,又或者屌佢水人出黎

我地嘅自大,令我地難以接受其他人,令自己接受唔到批評之餘

亦變得好容易批評人地,lunch哥就係一個例子

[我地好多時會用自己嘅經驗去同人講唔work,係呢度唔討論經驗係咪可以必然反映個件事嘅結果。比起講唔work,提出改善者更應該善用已有嘅經驗改善個plan,而唔係否定佢(我嘅諗法)]

第三點,太識走精面

呢個其實可以係優點,但逐漸變成缺點黎

例如一開始嘅遊行已經由中途站行(唔想係銅鑼灣逼)

到街頭嘅時候唔Blackblock唔帶gear淨出

到現時,準備移民然後繼續抗爭,又或者有行動唔出,淨做鍵戰

上述三樣接近無敵地出左力 又最低成本同危險 我唔係話有問題

但走精面其實都表示緊,出少左力

其他手足要逼你個份,由其他手足承擔被捕風險,由係香港嘅人承擔赤裸裸嘅壓力

係呢度唔評論上述行爲,因爲我覺得過到自己個關,你咪做

但宜家係要大家反思走精面呢個雙刃劍,係成爲左你嘅藉口,定係幫緊你抗爭?

講完,其實自己喊緊

今日等大家出現,見自己個區包含在內,就即刻出,出到去都預左少人,無諗到轉過頭就取消。等多陣就翻去,望下自己有份嘅tg grp,睇上少少先發現原來好多人都唔睇好,甚至係度冷嘲熱諷緊。可能係對家玩分化,但我相信都有唔少係自己人鬧緊自己人。我直到今日之前都無放棄過抗爭,放棄過香港。但今日有一刻,我一個人望住個十字路口同盞紅綠燈,我想放棄啦。然後我先諗翻起香港人其實一直都係咁。其實一直以黎都係冷嘲熱諷嘅社會,就係一班投機取巧嘅人。一直以黎就係上連登勿認真,一直學緊嘅就係顧掂自己唔洗理其他人。但再諗多諗,呢啲咪就上面講嘅比藉口自己,我咪又係想屌班唔出黎嘅人?

到呢一刻我先明,所謂嘅革命,第一步唔係令到社會有乜乜改善,唔係五大訴求

第一步,係針對我地香港人心態同風氣嘅革命

透過革命,革走我地成個社會嘅不足同唔好嘅風氣

透過不斷嘅失敗,磨走我地嘅自大,學識檢討

透過未來越黎越難,再無辦法走精面,唯有逆流而上

透過再無藉口,令自己認清現實,慢慢穩固革命資本,心態由輸變贏

