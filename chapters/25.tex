\chapter{我屋企係典型嘅一個窮人家庭}

我屋企係典型嘅一個窮人家庭

我屋企係典型嘅一個窮人家庭

父母都係從事服務業

兩個英文都唔識多隻

佢地從細對我灌輸嘅知識係「抵」

好似抵食同抵玩

啲嘢食一超過佢地心中嘅數

就會話唔抵食

轉買返啲抵食嘅

好似麵包舖啲pizza麵包/貴啲嘅包

佢地會話有咩特別

一啲都唔抵食

食餐包/雞尾包仲好過 又好味

然後佢地會臨收鋪先去買

因為平啲 攞嚟做我第二朝早餐啱晒

仲有一個例子係去酒樓飲茶時嘅

兩老平時好鍾意飲茶

所以我次次都比佢地拖去

大家都知蝦餃係酒樓算特點

價錢比較昂貴 但我細個又好鍾意食

但次次都唔比我叫

因為佢地話唔抵食

甚至唔健康都夠膽講

但轉頭又叫燒賣鳯爪呢啲更唔健康嘅

最心痛嘅有年親戚朋友由外國嚟香港玩

兩老請佢地飲茶 佢地個仔話想食蝦餃

我父母即刻叫咗兩籠

當時仲係小學嘅我係全部人面前講咗句

嘩 我平時想食都無得食嘅蝦餃啊

當晚返屋企比我老母用藤條打到雙腳紅晒

再講抵玩

同學問一唔一齊去迪士尼/海洋公園玩呢啲一定無份

因為係佢地眼中又係貴到仆街

去樓下公園玩仲好過

出國旅行又係 從來無帶過我出去

我大學畢業先第一次出國

中小學無零用錢過

得食飯同搭車

仲要係比到啱啱好嗰隻

多啲都無

一有咩想買就會扯到其他野

你成績咁差都好意思想買嘢?

但屋企一個人都幫唔到我

問英文唔識 問數學唔識

補習又嫌貴 唔比錢去上

另一邊就屌點解得6,70分

如果我有書讀一定叻過你

自己文化水平低但又希望望子成龍

咁嘅學習環境點會成到材?

過年利是錢比完我之後 又話幫我保管

話就話我嘅 實情一啲都唔用得

有年我有個模型好想買 就拎咗利是錢

比老母知道後又係打到我仆街

我駁咗句我嘅利是錢點解唔用得

結果打得仲西利

打下打下發覺都打咗好多

之後得閒再打啦

多謝睇到呢度嘅大家

