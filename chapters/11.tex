\chapter{原來自己係咁依賴老母}

[長文] 原來自己係咁依賴老母

講返少少背景先

自己屋企本身窮,我好細個果陣老豆就出去滾,所以十幾歲就老母兩個人住,老母學歷又唔高,可以做既野多數都係辛苦工,但佢都堅持賺錢為我供書教學,佢辛苦左成日每晚放工番到黎都仲會煮飯俾我食,但我仲成日會成日嚴送一樣,對佢好多不滿,又唔幫佢手做家務,有時佢問多幾句野我就已經會嚴佢煩,嚴佢咩都唔識,成日同佢嘈交。可能我由細到大都係自己一個人咁濟,根本唔識咩係自律,初中果陣每日返學訓覺,夜晚打機咁就一日,根本無理過任何屋企既人,好似只係為左打機而生存咁,老母俾我果零用錢仲瞞住佢用黎課金買點數,個金額仲要好大下,但佢完全唔知情,以為我用曬黎食野,出糧多左俾我零用錢亦都會增加。

咁到人大少少啦,開始有番自覺,知道佢既辛苦,但我就成日都口不對心,唔肯同佢講心底說話,我地之間既關係,係連母親節講句母親節快樂,影張合照,同佢講句多謝都會覺得尷尬。由細到大佢都同我講知識改變命運,於是我高中果陣就決定努力讀書,諗住將來報答佢,好好彩地考完,我真係入到想入既U,佢亦都好開心,好自豪咁同朋友曬命,收到果一刻佢好似仲開心過我咁。

但個天唔會完全順曬你意,響我二󲆱既某日突然收到醫院黎既電話通知我話佢收工搭車果陣遇上交通意外,我當堂驚到唔知點好,好彩有\lr{亻}{}響我身邊鼓勵我,陪我去醫院。由果陣開始我先發現自己係有幾緊張佢,幾咁依賴佢,好好彩地佢今次只係整親隻腳,坐左年幾輪椅,隻腳都慢慢行得到,佢唔方便行既呢段時間,我同佢相處既時間多左,關係亦都開始慢慢變好,到上年佢隻腳亦都好好彩好返曬,又無咩後遺症於是佢開始返番工,但之後其實我都睇得出佢每次放工番黎都好唔開心,有次仲聽到佢同朋友講話d同事成日針對佢,嚴佢隻腳之前有事,做野就住就住,做得好慢,但佢從來都未向我提過一句,每次番到黎都話今日好開心,完全都唔辛苦。

上個禮拜佢番到屋企果陣突然同我講今晚唔煮飯,話好累,好早就沖涼訓覺,同佢相處左咁多年,好少聽到佢講唔煮飯,但我都無理到就繼續做自己野,到半夜大概3,4點佢突然開燈整醒左我,我即刻睇下咩事,哦,原來佢肚餓係度搵野食,但越搵就越開始唔對路,佢眼神開始飄忽,面青口唇白咁,我同佢講既野佢都好似聽唔到咁,之後佢好辛苦咁同我講話心跳得好快,好似就黎跳出黎咁,咁我就即刻\lr{言}{}白車送佢入院,前前後後搞左兩個禮拜左右曬心電圖又剩,暫時確認係心絞痛,要排期照野,但未確認到係咪心臟病,住左一排院,醫院果邊都俾佢出院,只係開左脷底丸同阿士匹靈俾佢。

呢幾日番到屋企佢都好精神咁無咩野同未入院前一樣,佢仲講笑咁話星期一(即係22號今日就返得工添),但世事總係唔如意,今晚同佢食完返番屋企佢陣,佢就話條腰好痛,痛到係連行路都有障礙,幾經辛苦番到屋企,佢痛到即刻要攤響梳化,攤左一陣腰無咁痛,頭又開始痛,之後佢話連心都開始好似有野壓住咁,含完脷底丸都無效,佢開始連眼都擘唔大,情況仲差過對上一次入院,到我\lr{言}{}完白車 佢響神智不清之下竟然係叫我打俾老豆,擔心萬一有咩事得番我自己一個叫我去搵老豆,我眼淚即刻留到唔停,叫佢唔洗緊張白車好快黎到。呢半個鐘應該係我人生最萬長既半個鐘,由屋企到上白車,由白車到入院,佢冷汗係完全無停過,亦都不停同救護講話好辛苦,睇到我個心拿住拿住,但又唔想佢擔心我,只可以響隔離陪住佢一直係扮無野,扮堅強。

入完院要即刻推入急救房,我個人變到好慌,隻手不停咁震,好擔心佢有事,果一刻先發現原來自己所謂既慣左自己一個,只係因為當時既天真無知,萬一佢真係出左事得番自己一個都唔知點算。搞左一大輪,佢由急救房出黎我先鬆返口氣,雖然佢個狀況都係好差但個人都係清醒,醫院果邊話留院觀察幫佢詳細咁照下咩事跟住就開始推佢入病房,但病房依家根本唔俾人入去,我最後見到佢只係佢響急救房出黎既一瞬間,佢好勉強擘大眼咁望住我,用好冰既手拖住我講左句「唔洗擔心我,過幾日轉涼呀著多件衫」,聽完之後眼淚更加擁曬上黎,但響佢面前我緊係響度死頂,扮堅強,直到離開左佢視線淚線已經更加控制唔到。

有時真係好唔明到底點解個天可以咁唔公平,點解一個咁善良既媽媽要受咁多痛苦?由生完我果一刻,活左咁多年根本無享過福,生完我出黎好似仲害左佢咁,覺得佢搞成咁都係因為自己迫到佢咁。依家可以做既只係等醫院報告,又無得探佢,真係感覺好無助,好憎自己咩都做唔到,但無論點都好,我會堅強,響你同病毒搏鬥果陣照顧自己,做好家務汁到間屋靚一靚,等你番屋企,求下上天俾個機會我照顧番你,帶你享下福啦好嘛?

打呢篇野唔係為左博咩同情,而係想提醒各位好多野都講唔埋,生命呢樣野其實真係好兒戲。希望咁多位真係要愛錫你地屋企人,對佢地好,無論點都好佢地都係養大你既人,最錫你既人,佢地所做既野都係為你地好,唔好好似我咁出事先黎後悔。

簡簡單單咁講句,我愛你或者多謝你已經係父母最大既恩慰。

