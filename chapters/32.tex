\chapter{移民英國一周年}

移民英國一周年,之前一篇分享對於收支平衡既擔心,今次講下心態上既調整。

第一,天氣方面,我開頭以為自己唔怕凍應該會好易適應,但黎左一年,發覺始終未算,見人地十度淨係著短袖背心,我真係唔得。識我既人都知我係中意夏天,勁討厭冬天既人,加上係中意上山下海,陽光與海灘既人,而英國有半年時間都算係冬天,絶對唔易過。我特登揀wales 主要中意佢有山行有海灘,可惜世事就唔係想像咁美好,9月天氣就已經得10度,去咩行山海灘呢?冬天真係好漫長,尤其十一月開始較慢一個鐘,4點就天黑都算,最慘係冬天會基本上日日落雨,隔個星期就打風,成日橫風橫雨,莫講話去玩,出街都未必想。之前9月頭滯留係西班牙旅行,感覺好舒服,日日好好太陽,一返到英國就變左另一個世界,我終於明白點解咁多英國人中意去西班牙。

第二,食物方面,真係好少選擇。開頭黎,見到超市好大好興奮,但其實食物選擇唔多,肉類的確平過香港,質素唔錯,但海鮮、蔬菜、生果好少選擇,超市雖然係好大,啱買既野唔多。出街食仲攞命,Cardiff 依度既餐廳仲少選擇。其實我已經算中意食西餐,開始黎到覺得好新鮮,都想試吓唔同嘅餐廳。可惜食黎食去,都係fish and fries , burger 之類,頂籠rib、steak 好悶,連有paster嘅餐廳都唔多,有時都會食下泰國野日本野,但依度真係得好少選擇。

第三就係要接受依度既人處事態度。有時候打去customer service hotline ,等好耐不特止,有時仲傾得唔like ,會cut 你線,甚至鬧翻你轉頭,香港係無可能發生。另外送貨,遲到無通知好正常,完全無到都好正常,啲人做嘢真係好無交帶,唯有再追多幾次。約GP 睇醫生,難過登天,成日無得比你睇,有時要將個病情講得嚴重啲,或者唯有自己買藥食算啦。住得耐慢慢就要接受,呢度啲人嘅處事手法,有時係會好得人驚,完全唔可以用香港果套,感覺氣餒係好正常。

黎到一年,心理調整好重要。我地要接受好多野唔係想像中咁美好,我絕對唔會報喜不報憂。開頭黎到當然覺得好新鮮,啲屋好靚,啲公園好大,但之後就會覺得無乜特別,個個英國城市都差唔多。之前會成日揸車去附近城市玩,依家都少左。新鮮感無左,就要好好調整心理,用唔同方法去適應生活遇到既落差問題。另外多啲搵朋友傾訴下都好,睇下有咩大家都遇到既問題同解決方法。

最後,你問我會唔會後悔過黎,我會話一定唔會。因為英國有既最重要係自由既空氣,小朋友讀書最開心既,唔需要好似香港咁大壓力,當然無所謂紅色思想洗腦已經好好。而鄰里關係亦係香港比唔到,大家會互相幫助,互相分享美食,係開心既。而依度既居住環境當然比香港好好多,香港啲屋豆腐潤咁細,真係好侷促,周圍高樓大廈,依度就另一個世界。點都好,大家努力啦,要係度生活絕對唔容易。

