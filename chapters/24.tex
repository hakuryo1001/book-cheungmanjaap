\chapter{成日覺得廣東歌有種死唔斷氣既感覺}

\#1arkk暨tsm苦主

•2 年前

早兩日同同事傾開聽歌,講到對廣東歌(特別係近呢個世紀)有種死唔斷氣既感覺,好多假聲好多氣音,好慘好慘,我好慘我好慘,我天下間最慘,慘到想死,我要死啦,我死啦我死啦 .......呀!!!

無失戀既時候聽廣東歌會頹,失戀緊個陣聽廣東歌會頹上加頹。咁傾開呢個話題,我就研究廣東歌同歐美歌既差異,發現有2個原因導致有我好慘我好慘既感覺

第一個原因係旋律,廣東歌既旋律有9成都係單戀失戀慘慘慘情歌,而呢堆慘情歌全部都係同一個TONE既抒情歌,旋律慢,整體音樂大部份時間都處於高音既位置。而歐美既失戀歌,有抒情,有控訴,有快歌有慢歌,有高音有低音,會係唔同角度演繹失戀

舉個例,抒情既有 Passenger - Let Her Go

https://www.youtube.com/watch?v=RBumgq5yVrA

控訴既有 Lewis Capaldi - Someone You Loved

https://www.youtube.com/watch?v=zABLecsR5UE

輕快既有 Charlie Puth - We Don't Talk Anymore (feat. Selena Gomez)

https://www.youtube.com/watch?v=3AtDnEC4zak

第二個原因係歌手既表達,因為隻歌大部份時間高音,令到歌手長期要用假音氣音,而呢個高音既位置係大部份既歌手都唔係好處理到,每次都好似「鏈」住喉嚨唱咁,唱到好虛,唱到有D斷氣FEEL。特別係男歌手,幾乎無低音既部份,成首歌用晒假音唱。大佬,你男人黎架,比D雄性魅力黎聽下

以上係唔識音樂既細佬以聽眾既角度分析,想問下大家意見,順便介紹D歌比我聽

