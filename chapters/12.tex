\chapter{覺唔覺「父母已經比哂最好既野你」難聽過粗口?}

[長文] 覺唔覺「父母已經比哂最好既野你」難聽過粗口?

講明先,如果你睇完我篇文覺得唔同意,覺得我係不孝嘅,咁你應該係個由細到大都好幸福嘅人,你可以出返去,冇你嘅事

個仔細個果陣就話佢唔夠啊邊個個仔叻,成績唔夠邊個好,點解人地個仔得你就唔得

大個左就話點解人地仔女做律師醫生專業人士搵得多錢,點解你一蚊家用都唔肯比

話父母已經比哂最好嘅野你,點解又冇出色又乜又物

但係啲仆街父母永遠唔會照下鏡反省下自己

人地老豆老母同仔女嘅關係就好似好朋友咁唔會有代溝,你有冇?

人地老豆老母住大屋揸靚車食好野成日去旅行去周圍見識,你有冇?

人地老豆老母得閒就陪下仔女傾下計相處下鼓勵下佢地想做嘅事,你有冇?

人地老豆老母唔會拎人地同自己仔女比較,唔會嫌棄佢地做得唔好,你有冇?

窮,成長環境惡劣,衣食住行好多野本身已經要就住就住

食都食唔飽,以前中學飯錢一個禮拜得200蚊,即係一日40蚊,果陣學生餐平極都要廿到尾三十頭,仲要加埋早餐同埋放學食野,其實係食唔飽,而且係冇餘錢平時同同學仔出街玩。唔好同我講咩早餐食個包就算,極唔健康之餘,仲會影響朝早上堂表現,相信大家都見過女同學試過早會暈低,通常都係冇食野加埋天氣熱又要企係到曬\lr{午}{}\lr{食}{}又貪平買啲濕鳩撈麵當一餐,緊係發唔到育。

養兒防老觀念就覺得生多幾個第時養返你退休唔憂柴憂米,好似生仔生女一定會一表人才出人頭地咁

大佬呀,你買股票都唔係穩賺啦,更何況生仔咁大「投資」?

到左仔女大個左,出到嚟唔係父母心目中諗嘅一樣,就會怨啲仔女冇用,冇出色,唔生性

係個咁差嘅環境中成長,廢老廢家長只會踩到自己嘅仔女一文不值,講咩人地咁叻你咁蠢第時屎都冇得食乞衣都冇得做

自己就打份牛工早出晚歸,同仔女一齊傾計、食飯嘅時間根本少到數唔到

仔女好多時侯發生咩事都係自己一個面對,好孤獨,有咩問題都係自己解決,有部份就會學壞做仔妹。

有好多人讀緊書果陣已經要出嚟做pt,叫比啲零用錢自己洗

但係其他同學可能係冇經濟負擔,專心學業或者發展自己興趣

而窮人細路要發展自己興趣?首先要自己出去打工,想問屋企人攞錢根本冇可能,因為平時去茶記加兩蚊嗌凍飲/去麥記要加大,老豆老母都會屌,更何況去上咩興趣班或者學咩樂器。係街頸渴想買野飲都唔比,老母會叫你返屋企飲水

窮人細路大個左之後一樣打份牛工,搵雞碎咁多,唔敢結婚生仔 。睇新聞見到,原來人地老豆老母送比仔女嘅生日禮物可以係一層樓。望返自己,細個老母買個百零蚊嘅玩具比自己之後仲比說話我聽,又話貴又嫌呢樣嫌果樣

身邊大把人,由細到大都冇房,做咩都係個廳做,張床都係廳,冇私人空間

其實上一代只要有份正正常常嘅工,努努力力到而家都會有層供完嘅樓,亦可以照顧到自己起居飲食,唔需要仔女比家用。再勁d嘅,可以幫仔女比埋首期,甚至送層樓比佢。其實,老豆養仔 仔養仔 先係正常,而唔係養兒防老。如果你連養一個細路嘅條件都冇嘅話(講緊係照顧基本起居飲食,有舒適嘅成長環境),咁就唔該你唔好生啦,唔好害左個細路,細路係無辜。

去返窮人家庭,父母只會講已經比哂最好既野你,你仲想點?

但係你有冇問過自己,你比過啲咩仔女?

如果你話,供書教學,比飯你食

咁我想問,有邊個父母唔使?

呢啲係做父母嘅責任,點解可以講到皇恩浩蕩,講到好偉大咁?

如果你去領養,佢第時出人頭地,做人好開心好有意義,咁你咪偉大囉

如果唔係嘅話,你唔好講到自己幾咁辛苦,生仔係你地自己決定,要怨就怨你自己,係你地自己攞嚟

仔女變成點你地有責任,唔係咩都賴落人地到

可能你話,咁窮人細路比心機,追返啲時間咪得,努力向上游

但係事實上,係極難,因為由細到大嘅成長環境唔同,見識唔同,社交圈子都唔同。你睇下讀醫科讀法學入面有幾多係中產名校出身,有幾多屋邨仔屋邨中學出身就知。

食得咸魚抵得渴,冇咁大個頭就唔好戴咁大頂帽

講到尾,窮人唔好生仔

