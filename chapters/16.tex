\chapter{何謂分化}

何謂分化

1.分化是什麼?

「分化」其實好簡單,有一句說話可以解單總結。就係宇宙級革命家毛澤東所講既:

『所謂政治,就是把我們的人搞得多多的,把敵人搞得少少的。』

特別係最後一句「把敵人搞得少少的」,就已經可以概括曬中國共產黨咁耐以黎既「分化術」。

就是要將敵人多而化少,分而破之。將敵人由200萬變成100萬,100萬變成50萬。到最後孤立少數主要敵人,一舉擊破。

2.為何需要分化?

要了解「分化」,就首先要明白:點解要分化。

世上有無限咁多兵器,飛機大炮,刀劍槍械,仲有嚴刑峻法。你點解要諗到用分化?

好簡單,因為如果你面對既敵人,係無辦法用上述武力同法制解決既,就唯有用「分化」。

『把我們的人搞得多多的,把敵人搞得少少的。』--依句說話據講係毛澤東係延安時期寫低既(待考據,可代補充),延安時期大致橫跨抗日戰爭去到國共內戰。共產黨以延安為根據地,毛澤東同時面對緊國民黨、日軍、再加埋黨內唔同派系、甚至蘇共力量既壓力。

對毛sir黎講,軍力唔夠、人唔夠、甚至自己友唔夠,點做?分佢老母化。

當你既敵人係一大班群眾,你無辦法殺曬咁多人,拉曬咁多人。武鬥不行,就唯有文鬥,其中最強技術,就係「分化」。

3.分化的原理:敵意

去到依度,你或者已經諗到,到底分化係咩黎。其實好簡單,一幅圖可以表達曬:

係咪好簡單呢哈姆大郎。其實「分化」,最重要既只有一樣野:敵意。

聽落好似好玄妙,但其實大家由細到大都一定有經歷過類似既樣野:

A:我鐘意玩PS

B:我鐘意玩XBOX

C:on9仔先玩XBOX,A,我都係玩PS,你以後同我玩啦,唔好同B玩啦,你唔覺B ON9咩

A:屌,我一直都覺,佢打J成日睇三上悠亞,我覺得河合明日菜先正

假設依個世界唔存在PS/XBOX之間既分界線,A同B係可以共存,而C亦唔能夠勾起A既敵意,去離開B。

由此可以再推導更深一層既分化術原理:

事物有差別唔代表對立,但強調事物有差別,可以引致差別之間既敵意,進而分化。

面對一個團結而難攻既群體,只要有人劃底一條線,而團體入面所有人都無辦法消除依條線,甚至不斷討論到底條線既位置係邊(唔一定好似幅圖咁係正中心),引發兩派人之間既敵意。

到最後,依一個圓型/群體,就一定會被條線分割。換言之,姐係割席。

4.如何實際執行分化?

上面係理論野,講下實際野。

現實生活唔係個都玩PS同XBOX(我兩樣野都無),咁你點樣分化一大班人,甚至分化一班本身有共同目標既人?條線應該擺邊?

好簡單,當對方只有一個目標,咁你就要強行演釋目標既差別,然後強調差別既存在。

例如,有20個人都想去日本旅行。

本來打算一齊出發,結果有第21個人因為各種原因(妒忌/無錢/反日...etc),而唔想依班人去得成日本。

佢就開始同20人入面既第19個人講:「其實呢,我覺得識去日本既,一定係去大阪,東京真係唔抵去。你覺得呢?」

第19個人未必對大阪或者東京有意見,但當佢拎出黎同20人討論既時候:

關西派既第11人就開始吹奏大阪;

第6人係東京派,佢就開始反對第11人既意見;

第13人同第11人係朋友,佢就會幫口反對第6人;

第7人一直暗戀第6人,佢就又會幫手屌第13人同第11人:

結果,依20人最後要分開兩批人去,去大阪個10人,又因為訂唔切酒店而延期,只得去東京既一半人去到。

第21人又再出現了,同去到東京既個10個人,入面既其中1人講:

「其實呢,我覺得去識去東京既,一定係去原宿,涉谷真係唔抵去。你覺得呢?」

--直到最後得番1個人。

第21就同最後1個人講:係囉,去咩日本姐,日本on9仔先去。

然後就無人再去到日本了。

聽落好似好荒謬,但如果真係發生,你會唔會有少少心寒?

因為第21人佢咩都唔洗做,唔洗子彈唔洗槍,就可以達成目的。

現實生活入面,情況當然唔會咁單純,你要分化對方,亦唔係一句「你覺得呢」就可以解決。

但無論任何行動、目標、想法,其實都有無限演繹方式。

只要你拎住其中一邊,再強調另一邊既差別,勾起兩派人既對立同情緒,就可以逐步瓦解對方,把「敵人搞得少少的」。

再引一段黎自毛澤東既文章《矛盾論》,解釋如何瓦解敵人:

「《水滸傳》上宋江三打祝家莊,兩次都因情況不明,方法不對,打了敗仗。後來改變方法,從調查情形入手,於是熟悉了盤陀路,拆散了​​李家莊、扈家莊和祝家莊的聯盟,並且佈置了藏在敵人營盤裡的伏兵,用了和外國故事中所說木馬計相像的方法,第三次就打了勝仗。列寧說:要真正地認識對象,就必須把握和研究它的一切方面、一切聯繫和『媒介』」

5.「我」係咪分化撚?

你或者會問:喂,咁東京真係唔抵去喎,我又真係覺得玩XBOX既真係on9(再次強調,我兩樣野都無),咁唔通我屈就自己?

唔通我提出相反既意見,唔通我唔同意某個想法,我就係分化撚?

答:仲記唔記得,上面講到,點解要分化敵人?

因為敵人係無辦法用武力解決既大型群體,目標一致會影響我方既利益。

咩係「敵人」?

敵人唔係你唔鐘意、你睇唔順眼既人、唔同意你意見既人,而係會影響你實際利益既人,同你爭女/仔,同你爭機位,同你爭奪政權既人。

你分化佢,就令佢無辦法同你爭奪利益,令佢失去敵人既資格。

調轉講,即使你係團體入面持有相反意見,如果大家仲係目標一致,仲係一個可以威脅對家既團體,不論你對東京有咩意見,你再鐘意玩XBOX,只要群體仲可以一齊行動,你提出再多既反對意見,你都唔係分化撚。

6.「我」/佢」係咪會引致分化?

分化點解咁勁,毛澤東點解用足一世,當然唔係咁簡單。

因為人人都可以被分化而不自知。

先問下大家一個問題,如果依個世界得兩個人,你同我,咁你點樣分化我?

無錯。係分化唔到。

要分化,就要有互相對立既敵意。

我對我自己點樣有敵意?你先搞到我人格分裂可能得既。

要分化團體,一隻手掌係拍唔響。

「分化」最無敵之處,係一班人入面,只要令有一個人或少數人有敵意(好似上面既日本旅行團例子),個份情緒就好容易散佈,令成個團體受感染。

甚至你本身即使唔係想分化人,只係想做和事佬,出黎講句:

「起,唔緊要啦,咪去完東京再去大阪。」

都可能會被關西派既人屌:

「點解一定要去東京先再去大阪先?明明大阪近啲呀依家。你係咪東京派既分化撚!?」

如果你沉不住氣,屌番轉頭,咁笑到最後既,就只有唔想大家去日本既第21人。

所以:分化撚唔一定真係敵人,只要有敵意,引致分化既,一樣可以係你同我。

7.如何防止分化?

咁係咪已經玩撚完?

或者你會覺得,分化咁勁,大家中左招都唔知,仲可以點?

毛澤東成世人經歷過大大小小的黨爭,係幾乎戰無不勝,仲要笑到最後。精通分化,係咪就真係天下無敵?

的確,只要對方係一個群體,而目標有多種演繹方式,當我方派出唔止一個分化撚,只要手段恰當,思路清晰,幾本上係無得輸。

當劃出對立既界線,帶動敵意,敵人只要on99生勾勾,開始互鬥,鬥顏色夠深,鬥接近中心,就會被分化。

由圓形變成半圓,由半圓變成四分一圓,再變到屍骨無存。

不過,天下間,無野係完美同無缺陷既,分化都係。

分化既弱點只有一個:胸襟。

唔係依啲胸襟,不過見你肯睇到依度,獎勵下你姐。

回歸正題,面對分化,唯有胸襟可以抵抗,要練功必先練心,

唯有胸襟,先可以化解分化既核心元素:敵意。

你要接受:

所有目標都唔會只有一個途經

所有行動都唔會只有一個方法

所有方法都唔會只有一個效果

所有理念都唔會只有一個演繹

所有目的地都唔會只有一條路

唔同人有唔同既意志

唔同人有唔同既能力

唔同人有唔同既專長

唔同人有唔同既手段

唔同人有唔同既表達方式

大家有唔同既口才、想象力、行事風格、做事節奏。你行得慢,我行得快,你鐘意坐車,我鐘意坐飛機。你鐘意三上悠亞,我鐘意河合明日菜。你鐘意去東京,我唔單止唔鐘意大阪,仲比較鐘意去歐洲--但唔緊要,我地既目標仍然一致,無野分化到我地。

一個群體之所以令敵人無計可施,要不斷分化,正因為團體入面既人,有胸襟去兼容所有既可能性。

就算有任何撚屌係團體入面劃一條線,你都有胸襟去跨越依條線,無視敵意既起源,而唔係將條線另一邊既人,當係仇家,亦唔再當佢係同路人。

只要記得目標,記得令團體之所以為團體既意志,有胸襟去包容唔同意見既人,就可以令「分化」無用武之地。

記住,目標。意志。胸襟。

兄弟爬山

各自努力

