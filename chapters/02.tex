\chapter{我阿哥成為󱃡我同老豆老母之間的隔閡}

(放負長文慎入) 

我阿哥成為󱃡我同老豆老母之間的隔閡

小弟一家四口,老豆老母阿哥同我。

老母係傳統家庭主婦。

嫁󱃡畀老豆就冇打工。

老豆一個人養起頭家。

屋企背景都簡單正常冇乜問題。

直至阿哥讀唔到書中五畢󱃡業。

周圍打散工冇乜大志。

渾渾噩噩咁轉工快過揭書。

而我有幸搵到份正常寫字樓工。

有穩定󱝚收入同前景。

基本上我冇嘢要老豆老母擔心。

年前阿哥中出左條女做人老豆。

老豆老母攞晒積蓄出嚟比阿哥買樓。

等佢結婚組織家庭。

亦因為咁老母同我講屋企宜家住緊層樓。

會畀我同老婆一齊住。

我老婆都好好肯同我老豆老母同住
        
幫輕我老母做家頭細務打理頭家大小事。

本來都相處得好融洽。

但近幾年疫情問題

阿哥失業搵唔到錢供唔起層樓

阿嫂份糧拎晒嚟養個仔同照顧自己娘家。

最後佢地賣左層樓

同我老母講要一家三口搬返嚟屋企。

以我所知賣樓舊錢冇畀返老豆老母。

換言之變󱃡我同老婆冇地方住。

老母同我講我兩公婆都有份叫做穩定󱝚工。

但阿哥收入唔穩定加上有個仔要養。

希望我同老婆搵地方搬走。

體諒吓阿哥困境。

仲話相信我哋他朝一定可以靠自己能力上車。

我同我老婆都覺得好失望。

原來一個人無用就可以理直氣壯咁攞晒所有資源。

有能力󱝚人就理所當然乜都要靠自己。

咁多年嚟自問做仔冇要過老豆老母擔心。

好好地做人讀完書打份正經工。

相反阿哥長期要老母照顧勞氣傷神。

估唔到最後老母都決定叫我走。

除󱃡心灰意冷都唔知仲有乜可以講。

我有幸娶到我老婆。

佢叫我唔洗擔心點同外母交代。

一切交比佢處理。

仲安慰我話趁後生捱幾年等機會希望可以上車。

雖然我都知租地方洗費大要草錢難上加難。

樓價又持續高企我哋󱝚環境簡直雪上加霜。

我都唔知點面對外母。

外母對我好似自己仔咁。      

每次上去食飯都成枱餸。

個女嫁󱃡比我無成成要租樓住。

老婆有晚好坦白同我講佢好憎我阿哥。

對我老豆老母呢個決定都好心淡。

我又點會唔明白。

懂事開始自問負責照顧老豆老母。

呢個責任冇旨意過阿哥同我分擔。

佢亦都冇分擔過。

我都想做個乖仔孝順仔。

但呢刻我覺得我自己先係比全家遺棄󱝚一個。

呢個家已經唔再係我󱝚家。

