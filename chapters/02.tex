\chapter{我阿哥成為咗我同老豆老母之間的隔閡}

(放負長文慎入) 我阿哥成為咗我同老豆老母之間的隔閡

小弟一家四口(老豆老母阿哥同我)

老母係傳統家庭主婦

嫁咗畀老豆就冇打工

老豆一個人養起頭家

屋企背景都簡單正常冇乜問題

直至阿哥讀唔到書中五畢咗業

周圍打散工冇乜大志

渾渾噩噩咁轉工快過揭書

而我有幸搵到份正常寫字樓工

有穩定嘅收入同前景

基本上我冇嘢要老豆老母擔心

N年前阿哥中出左條女做人老豆

老豆老母攞晒積蓄出嚟比阿哥買樓

等佢結婚組織家庭

亦因為咁老母同我講屋企宜家住緊層樓

會畀我同老婆一齊住

我老婆都好好肯同我老豆老母同住

幫輕我老母做家頭細務打理頭家大小事

本來都相處得好融洽

但近幾年疫情問題

阿哥失業搵唔到錢供唔起層樓

阿嫂份糧拎晒嚟養個仔同照顧自己娘家

最後佢地賣左層樓

同我老母講要一家三口搬返嚟屋企

以我所知賣樓舊錢冇畀返老豆老母

換言之變咗我同老婆冇地方住

老母同我講我兩公婆都有份叫做穩定嘅工

但阿哥收入唔穩定加上有個仔要養

希望我同老婆搵地方搬走

體諒吓阿哥困境

仲話相信我哋他朝一定可以靠自己能力上車

我同我老婆都覺得好失望

原來一個人無用就可以理直氣壯咁攞晒所有資源

有能力嘅人就理所當然乜都要靠自己

咁多年嚟自問做仔冇要過老豆老母擔心

好好地做人讀完書打份正經工

相反阿哥長期要老母照顧勞氣傷神

估唔到最後老母都決定叫我走

除咗心灰意冷都唔知仲有乜可以講

我有幸娶到我老婆

佢叫我唔洗擔心點同外母交代

一切交比佢處理

仲安慰我話趁後生捱幾年等機會希望可以上車

雖然我都知租地方洗費大要草錢難上加難

樓價又持續高企我哋嘅環境簡直雪上加霜

我都唔知點面對外母

外母對我好似自己仔咁

每次上去食飯都成枱餸

個女嫁咗比我無成成要租樓住

老婆有晚好坦白同我講佢好憎我阿哥

對我老豆老母呢個決定都好心淡

我又點會唔明白

懂事開始自問負責照顧老豆老母

呢個責任冇旨意過阿哥同我分擔

佢亦都冇分擔過

我都想做個乖仔孝順仔

但呢刻我覺得我自己先係比全家遺棄嘅一個

呢個家已經唔再係我嘅家

