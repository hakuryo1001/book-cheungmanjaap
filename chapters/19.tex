\chapter{其實香港人係有套共同信仰,當呢個信仰動搖係會開始懷疑人生}

其實香港人係有套共同信仰,當呢個信仰動搖係會開始懷疑人生

呢篇文我一路諗一路打冇組織過,所以好難睇,如果你仲願意睇可以一路睇一路思考,持開放態度一起討論幫我補充。

——————————-

其實香港人係有套共同信仰,當呢個信仰動搖係會開始懷疑人生意義。

大家由細到大都被灌輸一套社會信仰,只管信不敢思考,過於思考會不快樂會被歸類為離地不成熟等等。

簡短講下呢一套信仰:

每個人一出世就被賦予階段性任務,首先要完成12年免費教育,填鴨式學習做練習狂考試,考小考中學考大學。只要考得到只要考得好就高人一等,將來會有好既成就。大家,包括家長老師同學都唔會思考呢套教育制度係咪真係可以幫助大家發展長處,即使有人大膽提出就會比人話 “不嬲係咁架啦” “一向都行之有效” “諗埋啲無謂野對你有咩用?” 大家只管信就得!

成長過程中被灌輸,父母同老師係絕對正確,警察係好人,政府收左稅處處為市民服務,讀唔到書唔會成功,你一定要跟住有權威既人既指令去行。”我食鹽多過你食米” “個個都係咁㗎啦” 就係咁樣大家就只管信不要問!

專業人士應該係備受尊重,因為讀書讀得多成績好所以叻過曬讀唔成書既人。警察政府應該備受尊重,因為佢哋係正義係為我地好。身為警察既自己又會覺得着起套制服行出黎就應該得到市民尊重!呢一切大家又只管信,然後用盡人生頭20-30年想做專業人士打政府工或者化身正義之士。無人問呢一切標籤係咪真,但你不用思考只管信吧!

讀完書出到社會,得到新任務 “買樓 結婚 生仔”

出到社會後香港人又相信工作就係為左買樓,買樓好重要,無樓就結唔到婚,生仔好重要唔生唔得。無人問點解,大家只管信!即使有人大膽問:香港樓咁貴唔合理喎做樓奴供成世,日日返工OT無錢收,唔OT怕比人炒,為左間豆膶咁細既屋做半生機械人,咁嘅人生有咩意義?成個系統好似好唔個理喎!得到嘅回覆係“個個都係咁㗎啦” “呀邊個邊個咪成功囉” “努力啲一定得架” “成熟啲啦唔好咁離地” 你信啦!個個都係咁做架!

香港人嘅人生意義完全基於呢啲信仰,大家從來冇諗過人生為乜野而活,因為跟本無咁既時間去諗。或者咁講,因為大家個腦已經容納左一大套信仰,再冇思考空間,就算有人開始思考開始提出問題都會被嘲笑被喝止。”一路都係咁架啦!” “從來行之有效你諗咁多做乜?”

過去幾個星期,香港人都進化左,大家開始知道一直相信既事情唔係真相,父母老師唔係絕對正確,唔係全部警察都係好人,政府唔係處處為人民着想,專業人士都好多蠢人衰人,而身為專業人士都唔代表會備受尊重等等⋯

大家開始對呢套行之有效既信仰懷疑,亦即開始清醒。但係同時又好似失去緊人生嘅意義。有人會崩潰想結束生命以死控訴,有人會死守信仰排斥覺醒嘅人,有人會清醒過來開始思考人生意義開始思考到底乜嘢係岩乜野錯。

最後我想講,即使你已經崩潰千祈唔好結束生命,我地要不斷思考,不斷追求真理,進化過程係痛苦但總係會有同路人一齊進化,你不孤單!

