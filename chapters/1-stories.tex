昔傳,萬川山陝,內藏奇國,䋛盛瓊麗,有䭁目之俗。民以目為見官,見為智之端,故民奉目予王食,以明其聖賢,壯其國勢,保民福祉。是以食目為國俗,是故國名瞭。



從前,喺一個山脈連綿嘅一個山卡罅地方,有一個國家。呢個國家頗為有錢,都可以叫做大國。大國佢文明頗為發達,衣服華麗,飲食講究,禮儀繁多,法公律義,人民彬彬有禮,說話有理則,議論求真義,詩詞有邃情,文章黼黻。坐喺山谷嘅大國,四周圍都見唔到文明同自己相若嘅其他邦國。人民都為佢哋自己嘅成就感到自豪。



咁喺離呢一個大國冇幾遠嘅地方,就有咗一個小邦城。佢哋未至於細到淨係得條村,亦唔係淨係得一個城咁大把,所以都可以姑且叫做小國咁。呢個小國唔係太有錢,物質文明唔太發達,啲國民好純樸,思想簡單捷率。佢哋人民嘅生活水平未算差劣,但又未算話人人有錢,只可以話係咁意搵到兩餐。佢哋算叫做食得好,未可以日日大魚大肉,但粗茶淡飯又一餐之後都可以有魚有雞。佢哋都著得好,未算個個大紅大紫錦衣當麻布,但至少冬天着得暖,出得街見得人。住嘅話,房屋都夠大可以成家立室,冬暖夏涼,小康之家林立。佢哋冇咩科技發展,所以發展都幾緩慢。佢哋自成一國,與世無爭。



大國,覺得佢哋咁強大,咁富足,眼見都喺佢哋隔籬嘅小國冇自己咁掂,戥佢哋唔抵,甚至乎覺得自己有責任去幫下佢哋。佢哋諗下諗下就覺得:啊,不如將佢哋納入我哋嘅版圖啦。我哋可以繼而喺經濟同政治上照顧佢哋,我哋可以分享我哋科技、經濟、同文明上嘅成果,關照關照。另外一方面,如果佢哋有啲咩資源或者啲咩知識嘅話,又可以喺我哋嘅國家度貢獻黎扶持返我哋。相得益彰,雙贏,何樂而不為呢?



大國嘅國王同朝廷上下嘅文武百官謦啊謦,冇幾耐就拍板決定為呢個可憐嘅小國作佢哋作為鄰居應該做嘅嘢。



於是乎有一日,大國就拿拿臨派咗一隊兵同官吏,去咗小國嗰度,提出佢哋嘅兩國合併嘅建議。冇兩句,小國就畀大國吞併咗。



個小國都冇乜說話,橫掂自己又唔係特別富強,又唔係零舍有米,而家有一個比自己勁,比自己先進識撈嘅國家嚟,都咪話唔話有少少接濟幫助嘅感覺。所以都冇乜異議。日子開初嗰陣根本完全唔覺得有改變,統治咗一陣,可能有啲小改變,但係都係好嘅改變,冇咩係搞得唔掂令佢惹人口實。個小國嘅人民都冇咩說話仲覺得幾滿意,甚至覺得當初被呢個大國吞併咗有少少開心,係恩賜。



之不過,個小國嘅人殊不知嘅係,大國有一個好特別嘅文化教條:大國嘅人,上至國王宰相文武百官,下至市井僂儸文盲咕喱,人人都好受一套嘅諗法。佢哋認為,人嘅眼睛,係一個好犀利嘅器官。佢哋認為,人嘅眼睛,睇到嘢,俾人知曉到世間中嘅萬物同機理,俾人獲得到知識,智慧,哲知—眼睛,對佢哋來講,就係智慧嘅開端,係最犀利嘅知官,有不可言喻,神秘嘅能力。大國嘅人從呢一度出發,就慢慢歷史地形成咗一個國家共識,繼而化之成為咗一個習俗。佢哋就覺得,為咗國家發展同人民幸福,佢哋嘅國王,就梗係要有咁有智慧得咁有智慧—咁先至可以將佢哋嘅國家打理得最好㗎嘛。既然眼睛咁犀利,可以賦予人智慧,就應該透過眼睛,嚟將智慧放喺國王身上。所以,就每一日,國王都會喺早朝之前,舉行一個小典禮。滿朝文武百官會望住國王,等一個官吏呈上一個金盤。擺咗金盤上面嘅,係一對人眼。國王就會喺呢個典禮度,將呢一對眼睛食落肚,以補充同增益國王自己嘅智慧。大國嘅人民都好真誠地、虔誠地深信呢個習俗嘅功效,亦可堅持地秉持住呢個國策同佢背後嘅理念,甚至乎會爭先恐後為國獻上自己嘅眼睛。佢哋覺得可以為國王奉獻上自己嘅知官,是為奉微,卻光榮無比。佢哋亦都相信,佢哋之所以咁強大,文明咁發達,係因為佢哋將呢一個嘅機制當國俗。個大國,就係喺咁樣嘅基礎上發展出嚟。



荏苒未幾,個大國後尾派咗一小隊人去個小國度。個小國嘅人問佢哋有何貴幹,何事大駕光臨。大國嘅官吏就話,大國同小國合併已久,小國享受咗大國嘅恩澤同庇蔭,應該一同負擔建設當中嘅成本。國王嘅智慧要增益,就要食眼睛,小國嘅人民要奉獻一個人嘅眼睛畀國王佢食,等佢可以繼續智慧無邊,畀陛下佢好好管理國家,大家就可以繼續發展,繁榮安定。



小國嘅人民聽到呢個嚟自佢哋宗主國嘅官吏咁講,仲要講嗰陣語氣不顫,完全唔覺得自己荒謬恐怖,直筆甩毫無猶疑停頓講完之後顏色不變,小國嘅人民嚇到六神無主、 魂飛魄散。覺得最恐怖嘅就係,佢哋見到大國嘅人見到佢哋嘅反應,唔單只係冇同情或者理解,反而係有一種不以為然,覺得食眼睛補智慧理所當然,對小國國民嘅嬲怒散溢住一種百思不得其解嘅錯愕。佢哋大國嘅人,完全冇一種侵略者嘅殺戮慾,冇想過血腥殺人舐舐脷嘅變態,反而係一種「而家有嘢益你」,你好我好大家好嘅氛圍。



小國嘅人齊聲反對,聲嘶力竭,嗌爆喉嚨反對啦,仲差啲搞到爆發暴力衝突,爭啲搞到啲農民同工人揦埋晒架撐抨晒佢扯,糾纏咗好耐。



但係大國嘅人,覺得咁小事搞到成個大頭佛出嚟,係因為小國嘅人,不幸文明落後,思維萌塞,不通情達理。佢哋唔通曉文明之道啫,唔明乜嘢先至係對自己好,所以先至會咁嘈冤巴閉。另一方面,佢哋亦覺得根本就冇咩大件事,小國嘅人根本就反應過份,有少少諸多事實,冇嘢搵嘢齮齕。如果佢哋真係安安靜靜坐低,平心靜氣理性諗諗,就會知道大國係有道理,咁樣發展先至係啱。於是,佢哋就覺得不如索性幫佢哋做主啦,繞過佢哋啦,橫掂佢哋遞時明白咗之後都會感恩我哋當年為佢哋做咗呢一個正確同啱嘅決定。話口未完諗到呢一點,大國嘅官吏就索性由佢哋作主揸弗,喺一個月黑風高嘅夜晚度,風聲颯颯,落木蕭蕭,大國嘅人就喺夜媽媽裡面擄走左一個細路,跟住就全部人一齊同佢返返去大國嘅首都,準備將佢嘅眼睛畀國王享用。



個細路被擄走咗之後,就喺大國嘅宮殿度畀人挖咗兩隻眼出嚟。對眼血淋淋,擺咗上金盤,鮮艷嘅血喺炅熠熠(粵音:炩臘臘)金漆漆嘅金盤上面啷咗兩啷,就呈上咗畀國王。國王「唂」一聲就將兩粒眼珠吞晒落肚。國王話佢好高興,好滿意,好恩惠,因為小國嘅人,而家都可以同佢哋一齊分享同建設國家,為幸福努力奮鬥。佢亦對塊面而家有兩個黑眯鼆咕窿嘅細路講,應該為佢作出嘅貢獻而感到自豪,大國同小國嘅發展越嚟越好。佢話,因為國王嘅睿智會越嚟越深邃,剖析國事嘅見解會越嚟越精闢,治理國家嘅駕馭能力會越嚟越成熟,國家好就大家好。朝廷上四周圍嘅大國官員將領妃子諫議士,紛紛點頭同意,笑顏滿面。喺朝廷上有啲之前奉獻過自己眼睛嘅人,都好真誠地流露出開心同感動嘅感情。空寥寥嘅眼袋,好似有感動之淚喺度晃動。事畢後,個細路被賞賜千金寶物錦繡,載滿晒成列嘅車馬。佢哋就浩浩蕩蕩連人帶禮,同成班官員同侍兒,返屋企,送兒還故鄉。



佢哋返到小國一刻,小國嘅國民就二話不說,即刻將隨團嘅官吏侍兒全部一個不留殺晒。



佢哋仲將佢哋嘅眼睛全部搲晒出嚟,然後推咗一座小山,擺咗喺通入小國嘅山路。帶返嚟嘅靚貨同寶物全部都燒毁晒揼晒落山。



消息冇幾耐就傳返去大國度。舉國驚訝驚愕無語,而公眾驚愕驚詫背後,所謂冷靜理性嘅,都拗爆頭,諗唔明點解小國嘅人會咁樣做,有咩意思,有咩含意。佢哋唔明,點解小國嘅人民會唔接受畀國王食眼精黎提升智慧呢一個嘅做法。眼睛有思哲嘅功效,大家貢獻眼睛畀國王食,食咗國王管治施政更加有智慧,政策更加優質有效,國家好大家好人人好,乜唔係好淺白直接嘅道理咩?點解會睇唔到呢一點呢?仲咁大反應添。諗下諗下,甚至開始唔信使團被殺清光呢件事,覺得根本冇可能會有人咁不可思議睇唔到食眼睛背後個咁明顯、咁不言自明嘅邏輯。



於是大國嘅人就派人去查證,雖然冇入到小國,但係就將嗰大咋嘅眼珠帶返返去。國王朝廷同文武百官見到嘅一刻,驚嚇無語,朝廷上下誠惶誠恐憤怒不惑。驚嚇同驚愕過後,就係憤怒同憤恨。我哋咁關照小國嘅人民,佢點可以咁做呢?點可以咁野蠻呢?點解可以咁㒼塞(粵音:文塞), 思想咁狹隘咁不化?點解咁明顯嘅道理都睇唔明,睇唔到大國所做嘅嘢背後嘅邏輯?



大國嘅大臣、官吏,諫議士同大學生同國王就呢樣嘢商討商討,冇幾耐之後就出咗條定案決定咗咁樣做。



佢哋出兵至小國。大國軍力強盛,小國不堪一擊,好快就冧低咗。擊破咗小國之後,大國將小國嘅所有人,男女老幼、公卿大夫、山林百姓,冚唪唥全部集埋晒一齊喺廣場度。大國對佢哋講,大國文明昌盛發達,胸襟浩瀚,唔會計較前嫌。大國會向小國彰顯大國應有嘅恩澤大道,以德報怨,以仁待民。佢哋話,小國之所以咁樣做,係因為小國落後,唔明大道理,所以唔會怪罪小國,亦唔會懲罰,更反而會幫佢哋喺文明程度上作出全面嘅提升,等小國佢哋都可以喺大國文明、先進、前衛,同正確的角度同觀點度睇到世間萬物玄義。咁樣,大國同小國就可以一齊同心協力,共創新天。



講畢後,大國就將小國嘅所有人,男男女女,上至華髮老人,下至初生蘇蝦同五歲細蚊仔,一個不留將佢哋眼睛全部挖晒出嚟,然之後餵返晒比佢哋自己食,等佢哋可以一享食眼睛所帶嚟嘅智慧。