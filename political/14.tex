\chapter{【以香港作為信仰】}

https://lihkg.com/thread/2335214/page/1

\#1香城法絲



【以香港作為信仰】

國安法後抗爭進入休整期,抗爭陣營充斥住無力感,出現好多內部矛盾同互相指責。勝出民主派初選嘅張可森曾寄語希望大家以「香港作為信仰」,雖然信仰對香港人嚟講好陌生,但共同嘅信仰正正係我哋呢一刻最需要嘅嘢。

信仰戰勝極權人

擁有信仰唔代表係信奉某種宗教或神明,宗教只係其中一個表現信仰嘅方式。信仰係一種我哋生而為人,選擇相信嘅價值觀;可以係生活方式,亦可以係人生意義。喺極權統治下,人民失去集會自由同表態嘅空間,難以透過參與群眾運動凝聚力量,個體都變得非常脆弱。縱觀歷史長河,好多經歷過極權統治嘅民族都各有宗教信仰,幫助佢哋喺苦難中維繫群眾。猶太人散落世界各地,因為猶太教義而決心復國;波蘭人以天主教作為後盾,喺共產統治下保存到文化同傳統;藏人信奉藏傳佛教,不怕犧牲同中共戰鬥到底。極權冇辦法奪去人民內心嘅信仰,信仰係人民對抗極權嘅希望。

為信仰而抗爭

當街頭抗爭稍為靜止,有人會為表面抗爭人數嘅減少而灰心喪志,用冷嘲熱諷嘅方式指責其他人付出唔夠多,有從眾同比較心態。信仰係每個人內在嘅信念,反而唔會輕易受外在因素影響。當抗爭漸漸失去畫面同關注,就係真正考驗信仰嘅時候。我哋甚少見基督徒會因信徒嘅多寡而自怨自艾,甚至放棄信仰。哪怕世上只剩下一位信徒,佢哋仍然視傳福音為使命。當你立志以抗爭為使命,就無需太在意民智嘅高低、計較各人付出嘅多少。抗爭唔係因為香港人值得與否,而係要對得住自己嘅身份;堅持亦唔係因為從眾,而係為自己心中嘅信仰。

相信香港呢片土地

面對極權,有人選擇移民,並經常引用猶太人做例子,表示離開唔代表放棄抗爭。猶太人作為離散民族,飄泊流亡過千年,終能復國全因為有共同信仰去維繫自身共同體。佢哋堅信猶太教義中復國嘅預言,即使散落各地都仍然心繫以色列本土。若以香港作為信仰,「煲底之約」就等同於香港人信仰嘅預言。世上只有一個煲底,無論係流亡海外定係身陷囹圄,我哋都期待煲底下除罩相擁嘅一日。移民只係手段,大家追求嘅係喺香港呢片土地上實現民主自由,而非到外國去享受自由嘅果實。維繫香港人嘅係紮根本土嘅信仰,就算被迫要逃離家園,都要確信香港呢片土地係唯一安身立命之所,係信仰嘅應許之地,別無他處。

相信每一個香港人

有人擔心香港人會放棄抗爭,背棄曾經相信嘅價值,但信仰能透過個人經歷得以鞏固;有人覺得雨傘後難再有大型抗爭,結果五年後香港人成就咗一場波瀾壯闊嘅革命。信仰嘅種子早已透過共同經歷栽種喺我哋心中,並會隨着實踐中萌芽成長。以香港作為信仰、相信香港人依個共同體,過去一年嘅抗爭足以見證香港人嘅勇敢、智慧同團結。不論你、我、他或她,早已成為香港嘅信徒,一息尚存,抗爭到底。

相信香港終會重光

缺乏信仰嘅人容易迷失,因為太過著眼於最後嘅成敗得失。信仰嘅強大在於即使睇唔到結果,你都願意繼續相信。正如基督徒唔會質疑耶穌會否再臨,只係未知幾時重臨,信徒應該著眼於實踐信仰嘅過程,努力去傳福音。同樣地,我哋要堅信香港終會重光。如果連香港人自己都唔相信會贏,試問又點可能實現夢想呢?喺黎明來到之前,作為香港人要莊敬自強,生活上無時無刻緊記要實踐信仰。哪怕再漫長嘅黑暗,擁有信仰嘅香港人都能化為一點燭光,照亮身邊孤單困倦嘅手足,共同迎來重獲新生嘅香港。

用一生證明香港嘅存在

假若你缺乏宗教信仰,又或者對香港未來充滿疑惑同困倦,請別輕言放棄。懇請你再次相信香港,相信香港人。將過去一年共同經歷嘅苦難、手足之情同願景,視為一種我哋甘願為此抗爭、堅持同犧牲嘅信仰。

巴勒斯坦裔學者 Edward Said 講過巴勒斯坦人嘅命運,係要用一生去證明巴勒斯坦嘅存在。作為香港嘅信徒,我希望可以用一生去證明香港嘅存在。只要我哋仍然相信香港,香港就會存在。

在晚星墜落徬徨午夜,願各位堅守香港人嘅信仰,煲底見。

作者:良善如你

