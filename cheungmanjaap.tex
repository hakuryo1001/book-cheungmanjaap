\documentclass[a5paper, 10pt, openany]{book} % A5 paper size

% \usepackage[paperwidth=148mm, paperheight=210mm, top=1.27cm, bottom=1.27cm, inner=1.9cm, outer=1.27cm, headsep=1.5cm, footskip=1.75cm]{geometry} % Custom dimensions and margins
% Adjusted margins to ensure space for page numbers
\usepackage[paperwidth=148mm, paperheight=210mm, 
         top=1.27cm, bottom=2.5cm, inner=2.2cm, outer=1.27cm, 
         headsep=1.5cm, footskip=1.5cm]{geometry}


\usepackage[utf8]{inputenc}
\usepackage{ctex}

\usepackage{fancyhdr}
\pagestyle{fancy}
\fancyhf{} % Clear all header and footer fields
% \fancyfoot[C]{\thepage} % Center the page number at the bottom
% Define left and right page numbering
\fancyfoot[LE]{\thepage} % Left side for even pages
\fancyfoot[RO]{\thepage} % Right side for odd pages
% Ensure the chapter pages (plain style) also have this layout
\makeatletter
\let\ps@plain\ps@fancy
\makeatother

\input{preamble/main.tex}


% *-----------------------------------------------------------------------*
\begin{document}
% to avoid overfull hbox
\sloppy
% % \jcz{} must be run so the document can process jyutcitzi 
\jcz{}







% % \jczSourceHan{}

\tableofcontents

\input{chapters/0-sample.tex}
昔傳,萬川山陝,內藏奇國,䋛盛瓊麗,有䭁目之俗。民以目為見官,見為智之端,故民奉目予王食,以明其聖賢,壯其國勢,保民福祉。是以食目為國俗,是故國名瞭。



從前,喺一個山脈連綿嘅一個山卡罅地方,有一個國家。呢個國家頗為有錢,都可以叫做大國。大國佢文明頗為發達,衣服華麗,飲食講究,禮儀繁多,法公律義,人民彬彬有禮,說話有理則,議論求真義,詩詞有邃情,文章黼黻。坐喺山谷嘅大國,四周圍都見唔到文明同自己相若嘅其他邦國。人民都為佢哋自己嘅成就感到自豪。



咁喺離呢一個大國冇幾遠嘅地方,就有咗一個小邦城。佢哋未至於細到淨係得條村,亦唔係淨係得一個城咁大把,所以都可以姑且叫做小國咁。呢個小國唔係太有錢,物質文明唔太發達,啲國民好純樸,思想簡單捷率。佢哋人民嘅生活水平未算差劣,但又未算話人人有錢,只可以話係咁意搵到兩餐。佢哋算叫做食得好,未可以日日大魚大肉,但粗茶淡飯又一餐之後都可以有魚有雞。佢哋都著得好,未算個個大紅大紫錦衣當麻布,但至少冬天着得暖,出得街見得人。住嘅話,房屋都夠大可以成家立室,冬暖夏涼,小康之家林立。佢哋冇咩科技發展,所以發展都幾緩慢。佢哋自成一國,與世無爭。



大國,覺得佢哋咁強大,咁富足,眼見都喺佢哋隔籬嘅小國冇自己咁掂,戥佢哋唔抵,甚至乎覺得自己有責任去幫下佢哋。佢哋諗下諗下就覺得:啊,不如將佢哋納入我哋嘅版圖啦。我哋可以繼而喺經濟同政治上照顧佢哋,我哋可以分享我哋科技、經濟、同文明上嘅成果,關照關照。另外一方面,如果佢哋有啲咩資源或者啲咩知識嘅話,又可以喺我哋嘅國家度貢獻黎扶持返我哋。相得益彰,雙贏,何樂而不為呢?



大國嘅國王同朝廷上下嘅文武百官謦啊謦,冇幾耐就拍板決定為呢個可憐嘅小國作佢哋作為鄰居應該做嘅嘢。



於是乎有一日,大國就拿拿臨派咗一隊兵同官吏,去咗小國嗰度,提出佢哋嘅兩國合併嘅建議。冇兩句,小國就畀大國吞併咗。



個小國都冇乜說話,橫掂自己又唔係特別富強,又唔係零舍有米,而家有一個比自己勁,比自己先進識撈嘅國家嚟,都咪話唔話有少少接濟幫助嘅感覺。所以都冇乜異議。日子開初嗰陣根本完全唔覺得有改變,統治咗一陣,可能有啲小改變,但係都係好嘅改變,冇咩係搞得唔掂令佢惹人口實。個小國嘅人民都冇咩說話仲覺得幾滿意,甚至覺得當初被呢個大國吞併咗有少少開心,係恩賜。



之不過,個小國嘅人殊不知嘅係,大國有一個好特別嘅文化教條:大國嘅人,上至國王宰相文武百官,下至市井僂儸文盲咕喱,人人都好受一套嘅諗法。佢哋認為,人嘅眼睛,係一個好犀利嘅器官。佢哋認為,人嘅眼睛,睇到嘢,俾人知曉到世間中嘅萬物同機理,俾人獲得到知識,智慧,哲知—眼睛,對佢哋來講,就係智慧嘅開端,係最犀利嘅知官,有不可言喻,神秘嘅能力。大國嘅人從呢一度出發,就慢慢歷史地形成咗一個國家共識,繼而化之成為咗一個習俗。佢哋就覺得,為咗國家發展同人民幸福,佢哋嘅國王,就梗係要有咁有智慧得咁有智慧—咁先至可以將佢哋嘅國家打理得最好㗎嘛。既然眼睛咁犀利,可以賦予人智慧,就應該透過眼睛,嚟將智慧放喺國王身上。所以,就每一日,國王都會喺早朝之前,舉行一個小典禮。滿朝文武百官會望住國王,等一個官吏呈上一個金盤。擺咗金盤上面嘅,係一對人眼。國王就會喺呢個典禮度,將呢一對眼睛食落肚,以補充同增益國王自己嘅智慧。大國嘅人民都好真誠地、虔誠地深信呢個習俗嘅功效,亦可堅持地秉持住呢個國策同佢背後嘅理念,甚至乎會爭先恐後為國獻上自己嘅眼睛。佢哋覺得可以為國王奉獻上自己嘅知官,是為奉微,卻光榮無比。佢哋亦都相信,佢哋之所以咁強大,文明咁發達,係因為佢哋將呢一個嘅機制當國俗。個大國,就係喺咁樣嘅基礎上發展出嚟。



荏苒未幾,個大國後尾派咗一小隊人去個小國度。個小國嘅人問佢哋有何貴幹,何事大駕光臨。大國嘅官吏就話,大國同小國合併已久,小國享受咗大國嘅恩澤同庇蔭,應該一同負擔建設當中嘅成本。國王嘅智慧要增益,就要食眼睛,小國嘅人民要奉獻一個人嘅眼睛畀國王佢食,等佢可以繼續智慧無邊,畀陛下佢好好管理國家,大家就可以繼續發展,繁榮安定。



小國嘅人民聽到呢個嚟自佢哋宗主國嘅官吏咁講,仲要講嗰陣語氣不顫,完全唔覺得自己荒謬恐怖,直筆甩毫無猶疑停頓講完之後顏色不變,小國嘅人民嚇到六神無主、 魂飛魄散。覺得最恐怖嘅就係,佢哋見到大國嘅人見到佢哋嘅反應,唔單只係冇同情或者理解,反而係有一種不以為然,覺得食眼睛補智慧理所當然,對小國國民嘅嬲怒散溢住一種百思不得其解嘅錯愕。佢哋大國嘅人,完全冇一種侵略者嘅殺戮慾,冇想過血腥殺人舐舐脷嘅變態,反而係一種「而家有嘢益你」,你好我好大家好嘅氛圍。



小國嘅人齊聲反對,聲嘶力竭,嗌爆喉嚨反對啦,仲差啲搞到爆發暴力衝突,爭啲搞到啲農民同工人揦埋晒架撐抨晒佢扯,糾纏咗好耐。



但係大國嘅人,覺得咁小事搞到成個大頭佛出嚟,係因為小國嘅人,不幸文明落後,思維萌塞,不通情達理。佢哋唔通曉文明之道啫,唔明乜嘢先至係對自己好,所以先至會咁嘈冤巴閉。另一方面,佢哋亦覺得根本就冇咩大件事,小國嘅人根本就反應過份,有少少諸多事實,冇嘢搵嘢齮齕。如果佢哋真係安安靜靜坐低,平心靜氣理性諗諗,就會知道大國係有道理,咁樣發展先至係啱。於是,佢哋就覺得不如索性幫佢哋做主啦,繞過佢哋啦,橫掂佢哋遞時明白咗之後都會感恩我哋當年為佢哋做咗呢一個正確同啱嘅決定。話口未完諗到呢一點,大國嘅官吏就索性由佢哋作主揸弗,喺一個月黑風高嘅夜晚度,風聲颯颯,落木蕭蕭,大國嘅人就喺夜媽媽裡面擄走左一個細路,跟住就全部人一齊同佢返返去大國嘅首都,準備將佢嘅眼睛畀國王享用。



個細路被擄走咗之後,就喺大國嘅宮殿度畀人挖咗兩隻眼出嚟。對眼血淋淋,擺咗上金盤,鮮艷嘅血喺炅熠熠(粵音:炩臘臘)金漆漆嘅金盤上面啷咗兩啷,就呈上咗畀國王。國王「唂」一聲就將兩粒眼珠吞晒落肚。國王話佢好高興,好滿意,好恩惠,因為小國嘅人,而家都可以同佢哋一齊分享同建設國家,為幸福努力奮鬥。佢亦對塊面而家有兩個黑眯鼆咕窿嘅細路講,應該為佢作出嘅貢獻而感到自豪,大國同小國嘅發展越嚟越好。佢話,因為國王嘅睿智會越嚟越深邃,剖析國事嘅見解會越嚟越精闢,治理國家嘅駕馭能力會越嚟越成熟,國家好就大家好。朝廷上四周圍嘅大國官員將領妃子諫議士,紛紛點頭同意,笑顏滿面。喺朝廷上有啲之前奉獻過自己眼睛嘅人,都好真誠地流露出開心同感動嘅感情。空寥寥嘅眼袋,好似有感動之淚喺度晃動。事畢後,個細路被賞賜千金寶物錦繡,載滿晒成列嘅車馬。佢哋就浩浩蕩蕩連人帶禮,同成班官員同侍兒,返屋企,送兒還故鄉。



佢哋返到小國一刻,小國嘅國民就二話不說,即刻將隨團嘅官吏侍兒全部一個不留殺晒。



佢哋仲將佢哋嘅眼睛全部搲晒出嚟,然後推咗一座小山,擺咗喺通入小國嘅山路。帶返嚟嘅靚貨同寶物全部都燒毁晒揼晒落山。



消息冇幾耐就傳返去大國度。舉國驚訝驚愕無語,而公眾驚愕驚詫背後,所謂冷靜理性嘅,都拗爆頭,諗唔明點解小國嘅人會咁樣做,有咩意思,有咩含意。佢哋唔明,點解小國嘅人民會唔接受畀國王食眼精黎提升智慧呢一個嘅做法。眼睛有思哲嘅功效,大家貢獻眼睛畀國王食,食咗國王管治施政更加有智慧,政策更加優質有效,國家好大家好人人好,乜唔係好淺白直接嘅道理咩?點解會睇唔到呢一點呢?仲咁大反應添。諗下諗下,甚至開始唔信使團被殺清光呢件事,覺得根本冇可能會有人咁不可思議睇唔到食眼睛背後個咁明顯、咁不言自明嘅邏輯。



於是大國嘅人就派人去查證,雖然冇入到小國,但係就將嗰大咋嘅眼珠帶返返去。國王朝廷同文武百官見到嘅一刻,驚嚇無語,朝廷上下誠惶誠恐憤怒不惑。驚嚇同驚愕過後,就係憤怒同憤恨。我哋咁關照小國嘅人民,佢點可以咁做呢?點可以咁野蠻呢?點解可以咁㒼塞(粵音:文塞), 思想咁狹隘咁不化?點解咁明顯嘅道理都睇唔明,睇唔到大國所做嘅嘢背後嘅邏輯?



大國嘅大臣、官吏,諫議士同大學生同國王就呢樣嘢商討商討,冇幾耐之後就出咗條定案決定咗咁樣做。



佢哋出兵至小國。大國軍力強盛,小國不堪一擊,好快就冧低咗。擊破咗小國之後,大國將小國嘅所有人,男女老幼、公卿大夫、山林百姓,冚唪唥全部集埋晒一齊喺廣場度。大國對佢哋講,大國文明昌盛發達,胸襟浩瀚,唔會計較前嫌。大國會向小國彰顯大國應有嘅恩澤大道,以德報怨,以仁待民。佢哋話,小國之所以咁樣做,係因為小國落後,唔明大道理,所以唔會怪罪小國,亦唔會懲罰,更反而會幫佢哋喺文明程度上作出全面嘅提升,等小國佢哋都可以喺大國文明、先進、前衛,同正確的角度同觀點度睇到世間萬物玄義。咁樣,大國同小國就可以一齊同心協力,共創新天。



講畢後,大國就將小國嘅所有人,男男女女,上至華髮老人,下至初生蘇蝦同五歲細蚊仔,一個不留將佢哋眼睛全部挖晒出嚟,然之後餵返晒比佢哋自己食,等佢哋可以一享食眼睛所帶嚟嘅智慧。




% \chapter{旺角行}
% 旺角行

% 旺角街巷人來往,指夾香煙孤遊蕩

% 遊蕩花園心等望,友來一刻喜若狂

% 未見三月萬緒谷,此聚別我英倫讀

% 澤少先來黃生遲,寒暄盡省慳冗辭

% 此晚共歡不欲止,始謦已悲離別時

% 本作餞行宴會延,卿我皆知還有事

% 穿耳之約討云云,再不成事恨此生

% 大學五載將終告,傘運至今未曾做

% 霓虹艷麗二極俍[1],旺角逆韓仍時尚

% 港男港女MK味,粵音悠悠香港地

% 潮特兆萬多奇舖,戒環款式任君數

% 問主何價心意亂,此結義深銀包損

% 須臾萬世針一穿,一穿百念友情存

% 彼左我右閃炩炩,再買兩三笑盈盈

% 創興廣場食肆多,捕飛邊爐部隊鍋

% 百層揀盡中日韓,卒選放題節制喪

% 七色魚生大喜屋,蒸揚燒煮滿口福

% 清酒暢飲杯杯乾,精饈維美啖啖肉

% 餞宴盛惠二百七,淢[2]意未耗仍有凸

% 黃生終參我哋倆,無星夜晚多下場

% 共知密地各自往,私聚好處市中藏

% 昔日驕鐵今賤銅,冤魂未雪太子封

% 彌敦大道的士惡,貼錢買難人墮落

% 優步寶馬位來定,把握時間偈謦謦

% 不語並哀時下局,不義蓋天心鬱郁

% 槍林彈雨催淚驚,苦吟救命新屋嶺

% 邪道得勢英雄烈,元朗閩民何時滅

% 犬聲載道人委屈,藍衣亮刀斬見骨

% 每每下手不留情,血碎皮包欲手甩

% 有人不做做曱甴,語出身定決心殺

% 自持凜義瞄頭發,心花怒放手狠辣

% 揮棍韰爽[3]血淋淋,警寇歹作笑淫淫

% 藞苴手段打嚇氹,高叫我名滅聲𢫏[4]

% 男軀壯健可圍毆,女有小穴屌過夠

% 變態天倫齊齊玩,中出輪姦心燦爛

% 魔警發洩狎謔戲,玩完即棄落海死

% 黑衣義士夜不歸,街坊對芒[5]嘆閉翳

% 權不講理政乏義,天直民直胡話兒[6]

% 遂令百年法治體,不求達義求受規

% 中西禮樂俱崩壞,吾民尊嚴唔嬲踩

% 酷楚施盡半屍骸,誰人叫你話不乖?

% 似是而非官話語,未到三句服港豬

% 華帝赤納天下知,榮光歸港待何時

% 義仇義憤心中燒,達義之法唯私了

% 警寇手下無完膚,私了殘疾尚餘辜

% 對此極惡刑孰效,文革批鬥可參考

% 辱盡罪身尚有仁,此時澤少有餘猶

% 我笑彼思或藍色,澤少不忿百辭逆

% 語出方來我驚恐,瘋測祈求永不中

% 聚會卒之黃生來,幽靜擁抱紅酒開

% 喜些酒蓮兼暗梅,相望無懫[7]呼聲悔

% 亮火一擦點百愁,呼出煙雨念浮浮

% 浮浮幽愁語語思,呼呼泣訴不得志

% 黃生本作律士徒,冥中作祟困楚途

% 自家生意半工讀,朝勞晚冊未閒哭

% 身在福中知其福,不敢牢騷怨路曲

% 澤少自細優才生,入讀醫系心漸仁

% 天生帥氣人人道,強顏歡笑匿頹魂

% 憶昔仍享逸樂時,科科皆甲話般易

% 十五初奪紫荊𧘹,為港爭光無辰止

% 牛津面試光輝晉,光輝歲月忽暴殉

% 父去家散母薨逝,與妹相依捱生計

% 學業荒廢情斷繫,欄杆哭笑命無稽

% 蓮梅諸花忘愁借,苦海洋洋萬事亾[8]

% 顧影自憐聲嗟嗟,唯恥有辱更慘者

% 貿辭易念求意達,有否靈犀恐有惑

% 念緒纏纏冗辭長,誰曉冗辭回心響

% 夜色漸薄日繼出,筵席有散天地律

% 最後一煙功效淺,只求相別得拖延

% 晃淚殘顏講再見,封印此時共懷緬

% 鐵鳥載千半港生,粵音英語官話䘲

% 嘻哈閒語一身貴,貴族何曾履義為

% 脊斷哀號不留人,眼盲未聞飛霄雲

% 遠走高飛念重重,倫敦天氣灰濛濛

% 掟磚縱火手欲發,躲內對芒求刺殺

% 心在前線無獻奉,冷氣軍師匿懦容

% 煲底相聚待何日,凱旋歸來道莫窮

% 勸君明哲莫上前,背後發功效倍千

% 對斯雨絲似問道,避世怕死定正見?

% 越想慚漸愧流淚,書此奉微港萬歲

% [1] 俍,leong1,力丈切陰平。「老土」、「過氣」、「缺乏美感」之意,一般用於形容他人衣著打扮,或事物的外表形態及其裝飾等。用例:「佢著衫勁俍囉。」「個新建成嘅市政大樓嘅樣好俍呀。「俍」本義為「善於,擅長」,又解作「行走緩慢」,本音為「loeng4」,除了曾於古書和部分生僻詞中使用之外,基本上非常用漢字。採納「俍」是因為「俍」乃形聲字,且「俍」本用於形容他人衣著,故適合。

% [2] 淢,wek1,解「出去玩」、「出街蒲」之意,特別是指夜晚到煙酒風流之地花天酒地、男歡女愛。Wek1 據筆者理解並無所謂之本字,亦不見任何懷疑乃本字的選擇。愚見以為,可自行賜字。與其花費精力做字,不如死字復活,舊字新用,現取「淢」。「淢」乃死字,本義解作「急流」,粵語本音為 wik6。現正舊字新用且訓讀作罷。那取「淢」有沒有任何的內部邏輯和原因呢?理論上,按照傳統的漢字造字賜字理則,wek1 大概會以形聲字字來解決其有音無字的問題。而按照這套邏輯,不論其聲符是什麼,斷估其邊旁必定為「女」字旁。理論上應選用女字旁,無非是因為傳統漢字理則以女為樂,如「娛」、「耍」等。故此若然要承繼此邏輯,wek1 之字書符號理應像「女或」樣子的東西。我認為此案不可取。其一是因為沒有這個電腦字符,其麻煩不抵其得益。但最重要的是,斯字選延續著一種不要得的辱女思維和物化女性思維。取「淢」,是有見其字的水字旁。粵語人以水為財,斯理則可屢見於其詞彙措辭,茲不贅述。取「淢」,就是要強調和指出,「出去淢」的這個行為,是要花費很多錢(水)的。

% [3] 韰爽,haai1 song2,「韰」「爽」並列詞。「韰」則「high」,取死字「韰」以書之。茲創「韰爽」一詞,目的在於以粵語語素(茲則為英源粵語素)造詞,以鞏固其語素的生命力,防止死亡。

% [4] 𢫏,kam2。解「掩蓋」、「覆蓋」,例詞:「𢫏被」、「𢫏蓋」、「𢫏牌」。例句:「哎呀,個仔竟然怕生保怕到𢫏住塊面喎!」

% [5] 芒,mong1,「monitor」折取之後得「mon」繼而粵化之詞。

% [6] 直,嚴復對「right」的創譯。此創譯的選擇背後包含了非常深思熟慮的考慮,迴避了當今以「權」譯「right」而導致「might(權) is right(權)」的巨大文明性問題。 「天直」為「natural right」,「民直」則為「civil liberty」,「胡話兒」則為雙關語,因為「天直民直」是泰西的概念,也是西方最西方對中國指指點點的胡話兒,在某些人的眼中更是「胡話兒(屁話)」。

% [7] Judge 一詞,取「懫」字。其取字顯然為形聲字,但未嘗未有半絲會意字之意味。可考慮以「懫」來字書化「judge」一詞。「懫」本音 zi3,本義為「偏激、凶狠的怨恨」,又或「 阻止;塞滿。」基本上是死字。現正訓讀為 zat 6。取「懫」,一方面是因為其形聲結構合宜,其二是考慮到某程度上「心質為懫(Judge)」的結構有輕微的會意味道。

% [8] 亾,讀 he 3,去旡切陰去。意思豐富,《粵典》列「亾」有四解:一、敷衍、不認真、交行貨;二、無所事事,漫無目的地打發時間;三、求其、不認真、馬虎、得過且過;四、形容事情缺乏挑戰性,可以輕鬆完成。例句:「佢做野不嬲都係咁亾。」「我而家日日淨係喺屋企亾。」「份報告亾俾佢咪算囉。」「個測驗超亾,冇溫書都實合格。」坊間俗寫有「hea」、「迤」、「迆」等,最為普遍這為「hea」。茲取《學苑》中《提升香港話地位 — — 剝脫既定思想賦予口語詞彙書面寫法》一文之建議,取「亾」為其字。「亾」本為「亡」之異體字,但已乃死字一個。該文建議取「亾」為「hea」之賜字,是因為「亾」像一人躺臥在床上百無聊賴之形。茲實乃「形借」之為。「形借」類似「假借」,都是把已有之字重新註釋,注入新用法的技巧。「假借」借音借形不借義,「形借」則借形不借義,或借或不借音。形借字數量很少,基本上都只存在於網上語言,最為人知的例子莫過於「出獄」意思的「出冊」的「冊」。這裏的「冊」,意思不是「書冊」,而是取「冊」像監獄鐵欄之形,繼而從新註釋「冊」為監獄。斯「冊」借形借音不借義,而我們的「亾」則借形而義音均不借。其他的頗為流行的形借字包括「囧」、「厹」等等。
\printindex % Print the index










\end{document}
