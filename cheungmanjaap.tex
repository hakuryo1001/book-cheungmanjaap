\documentclass[a5paper, 10pt, openany]{book} % A5 paper size

% \usepackage[paperwidth=148mm, paperheight=210mm, top=1.27cm, bottom=1.27cm, inner=1.9cm, outer=1.27cm, headsep=1.5cm, footskip=1.75cm]{geometry} % Custom dimensions and margins
% Adjusted margins to ensure space for page numbers
\usepackage[paperwidth=148mm, paperheight=210mm, 
         top=1.27cm, bottom=2.5cm, inner=2.2cm, outer=1.27cm, 
         headsep=1.5cm, footskip=1.5cm]{geometry}


\usepackage[utf8]{inputenc}
\usepackage{ctex}

\usepackage{fancyhdr}
\pagestyle{fancy}
\fancyhf{} % Clear all header and footer fields
% \fancyfoot[C]{\thepage} % Center the page number at the bottom
% Define left and right page numbering
\fancyfoot[LE]{\thepage} % Left side for even pages
\fancyfoot[RO]{\thepage} % Right side for odd pages
% Ensure the chapter pages (plain style) also have this layout
\makeatletter
\let\ps@plain\ps@fancy
\makeatother

\input{preamble/main.tex}


% *-----------------------------------------------------------------------*
\begin{document}
% to avoid overfull hbox
\sloppy
% % \jcz{} must be run so the document can process jyutcitzi 
\jcz{}







% % \jczSourceHan{}
\begin{center}
    % super large font
    {\fontsize{100pt}{120pt}\selectfont
    {長}  \newline
    {文}  \newline
    {集}  \newline
    }
\end{center}
% \begin{center}
%     % super large font
%     {\fontsize{100pt}{120pt}\selectfont
%     \ruby{長}{} \newline
%     \ruby{文}{} \newline
%     \ruby{集}{} \newline
%     }
% \end{center}
\tableofcontents



\chapter{粵切字 Sample |  }

聲母

\begin{table}[H]
  \centering
  \begin{tabular}{|>{\centering\arraybackslash}m{2cm}|>{\centering\arraybackslash}m{2cm}|>{\centering\arraybackslash}m{2cm}|>{\centering\arraybackslash}m{2cm}|}
    \hline
    \begin{tabular}[c]{@{}c@{}}b 比\\ ⿱\end{tabular}  & \begin{tabular}[c]{@{}c@{}}p 并\\ ⿰\end{tabular}  & \begin{tabular}[c]{@{}c@{}}m 文\\ ⿱\end{tabular} & \begin{tabular}[c]{@{}c@{}}f 夫\\ ⿰\end{tabular}                                             \\
    \hline
    \begin{tabular}[c]{@{}c@{}}d 大\\ ⿱\end{tabular}  & \begin{tabular}[c]{@{}c@{}}t 天\\ ⿱\end{tabular}  & \begin{tabular}[c]{@{}c@{}}n 乃\\ ⿰\end{tabular} & \begin{tabular}[c]{@{}c@{}}l 力\\ ⿰\end{tabular}                                             \\
    \hline
    \begin{tabular}[c]{@{}c@{}}z 止\\ ⿰\end{tabular}  & \begin{tabular}[c]{@{}c@{}}c 此\\ ⿱\end{tabular}  & \begin{tabular}[c]{@{}c@{}}s 厶\\ ⿱\end{tabular} & \begin{tabular}[c]{@{}c@{}}j 央\\ ⿱\end{tabular}                                             \\
    \hline
    \begin{tabular}[c]{@{}c@{}}g 丩\\ ⿰\end{tabular}  & \begin{tabular}[c]{@{}c@{}}k 臼\\ ⿱\end{tabular}  & \begin{tabular}[c]{@{}c@{}}h 亾\\ ⿰\end{tabular} & \begin{tabular}[c]{@{}c@{}}ng \scalebox{0.5}[1.0]{乂}\scalebox{0.5}[1.0]{乂}\\ ⿱\end{tabular} \\
    \hline
    \begin{tabular}[c]{@{}c@{}}gw 古\\ ⿰\end{tabular} & \begin{tabular}[c]{@{}c@{}}kw 夸\\ ⿰\end{tabular} & \begin{tabular}[c]{@{}c@{}}w 禾\\ ⿱\end{tabular} & \begin{tabular}[c]{@{}c@{}}m/ng 𫝀\\ \ \end{tabular}                                         \\
    \hline
  \end{tabular}
\end{table}
% 韻母

韻母

\begin{table}[H]
  \centering
  \resizebox{\textwidth}{!}{ % Adjust the width to fit within the page
    \begin{tblr}{
      colspec={|X[c]|X[c]|X[c]|X[c]|X[c]|X[c]|X[c]|X[c]|X[c]|X[c]|}, % Column alignment
      hlines, % Horizontal lines
      vlines  % Vertical lines
      }
           & \empty          & -i               & -u               & -m               & -n               & -ng               & -p               & -t                            & -k               \\
      /aa/ & aa \linebreak 乍 & aai \linebreak 介 & aau \linebreak 丂 & aam \linebreak 彡 & aan \linebreak 万 & aang \linebreak 生 & aap \linebreak 甲 & aat \linebreak 压              & aak \linebreak 百 \\
      /a/  &                 & ai \linebreak 兮  & au \linebreak 久  & am \linebreak 今  & an \linebreak 云  & ang \linebreak 亙  & ap \linebreak 十  & at \linebreak 乜               & ak \linebreak 仄  \\
      /e/  & e \linebreak 旡  & ei \linebreak 丌  & eu \linebreak 了  & em \linebreak 壬  & en \linebreak 円  & eng \linebreak 正  & ep \linebreak 夾  & et \linebreak 叐               & ek \linebreak 尺  \\
      /i/  & i \linebreak 子  &                  & iu \linebreak 么  & im \linebreak 欠  & in \linebreak 千  & ing \linebreak 丁  & ip \linebreak 頁  & it \linebreak 必               & ik \linebreak 夕  \\
      /o/  & o \linebreak 个  & oi \linebreak 丐  & ou \linebreak 冇  &                  & on \linebreak 干  & ong \linebreak 王  &                  & ot \linebreak 匃               & ok \linebreak 乇  \\
      /u/  & u \linebreak 乎  & ui \linebreak 会  &                  &                  & un \linebreak 本  & ung \linebreak 工  &                  & ut \linebreak 末               & uk \linebreak 玉  \\
      /oe/ & oe \linebreak 居 &                  &                  &                  &                  & oeng \linebreak 丈 &                  &                               & oek \linebreak 勺 \\
      /eo/ &                 & eoi \linebreak 句 &                  &                  & eon \linebreak 卂 &                   &                  & eot \linebreak 𥘅$\_{\text{朮}}$ &                  \\
      /yu/ & yu \linebreak 仒 &                  &                  &                  & yun \linebreak 元 &                   &                  & yut \linebreak 乙              &                  \\
    \end{tblr}
  }
  \caption{韻母}
\end{table}

聲調

\begin{table}[H]
  \jcz{}
  \centering
  \begin{tblr}{
    colspec={|X[c]|X[c]|X[c]|X[c]|X[c]|X[c]|},  % Equal-width columns and centered text
    hlines,  % Draw horizontal lines
    vlines   % Draw vertical lines
    }
    1   & 2 & 3 & 4   & 5 & 6 \\
    󰘠、󰘦 & 󰘡 & 󰘢 & 󰘣、󰘧 & 󰘤 & 󰘥 \\
    󰝰、󰝶 & 󰝱 & 󰝲 & 󰝳、󰝷 & 󰝴 & 󰝵 \\
    分   & 粉 & 訓 & 墳   & 憤 & 份 \\
  \end{tblr}
  \caption{切字 聲調}
\end{table}



% \begin{table}[htbp]
%   \jcz{}
%   \centering
%   \renewcommand{\arraystretch}{1.5} % Adjust row height
%   \setlength{\tabcolsep}{4pt} % Adjust column padding
%   \resizebox{\textwidth}{!}{
%   \begin{tabularx}{\textwidth}{|X|X|X|X|}
%   \hline
%   % \rowcolor[HTML]{D0D0D0} 
%   \textbf{坊間漢羅混用} & \textbf{漢字已整理版本} & \textbf{漢字粵切字混用(未組裝)} & \textbf{漢字粵切字混用(已組裝)} \\
%   \hline
%   咁都係果D嘢嘎啦,廿鯪蚊個餐又湯又剩唔通有得你食天九翅咩?求求其其有D肉有D菜蛋白質澱粉質撈撈埋埋打個白汁茄汁黑椒汁咁撐得你懵口懵面咪Lui返去返工返學返廠返寫字樓囉。唔係你估真係搵餐晏仔咁簡單啊。咁跟飯定跟意粉啊? 
%   & 咁都係果啲嘢㗎啦,廿鯪蚊個餐又湯又剩唔通有得你食天九翅咩?求求其其有啲肉有啲菜蛋白質澱粉質撈撈埋埋打個白汁茄汁黑椒汁咁撐得你懵口懵面咪纍返去返工返學返廠返寫字樓囉。唔係你估真係搵餐晏仔咁簡單啊。咁跟飯定跟意粉啊? 
%   & 丩今´都係丩个´大子¯野丩乍`力乍`,廿力正⁼蚊個餐又湯又剩𠄡通有得你食天九翅文旡¯?求々其々有大子¯肉有大子¯菜蛋白質澱粉質撈々埋々打個白汁茄汁黑椒汁丩今´止生゙得你懵口懵面文兮`力句¯返去返工返學返廠返寫字樓力个¯。𠄡係你估真係搵餐晏仔丩今`簡單⺍乍⁼。丩今´跟飯定跟意粉⺍乍`?
%   & 󱜩都係󱟡󰦠野󱛒󰿒,廿󰻃蚊個餐又湯又剩𠄡通有得你食天九翅󰗘?求々其々有󰦠肉有󰦠菜蛋白質澱粉質撈々埋々打個白汁茄汁黑椒汁󱜩󰿽得你懵口懵面󰖚󰾠返去返工返學返廠返寫字樓󰼠。𠄡係你估真係搵餐晏仔󱜪簡單󰀓。󱜩跟飯定跟意粉󰀒? \\
%   \hline
%   \end{tabularx}
%   }
% \end{table}

\chapter{朋友失蹤左成年,最近先知道佢跳樓走左}

朋友失蹤左成年,最近先知道佢跳樓走左。

唔知點解就上左黎連登無病呻吟。

係我以前最想自殺嘅時光入面。

係佢誤打誤撞咁拉返我去班兄弟到。

雖然佢份人成日傻笑,講埋曬奇怪野成日鳩\lr{言}{}。

但唔知點解有條死仔係身邊,我覺得幾開心。

但後來到佢抑鬱嘅時候,我陪左佢唔少嘅夜晚

佢曾經話好清楚自己唔會尋死。

因為佢係一個好細膽嘅人。

就算個晚傾左好多,成個人舒暢左都好

佢訓醒第二朝都變返好抑鬱。

後來佢都漸漸遠離我地。

可能係呢份關心對佢黎講太難受。

可能係過唔到自己個關唔想俾其他人睇到佢咁。

可能係知道自己咁落去會俾好多負能量我地。

我地都再搵唔到佢,就係咁過左一年。

我收到警局嘅電話,話佢跳左。

其實我冇感到意外,只係覺得好無力。

好似史丁格的貓,你唔打開個箱    

永遠都唔知入面隻貓係生定死。

你唔打開,隻貓都仍然可能在生。

但一旦你打開左個箱,知道隻貓死左。

你就改變唔到個事實。

我知道我見到佢嘅個刻,就係打開箱個一刻。

只係咁,咁唔好彩,原來個結局係咁。
    
或者我應該再努力去搵佢?

或者我有能力避免佢嘅死?

就算最後改變唔到佢自殺嘅結局。

就算死係對佢黎講最好嘅解脫方式。

起碼走之前都有個道別,食一餐飯啦。

起碼等佢知道,有人會記得佢。

而唔係一個人不安咁走    

...或者呢樣係我一廂情願嘅諗法。

到頭來我為唔到佢做咩

係我無能為力,定係我冇盡力去做。

我都唔知。

我知道自殺唔係一種衝動。

因為我都曾經試過無數次係天台徘徊。

跳,係一瞬間嘅事。

諗,可能每日都諗左千百萬次。

我好想知,最後個刻佢究竟係咩心情,諗緊咩。

我好想理解佢,再了解佢最後一年嘅生活。

好想俾個擁抱佢。

唔係因為見到朋友唔開心,出於責任與身份去拯救。

而係因為你對我好重要,我想去咁做。

我冇刻意去安慰你啊,我唔係因為可憐你先陪你。

唔需要因為咁而感到壓力架傻佬。

係因為我陪你嘅時候,你都陪緊我,我覺得好安心。

我先聽你講咁多說話,先同你分享咁多事。

我唔係想怪你,因為我知你都無得揀,但係。。。

如果你就咁就走左去,點解當初你要拯救我。

我知道人終有一死,我知道呢個世界係好撚痛苦。

但同你圍爐取暖,苦中作樂。

我覺得我可以再走落去。

我都好想你陪我走落去。

你陪我,我陪你。

我好掛着你。

我地好掛着你。

大家都好掛着你。

一路走好。


\chapter{我阿哥成為󱃡我同老豆老母之間的隔閡}

(放負長文慎入) 

我阿哥成為󱃡我同老豆老母之間的隔閡

小弟一家四口,老豆老母阿哥同我。

老母係傳統家庭主婦。

嫁󱃡畀老豆就冇打工。

老豆一個人養起頭家。

屋企背景都簡單正常冇乜問題。

直至阿哥讀唔到書中五畢󱃡業。

周圍打散工冇乜大志。

渾渾噩噩咁轉工快過揭書。

而我有幸搵到份正常寫字樓工。

有穩定󱝚收入同前景。

基本上我冇嘢要老豆老母擔心。

年前阿哥中出左條女做人老豆。

老豆老母攞晒積蓄出嚟比阿哥買樓。

等佢結婚組織家庭。

亦因為咁老母同我講屋企宜家住緊層樓。

會畀我同老婆一齊住。

我老婆都好好肯同我老豆老母同住
        
幫輕我老母做家頭細務打理頭家大小事。

本來都相處得好融洽。

但近幾年疫情問題

阿哥失業搵唔到錢供唔起層樓

阿嫂份糧拎晒嚟養個仔同照顧自己娘家。

最後佢地賣左層樓

同我老母講要一家三口搬返嚟屋企。

以我所知賣樓舊錢冇畀返老豆老母。

換言之變󱃡我同老婆冇地方住。

老母同我講我兩公婆都有份叫做穩定󱝚工。

但阿哥收入唔穩定加上有個仔要養。

希望我同老婆搵地方搬走。

體諒吓阿哥困境。

仲話相信我哋他朝一定可以靠自己能力上車。

我同我老婆都覺得好失望。

原來一個人無用就可以理直氣壯咁攞晒所有資源。

有能力󱝚人就理所當然乜都要靠自己。

咁多年嚟自問做仔冇要過老豆老母擔心。

好好地做人讀完書打份正經工。

相反阿哥長期要老母照顧勞氣傷神。

估唔到最後老母都決定叫我走。

除󱃡心灰意冷都唔知仲有乜可以講。

我有幸娶到我老婆。

佢叫我唔洗擔心點同外母交代。

一切交比佢處理。

仲安慰我話趁後生捱幾年等機會希望可以上車。

雖然我都知租地方洗費大要草錢難上加難。

樓價又持續高企我哋󱝚環境簡直雪上加霜。

我都唔知點面對外母。

外母對我好似自己仔咁。      

每次上去食飯都成枱餸。

個女嫁󱃡比我無成成要租樓住。

老婆有晚好坦白同我講佢好憎我阿哥。

對我老豆老母呢個決定都好心淡。

我又點會唔明白。

懂事開始自問負責照顧老豆老母。

呢個責任冇旨意過阿哥同我分擔。

佢亦都冇分擔過。

我都想做個乖仔孝順仔。

但呢刻我覺得我自己先係比全家遺棄󱝚一個。

呢個家已經唔再係我󱝚家。


\chapter{香港人將被孤立 我地要做嘅係生存 唔係發動戰爭}

暴政報告重點關於教育,目的好明顯係要洗新一代腦,目的在於孤立反送中一代嘅香港人。

幾年後好快我地就會發現上層得益者掌勢者親政府,新一代後生冇腦我愛中國,得返依二百萬人對抗中共。

更甚,死嘅死走嘅走放棄嘅放棄,二百萬人十年後可能得返十萬人,少之又少。

所以好老實地講,香港依個地方已經玩完

中港共對付香港最主要嘅一招就係分化群體,先將香港人分成新舊香港人,再到黃藍,再到社會階層,老幼,男女,鬥黃。

從而防止香港人形成到有效嘅反抗力量,同時令我地感覺無助,少數化,失去反抗意志。

依個玩法,基本上係冇得拆,中共係一定會成功,而且目前單係鬥黃,和勇鬥,lunch哥鬥,龍門渣哥鬥,我地係冇方法防守,因為對方有主權,有制度,有武力,有資源,點玩都唔會輸。

所以首先第一步,請接受香港係無法防守嘅地方,分化無可避免。

第二步,要接受本地抗爭無出路,香港人嘅出路唔係反抗,只有生存。

理由同上一樣,對方嘅實力實在比我地強太多,無論幾多犧牲,單靠香港人唔會做到有效打擊,你話搞臭香港,一半一半啦,更多係政府自殺以求達到政治目的。

見到有post講要香港人真正反抗,我明白大家會覺得香港人冇盡全力,但問心當今世代有幾多個仲願意為信念而死?香港好榮幸,我都好榮幸,見到過去一年真係有好多人為信念而付出所有,我唔敢話自己係佢地嘅手足。

但問題亦在於,你地太偉大。你地係社會上極稀有嘅勇士,望下外國,望下爛到不得了嘅美國,世上真係無幾多人可以做到咁。

而由現實角度出發,絕大多數人只願有限度付出,先係社會常態。唔係要blame the victim話勇士錯,只係政治係一們人嘅課題,出發點應該由最平凡最廢嘅大眾廢柴出發,當我地眼中只有依啲英雄,只會反過來令自己內疚,令自己死係盲目追尋英雄電影嘅路上。

咁你即係話當人condom啦。好無奈地,係現實角度下,我地除咗銘記同支援依啲英雄意志,所能做嘅非常有限。

返到以大眾為本嘅角度下,本地抗爭自然難而搵到出路。

早前都有post講過,話歷史上往往由低層革命成功係呃鳩你,冇中上層嘅資源權力,邊到得黎推翻政府?

革命唔止係要破壞現有制度,更重要係置入新制度,你要比大眾知道革命成功後會有一個更好嘅政府,有人才,有方向,我地依啲廢人先會敢支持你,因為冇人會願意陪你冒住失去一切嘅風險進入無政府嘅狀態。

因此接住落嚟係我對目前形勢嘅主要結論:

生存嘅首要條件係 活下去

香港人需要嘅,係be water,求生存而非正面對抗。

第一個原因,係上文提到香港黃營將會無可避免地被孤立,實力差距之大,令打贏中共變得不可能,但亦唔代表要死清光。

支那利用分化群體去削弱我地,但始終依種手法有極限,目前難以做到東德人人都係秘密警察嘅水平,對於家庭同朋友關係,我地依然有最大嘅控制,所以應該由依啲最細嘅尺度開始,建立防衛。

好朋友間互相支持,堅守對民主香港嘅信念,就算出到門口要大叫我愛習主席都好,盡力令依個小圈子成為信仰嘅堡壘。依個圈子將會係你唯一嘅依據,將會係光復香港最後嘅存在地,唔好強己所難去守護群體,先守住身邊嘅人。

繼而由依個迷你群體出發,連結其他嘅堡壘,積累防禦嘅力量。有能力嘅絕對要離開香港,因為係你外國你可以有更大嘅堡壘,甚至公然吸納同路人,一切關於積累生存資本,試下當自己玩緊人在野,光復香港就係得救,只要生存到落去直到得救,食屎你都要食。

另外一個原因係階層問題,當下抗爭主力係堆書未讀完冇錢冇地位打架都冇力嘅細路,你想點贏?

但十年後二十年後,我地就會係社會上嘅中層,假如到時嘅你擁有咗財力權力,依個就會係光復香港嘅第二個機會,就算只有一萬個成功人士願意好似今日咁出錢出力,勝算都會比今日十萬個學生高。

最理想嘅場境,係香港人可以好似猶太人一樣,流向世界各地,努力建立好自己嘅資本後再發起戰鬥,然後回歸故土。

或者你會話,講就理想,等多十年腦都比人洗到一乾二淨啦

但我又真係唔明,假如你連上街比人打死,被強姦,坐一世監嘅勇氣都有,點解冇勇氣守護好自己嘅信仰堡壘。你信香港人可以跟你一齊血洗中南海,點解唔信你嘅手足可以一齊行落去?

你無信心嘅,其實我真係幫唔到你,但我好相信,依個先係真正考驗香港人有冇資格存續落去

政治係一場博奕,生存係一場博奕,而博奕需要嘅係資本。

唔好爭一時之氣,有時候用最現實嘅角度出發,反正會見到最有希望嘅曙光。

各位香港人,我地一齊堅守,努力向上游,十年後煲底見

利申全文冇冒犯任何同路人嘅意思,單純個人睇法。


\chapter{冇人好似我咁 好似成世都比屋企人拖累}

(長文慎入) 冇人好似我咁 好似成世都比屋企人拖累

畢業冇幾耐我揾3萬頭

屋企人要求我每月比15000家用

話係我應份 話唔比就不孝 話養大我咁多年供書教學

但其實我啲學費都係我自己問政府借返嚟

畢業咩成3-40萬學債

初時社會入世未深唔知 以為個個都比一半人工家用係正常

比咗大概半年問返其他人先知唔係

我肯比,但應份呢句說話對我嚟講真係好難聽

無耐又發現自己有左

屋企人又要我老公比10萬禮金

話如果唔係就唔尊重佢哋 冇誠意娶我

屋企人向我施壓 我老公又向我施壓

搞到當時我同我老公關係唔好

我老公唔係啲咩好有錢

但最後都比咗10萬我屋企

依家諗返我都覺得係賣女

有左期間發現自己要做緊急手術

唔做會有性命危險

要攞錢出嚟做

問屋企人嗰10萬用咗喺邊

佢哋話同我攪婚禮用哂

但我冇擺酒 只係租婚紗攝影簽字酒店食buffet

完全唔覺得會用到10萬

我開始唔信屋企人

成日覺得比得佢哋有事要錢佢哋都幫我唔到

手術之後我同bb平安

但自此我就決定淨係每個月比\$10000家用

屋企人係不滿

同佢哋解釋我都要為即將出世既bb草錢

第時開支會大

但佢哋都係不滿有說話聽但都唔輪到佢哋唔接受

間唔中同我講佢哋冇錢叫我比啲佢哋用

我就會質疑點解家姐有返工你哋唔問佢攞

佢哋就話佢人工低過我 揾得萬幾

話佢每月比得嗰3-4千點夠

呢刻我就知原來我一直受到不公平對待

我就要比一半人工 我家姐就比少過人工一半就ok

原因係佢哋覺得家姐揾得少 話我計較

呢刻我心都淡哂

我家姐明明大過我好多

佢3字頭 我2字頭咋…我仲要有自己家庭㗎…

我好介意 覺得屋企人一直剝削我喺我身上揸干揸淨

完全唔喺我立場諗

無意中發現老豆借錢爭人百幾萬債

暪住我哋成家按咗層樓

我媽激氣

我比我媽仲激氣

老豆當年如果唔係以氣用事辭咗份工

就唔洗淪落到而家咁樣

仲比我發現佢已經好多年冇俾過家用屋企

就係為咗維持間垃圾舖去借錢交租

借錢前完全冇同過屋企人商量

一啲尊重都無

家下仲要我哋成家幫佢執蘇州屎

話就話冇要我哋幫佢

但唔幫佢後果會點?我返屋企被人砍都唔知

為咗成家著想(你當我都係為自己)

每月我比22000屋企

得幾千自己搭車食飯返工

比咗4個月最後同佢反面

原因係比我發現佢返完工返到屋企唔洗手掂我個仔

仲要提咗好多次我先咁嬲 疫情下佢都夠擔死咁

個b 1歲都未夠㗎

話佢佢仲覺得自己啱 仲屌9我洗唔洗送個仔入院

我嬲得濟講句你都痴q線就入房

佢衝入嚟仲踢入我房門鬧我話忍咗我好耐

我推佢出去 佢大力捉實我隻手整到我損哂

呢啲人死不悔改 我決定以後都唔會再幫佢

佢哋大人嘅野自己搞掂

有咩事收數踩上嚟我住男家都好過同佢住

生完quit咗之前份工

原因係太大壓力日日被上司屌 日日喊

依家雖然揾得2xxxx

但勝在同事夠好 返工都返得開心

屋企人收咗成10000家用

成日借啲兒叫我買肉比個仔唔扣落家用

其實我真係揾得唔多 我仲要草錢買樓

我唔敢講全部但由出世開始個仔起碼有8成都係我比

今個月份糧找咗上個月成萬蚊卡數

其實今個月我冇錢連錢都草唔到先想屋企人扣落家用

就被屋企人話我計較

諗極都唔明 點解唔幫佢比錢就係我計較係我問題

我好唔開心 其實我都比咗一半人工佢

佢哋唔會日日都煮飯

我平時都會碌我張卡請佢哋食嘢飲野

刺身 壽司 海鮮 燒肉呢啲我想食嘅時候

我就要叫埋佢哋份量

變相我洗費double左好多

我都冇同佢哋計

其實屋企人生日我都有送部iPhone 名牌袋比佢哋

點解要咁對我。。。

其實係我自己諗埋一邊定係真係唔應該係咁?

有冇人可以指點下我

講明先我都想快啲儲到錢買樓

死都唔租樓幫人供樓

但我份糧就好似被屋企人食住咁…

其實我一直都好唔開心


\chapter{窮有窮養呢句野真係害撚死人}

(長文慎入) 窮有窮養呢句野真係害撚死人

\#1大意失賓周•5 年前

尋日同個fd去左個同學仔既生日會, 有感而發.

本身我識左呢個fd好多年, 中學到而家, 大家都出身基層,

但經歷就唔同, 詳細我唔講啦, 我唔可以話我而家好成功, 但收入算穏定同唔錯(唔係哂命, 只係要交代下),

但佢就打份唔係好穏定, 收入約1-2萬左右, 老婆做文員, 都係萬到尾左右, 而家租緊屋住.

5,6年前, 大家都各自結婚, 個陣我問過佢會唔會生小朋友, 佢話應該會啦,

我再問, 問左好多問題, 例如你ready喇咩? 邊個湊, 等等, 其中最重要一個, 你地個經濟能力負唔負擔到架?

佢就好爽快答我, 唔知架, 不過無話得唔得既, 窮有窮養架姐....

無耐, 大家都有左小朋友, 中間d野太長, 我唔講太多, 有人有興趣知我先再答.

我想講既係, 佢由個仔出世, 大部份BB野都係買2手或淘寶, 結果有次, 個仔唔知係咪咁, 有皮膚病,

佢竟然因為睇醫生貴, 走去信埋d阿阿婆買唔知d咩沖涼, 結果搞唔掂, 最後都要睇醫生, 但醫生話因為太耐, 所以就算醫返好都會有印...

跟住奶粉佢又走去買水貨, 好好彩食佢唔死.

好喇, 尋日去同學仔生日會, 禮貌上都會買份禮物給人, 佢就因為唔捨得, 去淘寶買d cheap雞野.....唉.

好啦, 咁個個小朋友都會帶d玩具去玩架嘛, 佢個仔帶去個d, 滑板車, 單車, 又死又舊, 有d位仲睇得出修補過.....

咁小朋友會交換玩下, 但佢個仔個d可能就係因為咁, 無人肯同佢換黎玩, 但佢個仔又想玩人個d, 結果就喊住同佢講話想要, 叫爸爸買個比佢, 佢就話好啦, 爸爸買比你,

但我知, 佢應該係暫時敷衍佢或者買二手或淘寶....

個刻我望住佢個仔, 覺得好可憐好心up....


% \chapter{(長文慎入)一齊七年 分手一個星期 每一刻都好難受}

% (長文慎入)一齊七年 分手一個星期 每一刻都好難受

% 我知道而家香港咁樣,仲因為愛情係到叫春好抵屌,好對唔住,但我真係好痛苦。

% 文長,但希望巴絲可以睇晒佢,因為我真係冇渠道講晒出黎。

% 9月最尾個日佢話即興約左個女仔朋友放工去食飯,

% 差唔多夜晚11點我上連登見到佢去食飯個區有兩個女手足比人拉左,

% 我擔心佢所以Telegram 問佢返黎未,佢冇應機,因為真係驚佢可能有事,

% 所以打左比佢,佢一聽電話,個背景靜到仆街,話食完飯,已經返到自己住個區,上左個女仔屋企,擺低左差電所以唔知我搵佢。

% 我唔知痴左邊條筋,㩒左落去Find My iPhone ,比我見到佢唔係個女仔屋企個區,之後cap 圖問佢,佢話呃左我,係食完飯之後,上左個男仔屋企吹水,

% 但冇做其他對唔住我既野,同個男仔曖昧左一排,話對唔住我,唔知道點樣面對我,話自己on9,貪玩貪新鮮。(其實係2個月左右之前,我用佢電話個陣,我已經發現佢del WhatsApp 對話紀錄,男人既直覺就覺得佢係條狗公,叫佢同依個男仔同期少啲聯絡,佢答應左我,一直以黎我都係問佢去邊 同咩人,佢去到好夜就叫佢快啲返,有時會煩多兩句,check 佢係邊真係第一次,點知一check 就爆大獲,有時我真係好後悔check左)

% 個一刻我成個人驚到震晒,完全唔識面對,好嬲同時個心好痛,因為佢做既野,

% 真係好陌生,完全唔係我認識既佢,開頭佢都仲問我係未冇得返轉頭,話仲愛我,但話冇左信任,心入邊有條刺,之後都會因為依件事鬧交再分手,話唔想浪費時間,仲去唔去台灣交比我決定。(本身9月尾book 好晒野等佢黎緊10月放三日連假放鬆下)

% 我諗得好清楚,我信佢係近期先同返個男仔WhatsApp 傾計曖昧,冇做其他對唔住我既事,所以想繼續同佢去台灣,因為台灣係我地第一次旅行既地方,我想係台灣重新開始,因為大家經歷過既野太多啦,所有既高高低低,一齊去過兩年working holiday,要磨合既野都磨合完,我一早已經認定左佢,我同佢講原諒佢,開頭心入邊有刺就一定,但既然佢寫錯字,我就做佢既擦膠,大家重新建立返信任,雖然要啲時間,但只要大家都用心,係可以感受到,而且因為係佢,所以我願意,因為我真係好愛佢,都答應佢今次會改晒我以前既唔好既行為,唔會再好似以前咁Hea住先/做一半,唔會再係咁婆媽,唔會再係佢推我先做,以後我都會行先再拖住佢行埋落去,因為我真係唔可以失去佢。

% 點知到左第二日佢放工,佢冇啦啦同我講都係算啦,唔去台灣啦,

% 約一日比返大家既野大家,然後好好放低,佢話佢想結婚(佢25),

% 覺得我一直都唔嗲唔吊,安於現狀,比左太多機會我,佢已經心灰意冷,

% 唔想再做我既全世界,平時我都會管下佢,今次依件事之後,一定管得更嚴重,

% 加上一直以黎既野,經過今次件事,令佢想分手。然後又話唔係因為個男仔,

% 佢會應承我之後都唔會再搵個男仔,因為知道個男仔唔係佢想要既,

% 最後叫我聽矛盾一生就會明。(我明啦,我真係明啦,可唔可以比埋最後一次機會我呀?)

% 係台灣個陣,我試過WhatsApp挽留佢,但換黎係佢既無情,睇住佢ig

% 一張一張咁del 晒我同佢既回憶,我又咩都做唔到既個一種無力感,

% 睇住佢Del 我Telegram、Line、換晒啲icon,好想死。

% 如果唔係咁啱有個兄弟都有三日連假,過黎台灣陪我,我好驚我係台灣亂諗野,

% 但佢陪我2日,最尾個日得返我一個個陣,自己留係民宿爆喊,

% 我真係放唔低佢,佢係我既全世界,7年黎都係圍住佢轉,可唔可以唔好留低我。

% 冇錯我依一刻真係冇足夠既本錢同佢結婚,但我係失去之後真係好大決心要做返好,要做到佢想要既野,我只係想佢留係我身邊,係我攰既時候望一望身邊,然後可以繼續努力。好掛住大家既外星話、暱稱、每日既早晨早抖、我地既所有...我真係好後悔唔再早啲努力做好自己,比到你想要既野你,我唔會講如果,因為只有如此,講再多既如果都係無補於事,但我真係好後悔同好嬲自己。

% 係分手既頭兩晚,我搵返我冇聯絡好耐既兄弟們,真係好慶幸佢地仲係到,冇佢地既陪伴我更加唔識面對,雖然佢地講既所有道理都明白,聽得入但做唔到,只要自己一個人既時候,就會亂諗野,佢地都有自己既生活,冇可能每一日都陪到我飲酒傾計開解我,更加明白冇可能日日煩住人,加上7年黎都圍住女朋友轉,已經冇同好多朋友聯絡,而自己本身講野算係話題終結者,好難搵到岩傾既人,加上我朋友真係唔多,可以講係得返佢地幾個,但同時又唔想對佢地講咁負面既野出黎,所以好多野都係收埋係心入邊,加上本身個人就係天生比較憂鬱多感,有時情緒起伏好大,要對住自己100%信任既人先可以做自己,以前就話可以同女朋友分享我既所有,而家?

% 而家每一晚都失眠,好似有抑鬱症咁,覺得每日都係陰天落雨,

% 得返自己一個行落去,時間停滯左,又好似一步一步行緊落個海咁,

% 慢慢浸過嘴、浸過鼻、眼,覺得咁樣浸係呢個世界終於冇咁重,唔想掙扎,

% 好想永遠留係呢個世界,就算唞唔到氣好難受都好,都唔想行返上水面。

% 每日都loop 緊奶共既supper moment 首 今天開始單調了(因為真係好中),

% 每一日都仲會擔心佢過成點,每一刻都係到諗要做返好自己,但係個人冇動力,

% 飯都唔想食,根本冇精神去踏出第一步去振作,就算變成一個更好既人又點,

% 佢已經唔係到。有時愈想忘記,愈記得清楚,個死人腦成日都自動播放同佢既回憶。然後又會諗如果我做到佢想要既野,有冇可能追返佢,好想比到幸福佢,同佢過埋下半世,好驚當我做到佢已經有另一半,或者佢根本依一刻已經對我死左心,我做咩佢都唔會返黎,雖然佢有講過依句說話,但我心入面一直叫自己唔好信。唔夠膽再搵佢,驚佢討厭我既死纏爛打,然後block 晒我,

% 到時我真係唔知我會做啲咩。

% 到依一刻我明白,返唔到轉頭,原來就係痛苦既源頭,每次出門返屋企,行經你樓下,個心都會痛,唞唔到氣,以前仲係一齊個陣,你成日問會唔會唔要你,我答唔會,又問我冇左你會點,我話會自殺,然後就話唔準咁諗,但諗到真係跳落去,可能令到你內疚,我就會冷靜返少少,但每當我爆喊個陣,心入邊就會彈依個選擇出黎,好似刺激緊我去做咁,同時好似同我講緊係我既解救。我真係放唔低,我真係好冇用,因為掛住你已經變左我既呼吸。好想有時光隧道,返去個晚你親口同我講要分開,最尾個一刻既擁抱,係你望返轉頭個陣,我應該要跑上去攬實你,可能仲有機會令你比多次機會我,但依家形同陌路,好想再聽到你把聲,你同我分享喜與憂,冇左你既世界係黑白色既,加上香港搞成咁,冇左你係我身邊,我再冇心力去對抗世間既恐怖啦,我最後在乎既都失去左,我對呢個世界已經冇希望。

% 雖然個個都叫我頹廢一排好啦,快啲放低,做個全新既自己

% 我都好想放低,但我諗我呢一世都唔會放得低,

% 有冇人可以話我知,點解明明發現佢呃住我之前個幾日,

% 大家都仲好開心咁相處緊,情侶會做既野都做齊,

% 點解可以係發現之後,兩日裡面變到咁狠心絕情、陌生?

% 有冇巴絲係仲未放低/放低左,可唔可以分享下。


\chapter{AO EO 一定要令運動成功 [長文]}

AO EO 一定要令運動成功 [長文]

首先好多謝你哋為香港文官統治咁多年作出貢獻。

你哋可能會問,我個個月五皮十皮,人上人,究竟企出嚟值唔值得。我以下有一個小小推演:

如果運動失敗,美國同歐盟肯肯定會制裁香港政府,甚至取消香港關係法。如果你係從事同警察及政制相關工作,你哋可能成為歐美制裁人士,永遠無法入境任何先進國家,亦永遠無法參與國際金融市場。 即使你係負責非政治工作例如交通福利教育,你亦有可能被拒絕歐美簽證。

唔好以為自己去外國,同外國官員握過吓手,就以為係國際深厚關係。美國一吹大雞,佢哋可以全部翻臉不認人。人哋一直俾面香港,有免簽証有工作假期有出入口協議,並唔係俾面你或香港政府,係俾面美國。

你可能會覺得冇所謂,因為你唔出國玩。 但你喺香港嘅工作同生活亦會受到影響:經過呢個月,中國已經完全了解到公務員系統係佢管治香港嘅最大敵人。中國會加速利用新香港人同地下黨員取代你哋。所以你喺政府入邊嘅晉升同發展,通通都會受到限制,甚至迫害你哋。 中國亦會安插更多政治助理,用黨委書記方式監視及控制你哋嘅日常工作。

送中條例同社會信用系統亦肯定會落實:因為你哋好可能會被外國永遠制裁/拒簽,你以及你嘅下一代,將會永遠被困喺呢個極權環境之中,痛苦地服務北方殖民主。

等大灣區完成之後,中國會威逼利誘香港人流散廣東各地,跟住搞次文革,叫本地人種族屠殺大部份香港人。曾經有過管治經驗同外國聯繫嘅你哋,識揸槍嘅香港克警,以及有宗族網絡嘅新界佬,應該會係首要目標。而防火長城以外,冇人會知,知亦冇人幫到你。

你見解放軍日日喺報紙喊打喊殺,點知一外訪只係搶奶粉。官員日日話國家尊嚴,個個送晒子女去美國生活,有錢佬日日都話愛國,全部衝去美國買房地產。加上全球政治向右轉,美有侵侵,英有約翰遜。 局勢發展應該好明顯,你哋係人中之龍,點會睇唔到。

結語:

如果你已經決定咗救香港,多謝你 。

如果你仲係十五十六,諗下以上幾點。

如果你今日已經係中共地下黨員,享受中國生活水平,覺得以上係危言聳聽,咁你生活如常就得。

天佑香港。


\chapter{Untitled document}

劏房生意原來真係好好賺

\#1紐約洋腸隊•3 年前

本身都知劏房好執, 不過無諗過條數係咁誇, 都係近呢1~2年無做野全力幫屋企先知

屋企有十幾個單位做劏房, 計埋有成 5~60 個劏房租緊出去, 平衡租到 4~5k

1個月廿幾萬袋袋平安, 多數做劏房都係唐樓, 又無管理費, 最多係1百幾十垃圾費, 收返多d 水電費果到都夠 cover, 無打里印稅又唔撚洗交

咁當然都會有租霸, 但係洗小小錢搵人同佢傾下計多數1個月耐會走, case 數量唔多所以都ok

早10年8年試過舊區重建收購, 1個單位收過千萬

其實好矛盾

做開深水埗研究

假如你係政府,今晚取締成個深水埗所有劏房

一晚之內有六七萬人要瞓街

而呢六七萬人入面,一大半係本地香港後生仔

而喺現實上面新移民住劏房好快就會上到樓

本地嘅後生仔通常要比新移民遲三四年先會上到

而且係偏遠嘅大西北


\chapter{屬於香港人嘅烏托邦}

https://lihkg.com/thread/1267396/page/1

鳩鳩地•5 年前

好多宗教都有所謂嘅烏托邦 [意指︰理想完美的境界,特別是用於表示法律、政府及社會情況。]

每個人都係好嚮往呢種地方 亦成為許多教徒堅持嘅動力

極端派伊斯蘭教甚至認為殉教 係至高無尚嘅死 非鼓吹血洗香港

無奈係香港現今情況下 要打一定唔夠黑警港共鬥

所謂嘅香港原住民 政府留晒地 有錢有樓住

就算反送中條例撤回 已前已經陸陸繼繼都會有更加多嘅政策去同化 一體香港

就算畀我哋爭取到有五十年不變 之後又可以點

打倒中國共產黨談何容易 CCTVB一台獨大 長輩連正確資訊取得嘅權利都冇

社會分歧只會愈嚟愈大 加上每日150支那入港 明日大嶼 不斷開放新關口等等 滲透只會不斷加深

不過香港人永遠都唔會冇退路 目前辦法當然係繼續抗爭 爭取民主自由

最後退路就係集資買島嶼鍵盤建國 由一個小小漁村再出發 尋求國際援助

撤資 全民勞動砍掉重練 買島其實早於97回歸已經有人提出

我主要想講嘅係大家一定唔好絕望 路有多條 可能最終被迫離開家園

但係我希望各位心中都要有一個烏托邦 成為未來堅持動力

自殺對事情幫助不大 加上消息被完全封鎖

返工不了都好 到時真係需要資金嘅時候都可以出一分力

雖然係遙不可及 但要做嘅話 香港人一定得

喺中共統治下 有思想嘅人係最痛苦 永遠唔會有所謂嘅烏托邦

要打倒中共 香港人獨力一定冇可能 其他國家領袖亦都唔會貿貿然出手相助 最多出把口


\chapter{[長文慎入]到底平時要點樣氹返女朋友}

[討論][長文慎入]到底平時要點樣氹返女朋友

話說小弟近排有個朋友拍拖 跟住個朋友問 佢女朋友成日要佢氹番佢(佢女朋友) 咁佢知道我拍咗拖都年幾 就問我點算好 然而我呢一個拍咗年幾拖嘅人 都係唔知點撚樣答佢。。所以就喺度問吓大家

其實講真 氹返女朋友呢啲嘢真係好小事 不過你班女人 呢樣要氹果樣要氹 做男朋友嘅有陣時都唔知氹唔氹你哋好 有第一次就有第二次 之後就要求越嚟越離譜 又要扮公仔 又要我哋死嗲爛嗲 大佬我哋都係男人嚟㗎 不過有陣時心情好都還可以 都頂得順 可以就下你啦咁 問題嚟啦 ! 我認真做咗成日野 我唔理係返工定返學定做功課做project 屌你仲要無啦啦發脾氣

話要我氹氹 我哋已經好撚攰 淨係想要少少私人時間做吓唔出聲唔諗嘢咁自己野 點知仲要無啦啦搞獲煩野我嘆吓 辛鳩苦 不過Face to face都還可以 都話可以攬下佢 錫下佢 嗲少少都話可以氹得返 傾電話嗰陣要氹真係救撚命 乜都做唔到 得一把聲 跟住唔氹又發脾氣 跟住又話自己冇嘢 其實自己嬲撚到爆 不過我哋依家都唔知你嬲緊乜氹番佢我自己聽返把聲都覺得打冷震啦 真係唔撚明點解啲女人咁鍾意叫我哋起電話氹你 出街面對面嗰陣唔發脾氣 一返到屋企就屌鳩你 然後要你氹 嘩 點頂

好啦 真係氹唔到啦 又屌鳩我哋 話我哋唔放佢做第一位 所以先氹唔到 我哋呢啲男人 朝頭早返工返學已經好撚攰 仲要賺錢養你(我唔係話女人賺唔到錢 而係我哋賺到嘅錢都會俾晒你哋用)返工返到咁辛苦 返到屋企仲要氹番你 其實真係冇乜心情啦已經 係就係有擺你係第一位 不過仲有第二位第三位第四位煩撚緊我 咁我係唔係可以唔返工 等你來養我 然後等你做我嘅唯一 然後乜嘢朋友都冇 唔出街 好似寵物咁等你返屋企

之後屌我都算 屌埋我哋同朋友出街玩唔理佢 大佬呀 同friend出街打波一個禮拜一日 同你出街一個禮拜三四日 然後話我哋唔理你 我頭都大撚埋如果當初唔係啲朋友 我可能只係一個油頭毒撚 唔識講嘢 唔識打扮 唔做運動 日日喺屋企 咁我又點樣溝你呀 我只可以講嘅係 依家嘅朋友先係造就今日嘅我 如果冇咗佢哋 你都未必會愛上我 可能你真係我嘅第一位 不過佢哋都有我心目中嘅位置 而家連基本見一面打場波都俾你屌 有陣時真係唔知點同你講好 講咗都晒鳩氣

不過點都好啦 係愛你嘅都會氹番你嘅 不過係好辛苦姐 我喺呢度都係發洩吓 純粹見到個朋友都係咁樣諗 原來咁樣嘅感受唔止得我一個有 不過發洩完都係要去氹番㗎啦而家。。 冇錯呀大家 我又要去氹佢啦而家

不過大家 到底要點撚樣先氹得返佢啊

絲打睇到呢篇嘢都諗吓你哋個男朋友係咪咁樣 你哋都要體諒吓我哋 呢篇野就當係比你哋有個男人角度睇下

身同感受

屌你都戇鳩鳩氹女仔都唔識

順便建議嘅ching可以講吓點樣氹番啲女 教下小弟


\chapter{原來自己係咁依賴老母}

[長文] 原來自己係咁依賴老母

講返少少背景先

自己屋企本身窮,我好細個果陣老豆就出去滾,所以十幾歲就老母兩個人住,老母學歷又唔高,可以做既野多數都係辛苦工,但佢都堅持賺錢為我供書教學,佢辛苦左成日每晚放工番到黎都仲會煮飯俾我食,但我仲成日會成日嚴送一樣,對佢好多不滿,又唔幫佢手做家務,有時佢問多幾句野我就已經會嚴佢煩,嚴佢咩都唔識,成日同佢嘈交。可能我由細到大都係自己一個人咁濟,根本唔識咩係自律,初中果陣每日返學訓覺,夜晚打機咁就一日,根本無理過任何屋企既人,好似只係為左打機而生存咁,老母俾我果零用錢仲瞞住佢用黎課金買點數,個金額仲要好大下,但佢完全唔知情,以為我用曬黎食野,出糧多左俾我零用錢亦都會增加。

咁到人大少少啦,開始有番自覺,知道佢既辛苦,但我就成日都口不對心,唔肯同佢講心底說話,我地之間既關係,係連母親節講句母親節快樂,影張合照,同佢講句多謝都會覺得尷尬。由細到大佢都同我講知識改變命運,於是我高中果陣就決定努力讀書,諗住將來報答佢,好好彩地考完,我真係入到想入既U,佢亦都好開心,好自豪咁同朋友曬命,收到果一刻佢好似仲開心過我咁。

但個天唔會完全順曬你意,響我二󲆱既某日突然收到醫院黎既電話通知我話佢收工搭車果陣遇上交通意外,我當堂驚到唔知點好,好彩有\lr{亻}{}響我身邊鼓勵我,陪我去醫院。由果陣開始我先發現自己係有幾緊張佢,幾咁依賴佢,好好彩地佢今次只係整親隻腳,坐左年幾輪椅,隻腳都慢慢行得到,佢唔方便行既呢段時間,我同佢相處既時間多左,關係亦都開始慢慢變好,到上年佢隻腳亦都好好彩好返曬,又無咩後遺症於是佢開始返番工,但之後其實我都睇得出佢每次放工番黎都好唔開心,有次仲聽到佢同朋友講話d同事成日針對佢,嚴佢隻腳之前有事,做野就住就住,做得好慢,但佢從來都未向我提過一句,每次番到黎都話今日好開心,完全都唔辛苦。

上個禮拜佢番到屋企果陣突然同我講今晚唔煮飯,話好累,好早就沖涼訓覺,同佢相處左咁多年,好少聽到佢講唔煮飯,但我都無理到就繼續做自己野,到半夜大概3,4點佢突然開燈整醒左我,我即刻睇下咩事,哦,原來佢肚餓係度搵野食,但越搵就越開始唔對路,佢眼神開始飄忽,面青口唇白咁,我同佢講既野佢都好似聽唔到咁,之後佢好辛苦咁同我講話心跳得好快,好似就黎跳出黎咁,咁我就即刻\lr{言}{}白車送佢入院,前前後後搞左兩個禮拜左右曬心電圖又剩,暫時確認係心絞痛,要排期照野,但未確認到係咪心臟病,住左一排院,醫院果邊都俾佢出院,只係開左脷底丸同阿士匹靈俾佢。

呢幾日番到屋企佢都好精神咁無咩野同未入院前一樣,佢仲講笑咁話星期一(即係22號今日就返得工添),但世事總係唔如意,今晚同佢食完返番屋企佢陣,佢就話條腰好痛,痛到係連行路都有障礙,幾經辛苦番到屋企,佢痛到即刻要攤響梳化,攤左一陣腰無咁痛,頭又開始痛,之後佢話連心都開始好似有野壓住咁,含完脷底丸都無效,佢開始連眼都擘唔大,情況仲差過對上一次入院,到我\lr{言}{}完白車 佢響神智不清之下竟然係叫我打俾老豆,擔心萬一有咩事得番我自己一個叫我去搵老豆,我眼淚即刻留到唔停,叫佢唔洗緊張白車好快黎到。呢半個鐘應該係我人生最萬長既半個鐘,由屋企到上白車,由白車到入院,佢冷汗係完全無停過,亦都不停同救護講話好辛苦,睇到我個心拿住拿住,但又唔想佢擔心我,只可以響隔離陪住佢一直係扮無野,扮堅強。

入完院要即刻推入急救房,我個人變到好慌,隻手不停咁震,好擔心佢有事,果一刻先發現原來自己所謂既慣左自己一個,只係因為當時既天真無知,萬一佢真係出左事得番自己一個都唔知點算。搞左一大輪,佢由急救房出黎我先鬆返口氣,雖然佢個狀況都係好差但個人都係清醒,醫院果邊話留院觀察幫佢詳細咁照下咩事跟住就開始推佢入病房,但病房依家根本唔俾人入去,我最後見到佢只係佢響急救房出黎既一瞬間,佢好勉強擘大眼咁望住我,用好冰既手拖住我講左句「唔洗擔心我,過幾日轉涼呀著多件衫」,聽完之後眼淚更加擁曬上黎,但響佢面前我緊係響度死頂,扮堅強,直到離開左佢視線淚線已經更加控制唔到。

有時真係好唔明到底點解個天可以咁唔公平,點解一個咁善良既媽媽要受咁多痛苦?由生完我果一刻,活左咁多年根本無享過福,生完我出黎好似仲害左佢咁,覺得佢搞成咁都係因為自己迫到佢咁。依家可以做既只係等醫院報告,又無得探佢,真係感覺好無助,好憎自己咩都做唔到,但無論點都好,我會堅強,響你同病毒搏鬥果陣照顧自己,做好家務汁到間屋靚一靚,等你番屋企,求下上天俾個機會我照顧番你,帶你享下福啦好嘛?

打呢篇野唔係為左博咩同情,而係想提醒各位好多野都講唔埋,生命呢樣野其實真係好兒戲。希望咁多位真係要愛錫你地屋企人,對佢地好,無論點都好佢地都係養大你既人,最錫你既人,佢地所做既野都係為你地好,唔好好似我咁出事先黎後悔。

簡簡單單咁講句,我愛你或者多謝你已經係父母最大既恩慰。


\chapter{覺唔覺「父母已經比哂最好既野你」難聽過粗口?}

[長文] 覺唔覺「父母已經比哂最好既野你」難聽過粗口?

講明先,如果你睇完我篇文覺得唔同意,覺得我係不孝嘅,咁你應該係個由細到大都好幸福嘅人,你可以出返去,冇你嘅事

個仔細個果陣就話佢唔夠啊邊個個仔叻,成績唔夠邊個好,點解人地個仔得你就唔得

大個左就話點解人地仔女做律師醫生專業人士搵得多錢,點解你一蚊家用都唔肯比

話父母已經比哂最好嘅野你,點解又冇出色又乜又物

但係啲仆街父母永遠唔會照下鏡反省下自己

人地老豆老母同仔女嘅關係就好似好朋友咁唔會有代溝,你有冇?

人地老豆老母住大屋揸靚車食好野成日去旅行去周圍見識,你有冇?

人地老豆老母得閒就陪下仔女傾下計相處下鼓勵下佢地想做嘅事,你有冇?

人地老豆老母唔會拎人地同自己仔女比較,唔會嫌棄佢地做得唔好,你有冇?

窮,成長環境惡劣,衣食住行好多野本身已經要就住就住

食都食唔飽,以前中學飯錢一個禮拜得200蚊,即係一日40蚊,果陣學生餐平極都要廿到尾三十頭,仲要加埋早餐同埋放學食野,其實係食唔飽,而且係冇餘錢平時同同學仔出街玩。唔好同我講咩早餐食個包就算,極唔健康之餘,仲會影響朝早上堂表現,相信大家都見過女同學試過早會暈低,通常都係冇食野加埋天氣熱又要企係到曬\lr{午}{}\lr{食}{}又貪平買啲濕鳩撈麵當一餐,緊係發唔到育。

養兒防老觀念就覺得生多幾個第時養返你退休唔憂柴憂米,好似生仔生女一定會一表人才出人頭地咁

大佬呀,你買股票都唔係穩賺啦,更何況生仔咁大「投資」?

到左仔女大個左,出到嚟唔係父母心目中諗嘅一樣,就會怨啲仔女冇用,冇出色,唔生性

係個咁差嘅環境中成長,廢老廢家長只會踩到自己嘅仔女一文不值,講咩人地咁叻你咁蠢第時屎都冇得食乞衣都冇得做

自己就打份牛工早出晚歸,同仔女一齊傾計、食飯嘅時間根本少到數唔到

仔女好多時侯發生咩事都係自己一個面對,好孤獨,有咩問題都係自己解決,有部份就會學壞做仔妹。

有好多人讀緊書果陣已經要出嚟做pt,叫比啲零用錢自己洗

但係其他同學可能係冇經濟負擔,專心學業或者發展自己興趣

而窮人細路要發展自己興趣?首先要自己出去打工,想問屋企人攞錢根本冇可能,因為平時去茶記加兩蚊嗌凍飲/去麥記要加大,老豆老母都會屌,更何況去上咩興趣班或者學咩樂器。係街頸渴想買野飲都唔比,老母會叫你返屋企飲水

窮人細路大個左之後一樣打份牛工,搵雞碎咁多,唔敢結婚生仔 。睇新聞見到,原來人地老豆老母送比仔女嘅生日禮物可以係一層樓。望返自己,細個老母買個百零蚊嘅玩具比自己之後仲比說話我聽,又話貴又嫌呢樣嫌果樣

身邊大把人,由細到大都冇房,做咩都係個廳做,張床都係廳,冇私人空間

其實上一代只要有份正正常常嘅工,努努力力到而家都會有層供完嘅樓,亦可以照顧到自己起居飲食,唔需要仔女比家用。再勁d嘅,可以幫仔女比埋首期,甚至送層樓比佢。其實,老豆養仔 仔養仔 先係正常,而唔係養兒防老。如果你連養一個細路嘅條件都冇嘅話(講緊係照顧基本起居飲食,有舒適嘅成長環境),咁就唔該你唔好生啦,唔好害左個細路,細路係無辜。

去返窮人家庭,父母只會講已經比哂最好既野你,你仲想點?

但係你有冇問過自己,你比過啲咩仔女?

如果你話,供書教學,比飯你食

咁我想問,有邊個父母唔使?

呢啲係做父母嘅責任,點解可以講到皇恩浩蕩,講到好偉大咁?

如果你去領養,佢第時出人頭地,做人好開心好有意義,咁你咪偉大囉

如果唔係嘅話,你唔好講到自己幾咁辛苦,生仔係你地自己決定,要怨就怨你自己,係你地自己攞嚟

仔女變成點你地有責任,唔係咩都賴落人地到

可能你話,咁窮人細路比心機,追返啲時間咪得,努力向上游

但係事實上,係極難,因為由細到大嘅成長環境唔同,見識唔同,社交圈子都唔同。你睇下讀醫科讀法學入面有幾多係中產名校出身,有幾多屋邨仔屋邨中學出身就知。

食得咸魚抵得渴,冇咁大個頭就唔好戴咁大頂帽

講到尾,窮人唔好生仔


\chapter{求開解}

[長文慎入] 求開解



本身有抑鬱症

由細個開始已經好想死但係又怕痛

算係等死嗰種 即係搭車想像下自己出車禍死反而令我個人更加平靜

家庭問題令我一直覺得為其他人付出先係我成個人嘅價值所在

觀念上亦覺得為左自己去生存係一件可恥唔好嘅事

所以我應該要好似一隻棋子咁

為左其他人成全大局要去犧牲自己

咁樣講好似好鳩咁但真係我由細到大嘅價值觀

我亦都一直覺得自己就係應該要咁樣生活落去

但我最近發現我冇我自己諗得咁乾脆

開始發夢之後由和理非慢慢變企中前

我好想好想企前啲

雖然有怕俾狗咬

但諗到咁多手足犧牲左咁多 我都唔可以辜負佢地

但係我開始突破唔到自己心理關口

之前有朋友一齊都好啲 有個識嘅人喺隔離感覺上安全啲

但之後啲朋友一係就唔出 一係就自己另行組隊

得翻我一個

和理非集會我都敢自己出

但遊行嗰啲真係會驚

諗起之前企前啲嗰陣腳都震埋

自己一個仲驚

然後覺得原來我冇自己想像中咁勇敢堅強

我冇想像中咁可以為其他人付出一切

跟住就覺得我人生嘅價值觀好似崩潰左咁

我曾經好想追求嘅品格 原來我一樣都冇

我到最後都只不過係一條自私撚

然後啲抑鬱開始翻翻黎

曾經覺得要同手足一齊撐住生存落去

宜家覺得我配唔起企喺手足嘅身邊

好想死 雖然咁講我知好賤 對唔住所有受傷嘅人

但係我希望吸tg吸死又好 俾槍射死又好

至少我條賤命好似叫付出左少少咁

最近休息左一段時 同大家講聲對唔住

呢段時間入面比較少出去

轉左主力去畫文宣同twitter

每次遊行都唔想睇live睇telegram 好辛苦

但都忍唔住㩒入去睇

每一日心入面好似有另一把聲同我講

我嘅本性就係一條自私撚

選擇好行邊一條路就要堅持住自己嘅生存方式

如果選擇要退場就唔好再翻轉頭

但係我唔想

我知我係一個好撚自私嘅大仆街

不過我都想盡我努力去唔好做一個賤人

我覺得我真係開始好迷茫

究竟我可以做啲乜

同埋究竟我幾時先可以死

-

Btw想順便問埋

單拖好撚驚點搞


\chapter{【以香港作為信仰】}

https://lihkg.com/thread/2335214/page/1

\#1香城法絲

•3 年前

【以香港作為信仰】

國安法後抗爭進入休整期,抗爭陣營充斥住無力感,出現好多內部矛盾同互相指責。勝出民主派初選嘅張可森曾寄語希望大家以「香港作為信仰」,雖然信仰對香港人嚟講好陌生,但共同嘅信仰正正係我哋呢一刻最需要嘅嘢。

信仰戰勝極權人

擁有信仰唔代表係信奉某種宗教或神明,宗教只係其中一個表現信仰嘅方式。信仰係一種我哋生而為人,選擇相信嘅價值觀;可以係生活方式,亦可以係人生意義。喺極權統治下,人民失去集會自由同表態嘅空間,難以透過參與群眾運動凝聚力量,個體都變得非常脆弱。縱觀歷史長河,好多經歷過極權統治嘅民族都各有宗教信仰,幫助佢哋喺苦難中維繫群眾。猶太人散落世界各地,因為猶太教義而決心復國;波蘭人以天主教作為後盾,喺共產統治下保存到文化同傳統;藏人信奉藏傳佛教,不怕犧牲同中共戰鬥到底。極權冇辦法奪去人民內心嘅信仰,信仰係人民對抗極權嘅希望。

為信仰而抗爭

當街頭抗爭稍為靜止,有人會為表面抗爭人數嘅減少而灰心喪志,用冷嘲熱諷嘅方式指責其他人付出唔夠多,有從眾同比較心態。信仰係每個人內在嘅信念,反而唔會輕易受外在因素影響。當抗爭漸漸失去畫面同關注,就係真正考驗信仰嘅時候。我哋甚少見基督徒會因信徒嘅多寡而自怨自艾,甚至放棄信仰。哪怕世上只剩下一位信徒,佢哋仍然視傳福音為使命。當你立志以抗爭為使命,就無需太在意民智嘅高低、計較各人付出嘅多少。抗爭唔係因為香港人值得與否,而係要對得住自己嘅身份;堅持亦唔係因為從眾,而係為自己心中嘅信仰。

相信香港呢片土地

面對極權,有人選擇移民,並經常引用猶太人做例子,表示離開唔代表放棄抗爭。猶太人作為離散民族,飄泊流亡過千年,終能復國全因為有共同信仰去維繫自身共同體。佢哋堅信猶太教義中復國嘅預言,即使散落各地都仍然心繫以色列本土。若以香港作為信仰,「煲底之約」就等同於香港人信仰嘅預言。世上只有一個煲底,無論係流亡海外定係身陷囹圄,我哋都期待煲底下除罩相擁嘅一日。移民只係手段,大家追求嘅係喺香港呢片土地上實現民主自由,而非到外國去享受自由嘅果實。維繫香港人嘅係紮根本土嘅信仰,就算被迫要逃離家園,都要確信香港呢片土地係唯一安身立命之所,係信仰嘅應許之地,別無他處。

相信每一個香港人

有人擔心香港人會放棄抗爭,背棄曾經相信嘅價值,但信仰能透過個人經歷得以鞏固;有人覺得雨傘後難再有大型抗爭,結果五年後香港人成就咗一場波瀾壯闊嘅革命。信仰嘅種子早已透過共同經歷栽種喺我哋心中,並會隨着實踐中萌芽成長。以香港作為信仰、相信香港人依個共同體,過去一年嘅抗爭足以見證香港人嘅勇敢、智慧同團結。不論你、我、他或她,早已成為香港嘅信徒,一息尚存,抗爭到底。

相信香港終會重光

缺乏信仰嘅人容易迷失,因為太過著眼於最後嘅成敗得失。信仰嘅強大在於即使睇唔到結果,你都願意繼續相信。正如基督徒唔會質疑耶穌會否再臨,只係未知幾時重臨,信徒應該著眼於實踐信仰嘅過程,努力去傳福音。同樣地,我哋要堅信香港終會重光。如果連香港人自己都唔相信會贏,試問又點可能實現夢想呢?喺黎明來到之前,作為香港人要莊敬自強,生活上無時無刻緊記要實踐信仰。哪怕再漫長嘅黑暗,擁有信仰嘅香港人都能化為一點燭光,照亮身邊孤單困倦嘅手足,共同迎來重獲新生嘅香港。

用一生證明香港嘅存在

假若你缺乏宗教信仰,又或者對香港未來充滿疑惑同困倦,請別輕言放棄。懇請你再次相信香港,相信香港人。將過去一年共同經歷嘅苦難、手足之情同願景,視為一種我哋甘願為此抗爭、堅持同犧牲嘅信仰。

巴勒斯坦裔學者 Edward Said 講過巴勒斯坦人嘅命運,係要用一生去證明巴勒斯坦嘅存在。作為香港嘅信徒,我希望可以用一生去證明香港嘅存在。只要我哋仍然相信香港,香港就會存在。

在晚星墜落徬徨午夜,願各位堅守香港人嘅信仰,煲底見。

作者:良善如你


\chapter{【長文】其實月夕行動好反映到我地目前嘅抗爭陰暗面}

\#1阿嫲飲可樂•4 年前

唔知大家係以一個咩expectation入黎睇呢篇文呢?自己對於標題有無答案?

小弟12年開始接觸社運(國教),去到今時今日,其實香港嘅抗爭變化好大

以前睇抗爭 只係會二分法 公民抗命v.s.武力抗爭 呢種諗法其實都根深蒂固落我地每一個人到

導致我地宜家會覺得“進化”、“跟上” 就係將我地認爲無乜用嘅公民抗命 轉化成 更激烈武力抗爭

然而,我地過去一年經歷話我地知,武力抗爭帶黎嘅推進其實同公民抗命,其實老實講真係相差唔多

所以我都可以大膽引用三步中出西門巴嘅說法“一般國家推翻政權既方法已經證實左冇用” 【附錄1】

面對目前好絕望,唔係幾覺得有未來嘅社會,我地嘅抗爭其實都慢慢將香港人陰暗面呈現出黎

以下講幾個見到嘅:

經歷多次嘅慘淡收場,我地係咁比藉口自己逃避,以下幾個,恕我唔能夠一一盡錄

“CCP咁強大” “班黃絲港豬唔值得贏” “2046仲街頭抗爭” 上面嘅statement 其實錯唔嗮

但我地如果一路係咁諗,其實只會固化思維,令我地走唔出 “我地贏唔到”嘅局面

自己比藉口,然後相信左藉口,沉迷左藉口,導致心態上已經輸左,呢個係第一點

第二點就係習慣性嘅揶揄同批評,而依兩樣野嘅背後係我地嘅自大

“我食鹽多過你食米”,依句我地咁討厭嘅句子其實都幾能描述我地嘅心態

月夕行動一出,已經有人話起名好On9,亦都有人話set rb/遊行 已經無意思

面臨呢個情況,搞手選擇批評批評佢嘅人,唔願意改善

之後見無料到,其他人就笑搞手On9,又或者屌佢水人出黎

我地嘅自大,令我地難以接受其他人,令自己接受唔到批評之餘

亦變得好容易批評人地,lunch哥就係一個例子

[我地好多時會用自己嘅經驗去同人講唔work,係呢度唔討論經驗係咪可以必然反映個件事嘅結果。比起講唔work,提出改善者更應該善用已有嘅經驗改善個plan,而唔係否定佢(我嘅諗法)]

第三點,太識走精面

呢個其實可以係優點,但逐漸變成缺點黎

例如一開始嘅遊行已經由中途站行(唔想係銅鑼灣逼)

到街頭嘅時候唔Blackblock唔帶gear淨出

到現時,準備移民然後繼續抗爭,又或者有行動唔出,淨做鍵戰

上述三樣接近無敵地出左力 又最低成本同危險 我唔係話有問題

但走精面其實都表示緊,出少左力

其他手足要逼你個份,由其他手足承擔被捕風險,由係香港嘅人承擔赤裸裸嘅壓力

係呢度唔評論上述行爲,因爲我覺得過到自己個關,你咪做

但宜家係要大家反思走精面呢個雙刃劍,係成爲左你嘅藉口,定係幫緊你抗爭?

講完,其實自己喊緊

今日等大家出現,見自己個區包含在內,就即刻出,出到去都預左少人,無諗到轉過頭就取消。等多陣就翻去,望下自己有份嘅tg grp,睇上少少先發現原來好多人都唔睇好,甚至係度冷嘲熱諷緊。可能係對家玩分化,但我相信都有唔少係自己人鬧緊自己人。我直到今日之前都無放棄過抗爭,放棄過香港。但今日有一刻,我一個人望住個十字路口同盞紅綠燈,我想放棄啦。然後我先諗翻起香港人其實一直都係咁。其實一直以黎都係冷嘲熱諷嘅社會,就係一班投機取巧嘅人。一直以黎就係上連登勿認真,一直學緊嘅就係顧掂自己唔洗理其他人。但再諗多諗,呢啲咪就上面講嘅比藉口自己,我咪又係想屌班唔出黎嘅人?

到呢一刻我先明,所謂嘅革命,第一步唔係令到社會有乜乜改善,唔係五大訴求

第一步,係針對我地香港人心態同風氣嘅革命

透過革命,革走我地成個社會嘅不足同唔好嘅風氣

透過不斷嘅失敗,磨走我地嘅自大,學識檢討

透過未來越黎越難,再無辦法走精面,唯有逆流而上

透過再無藉口,令自己認清現實,慢慢穩固革命資本,心態由輸變贏


\chapter{何謂分化}

何謂分化

1.分化是什麼?

「分化」其實好簡單,有一句說話可以解單總結。就係宇宙級革命家毛澤東所講既:

『所謂政治,就是把我們的人搞得多多的,把敵人搞得少少的。』

特別係最後一句「把敵人搞得少少的」,就已經可以概括曬中國共產黨咁耐以黎既「分化術」。

就是要將敵人多而化少,分而破之。將敵人由200萬變成100萬,100萬變成50萬。到最後孤立少數主要敵人,一舉擊破。

2.為何需要分化?

要了解「分化」,就首先要明白:點解要分化。

世上有無限咁多兵器,飛機大炮,刀劍槍械,仲有嚴刑峻法。你點解要諗到用分化?

好簡單,因為如果你面對既敵人,係無辦法用上述武力同法制解決既,就唯有用「分化」。

『把我們的人搞得多多的,把敵人搞得少少的。』--依句說話據講係毛澤東係延安時期寫低既(待考據,可代補充),延安時期大致橫跨抗日戰爭去到國共內戰。共產黨以延安為根據地,毛澤東同時面對緊國民黨、日軍、再加埋黨內唔同派系、甚至蘇共力量既壓力。

對毛sir黎講,軍力唔夠、人唔夠、甚至自己友唔夠,點做?分佢老母化。

當你既敵人係一大班群眾,你無辦法殺曬咁多人,拉曬咁多人。武鬥不行,就唯有文鬥,其中最強技術,就係「分化」。

3.分化的原理:敵意

去到依度,你或者已經諗到,到底分化係咩黎。其實好簡單,一幅圖可以表達曬:

係咪好簡單呢哈姆大郎。其實「分化」,最重要既只有一樣野:敵意。

聽落好似好玄妙,但其實大家由細到大都一定有經歷過類似既樣野:

A:我鐘意玩PS

B:我鐘意玩XBOX

C:on9仔先玩XBOX,A,我都係玩PS,你以後同我玩啦,唔好同B玩啦,你唔覺B ON9咩

A:屌,我一直都覺,佢打J成日睇三上悠亞,我覺得河合明日菜先正

假設依個世界唔存在PS/XBOX之間既分界線,A同B係可以共存,而C亦唔能夠勾起A既敵意,去離開B。

由此可以再推導更深一層既分化術原理:

事物有差別唔代表對立,但強調事物有差別,可以引致差別之間既敵意,進而分化。

面對一個團結而難攻既群體,只要有人劃底一條線,而團體入面所有人都無辦法消除依條線,甚至不斷討論到底條線既位置係邊(唔一定好似幅圖咁係正中心),引發兩派人之間既敵意。

到最後,依一個圓型/群體,就一定會被條線分割。換言之,姐係割席。

4.如何實際執行分化?

上面係理論野,講下實際野。

現實生活唔係個都玩PS同XBOX(我兩樣野都無),咁你點樣分化一大班人,甚至分化一班本身有共同目標既人?條線應該擺邊?

好簡單,當對方只有一個目標,咁你就要強行演釋目標既差別,然後強調差別既存在。

例如,有20個人都想去日本旅行。

本來打算一齊出發,結果有第21個人因為各種原因(妒忌/無錢/反日...etc),而唔想依班人去得成日本。

佢就開始同20人入面既第19個人講:「其實呢,我覺得識去日本既,一定係去大阪,東京真係唔抵去。你覺得呢?」

第19個人未必對大阪或者東京有意見,但當佢拎出黎同20人討論既時候:

關西派既第11人就開始吹奏大阪;

第6人係東京派,佢就開始反對第11人既意見;

第13人同第11人係朋友,佢就會幫口反對第6人;

第7人一直暗戀第6人,佢就又會幫手屌第13人同第11人:

結果,依20人最後要分開兩批人去,去大阪個10人,又因為訂唔切酒店而延期,只得去東京既一半人去到。

第21人又再出現了,同去到東京既個10個人,入面既其中1人講:

「其實呢,我覺得去識去東京既,一定係去原宿,涉谷真係唔抵去。你覺得呢?」

--直到最後得番1個人。

第21就同最後1個人講:係囉,去咩日本姐,日本on9仔先去。

然後就無人再去到日本了。

聽落好似好荒謬,但如果真係發生,你會唔會有少少心寒?

因為第21人佢咩都唔洗做,唔洗子彈唔洗槍,就可以達成目的。

現實生活入面,情況當然唔會咁單純,你要分化對方,亦唔係一句「你覺得呢」就可以解決。

但無論任何行動、目標、想法,其實都有無限演繹方式。

只要你拎住其中一邊,再強調另一邊既差別,勾起兩派人既對立同情緒,就可以逐步瓦解對方,把「敵人搞得少少的」。

再引一段黎自毛澤東既文章《矛盾論》,解釋如何瓦解敵人:

「《水滸傳》上宋江三打祝家莊,兩次都因情況不明,方法不對,打了敗仗。後來改變方法,從調查情形入手,於是熟悉了盤陀路,拆散了​​李家莊、扈家莊和祝家莊的聯盟,並且佈置了藏在敵人營盤裡的伏兵,用了和外國故事中所說木馬計相像的方法,第三次就打了勝仗。列寧說:要真正地認識對象,就必須把握和研究它的一切方面、一切聯繫和『媒介』」

5.「我」係咪分化撚?

你或者會問:喂,咁東京真係唔抵去喎,我又真係覺得玩XBOX既真係on9(再次強調,我兩樣野都無),咁唔通我屈就自己?

唔通我提出相反既意見,唔通我唔同意某個想法,我就係分化撚?

答:仲記唔記得,上面講到,點解要分化敵人?

因為敵人係無辦法用武力解決既大型群體,目標一致會影響我方既利益。

咩係「敵人」?

敵人唔係你唔鐘意、你睇唔順眼既人、唔同意你意見既人,而係會影響你實際利益既人,同你爭女/仔,同你爭機位,同你爭奪政權既人。

你分化佢,就令佢無辦法同你爭奪利益,令佢失去敵人既資格。

調轉講,即使你係團體入面持有相反意見,如果大家仲係目標一致,仲係一個可以威脅對家既團體,不論你對東京有咩意見,你再鐘意玩XBOX,只要群體仲可以一齊行動,你提出再多既反對意見,你都唔係分化撚。

6.「我」/佢」係咪會引致分化?

分化點解咁勁,毛澤東點解用足一世,當然唔係咁簡單。

因為人人都可以被分化而不自知。

先問下大家一個問題,如果依個世界得兩個人,你同我,咁你點樣分化我?

無錯。係分化唔到。

要分化,就要有互相對立既敵意。

我對我自己點樣有敵意?你先搞到我人格分裂可能得既。

要分化團體,一隻手掌係拍唔響。

「分化」最無敵之處,係一班人入面,只要令有一個人或少數人有敵意(好似上面既日本旅行團例子),個份情緒就好容易散佈,令成個團體受感染。

甚至你本身即使唔係想分化人,只係想做和事佬,出黎講句:

「起,唔緊要啦,咪去完東京再去大阪。」

都可能會被關西派既人屌:

「點解一定要去東京先再去大阪先?明明大阪近啲呀依家。你係咪東京派既分化撚!?」

如果你沉不住氣,屌番轉頭,咁笑到最後既,就只有唔想大家去日本既第21人。

所以:分化撚唔一定真係敵人,只要有敵意,引致分化既,一樣可以係你同我。

7.如何防止分化?

咁係咪已經玩撚完?

或者你會覺得,分化咁勁,大家中左招都唔知,仲可以點?

毛澤東成世人經歷過大大小小的黨爭,係幾乎戰無不勝,仲要笑到最後。精通分化,係咪就真係天下無敵?

的確,只要對方係一個群體,而目標有多種演繹方式,當我方派出唔止一個分化撚,只要手段恰當,思路清晰,幾本上係無得輸。

當劃出對立既界線,帶動敵意,敵人只要on99生勾勾,開始互鬥,鬥顏色夠深,鬥接近中心,就會被分化。

由圓形變成半圓,由半圓變成四分一圓,再變到屍骨無存。

不過,天下間,無野係完美同無缺陷既,分化都係。

分化既弱點只有一個:胸襟。

唔係依啲胸襟,不過見你肯睇到依度,獎勵下你姐。

回歸正題,面對分化,唯有胸襟可以抵抗,要練功必先練心,

唯有胸襟,先可以化解分化既核心元素:敵意。

你要接受:

所有目標都唔會只有一個途經

所有行動都唔會只有一個方法

所有方法都唔會只有一個效果

所有理念都唔會只有一個演繹

所有目的地都唔會只有一條路

唔同人有唔同既意志

唔同人有唔同既能力

唔同人有唔同既專長

唔同人有唔同既手段

唔同人有唔同既表達方式

大家有唔同既口才、想象力、行事風格、做事節奏。你行得慢,我行得快,你鐘意坐車,我鐘意坐飛機。你鐘意三上悠亞,我鐘意河合明日菜。你鐘意去東京,我唔單止唔鐘意大阪,仲比較鐘意去歐洲--但唔緊要,我地既目標仍然一致,無野分化到我地。

一個群體之所以令敵人無計可施,要不斷分化,正因為團體入面既人,有胸襟去兼容所有既可能性。

就算有任何撚屌係團體入面劃一條線,你都有胸襟去跨越依條線,無視敵意既起源,而唔係將條線另一邊既人,當係仇家,亦唔再當佢係同路人。

只要記得目標,記得令團體之所以為團體既意志,有胸襟去包容唔同意見既人,就可以令「分化」無用武之地。

記住,目標。意志。胸襟。

兄弟爬山

各自努力


\chapter{公屋出身 係咪普遍冇咩禮貌同家教?}

公屋出身 係咪普遍冇咩禮貌同家教?

見d住公屋同事 好似唔識禮貌咁

同老闆搭lift 唔識讓老闆出入lift先

成班人入又入先 出又出先 阻住個老闆

玻璃門又係呀 夠膽死成班要老闆幫佢地hold住玻璃門比佢地出入

新入職個班 唔識叫人X生乜生先

直接叫人英文名 以為好熟

未熟個陣好心叫人禮貌d 陳先生 陳小姐 陳女士

冇幾日就 Peter peter咁叫 個同事高佢幾級…

派月餅就最撚頂癮 唔洗錢 老闆派比你

係到厭三厭四 話依個品牌唔好食 叫老闆下年轉第二個品牌…..

我心諗 公屋父母有冇家教依樣野?

識既公屋仔女係冇咩家教

識既公屋仔女幾好禮貌


\chapter{其實成日討論窮人生仔正仆街,大家認為要有幾多錢or 總資產 先唔算窮人}

其實成日討論窮人生仔正仆街,大家認為要有幾多錢or 總資產 先唔算窮人?

我都見過好多老豆老母錫住黎養 但唔識教

送到去外國只淪為別人眼中嘅敗家仔

有唔少男男女女都會中意玩,做usb/公廁

讀書就買essay

得閒吸下大麻

你地係香港就慢慢匯錢俾仔女繼續提供美好環境俾佢地玩啦

呢啲係我90後係外國讀書見番黎嘅

愈有錢愈危險

我自己唔算好,但因為屋企無錢,人地玩嘅時候我都只係餐廳做下part time ,會知道每一分錢得來不易嘅

認真嘅

你睇下富豪仔女嘅花邊新聞就知

好多都吸毒,亂搞男女關係


\chapter{其實香港人係有套共同信仰,當呢個信仰動搖係會開始懷疑人生}

其實香港人係有套共同信仰,當呢個信仰動搖係會開始懷疑人生

呢篇文我一路諗一路打冇組織過,所以好難睇,如果你仲願意睇可以一路睇一路思考,持開放態度一起討論幫我補充。

——————————-

其實香港人係有套共同信仰,當呢個信仰動搖係會開始懷疑人生意義。

大家由細到大都被灌輸一套社會信仰,只管信不敢思考,過於思考會不快樂會被歸類為離地不成熟等等。

簡短講下呢一套信仰:

每個人一出世就被賦予階段性任務,首先要完成12年免費教育,填鴨式學習做練習狂考試,考小考中學考大學。只要考得到只要考得好就高人一等,將來會有好既成就。大家,包括家長老師同學都唔會思考呢套教育制度係咪真係可以幫助大家發展長處,即使有人大膽提出就會比人話 “不嬲係咁架啦” “一向都行之有效” “諗埋啲無謂野對你有咩用?” 大家只管信就得!

成長過程中被灌輸,父母同老師係絕對正確,警察係好人,政府收左稅處處為市民服務,讀唔到書唔會成功,你一定要跟住有權威既人既指令去行。”我食鹽多過你食米” “個個都係咁㗎啦” 就係咁樣大家就只管信不要問!

專業人士應該係備受尊重,因為讀書讀得多成績好所以叻過曬讀唔成書既人。警察政府應該備受尊重,因為佢哋係正義係為我地好。身為警察既自己又會覺得着起套制服行出黎就應該得到市民尊重!呢一切大家又只管信,然後用盡人生頭20-30年想做專業人士打政府工或者化身正義之士。無人問呢一切標籤係咪真,但你不用思考只管信吧!

讀完書出到社會,得到新任務 “買樓 結婚 生仔”

出到社會後香港人又相信工作就係為左買樓,買樓好重要,無樓就結唔到婚,生仔好重要唔生唔得。無人問點解,大家只管信!即使有人大膽問:香港樓咁貴唔合理喎做樓奴供成世,日日返工OT無錢收,唔OT怕比人炒,為左間豆膶咁細既屋做半生機械人,咁嘅人生有咩意義?成個系統好似好唔個理喎!得到嘅回覆係“個個都係咁㗎啦” “呀邊個邊個咪成功囉” “努力啲一定得架” “成熟啲啦唔好咁離地” 你信啦!個個都係咁做架!

香港人嘅人生意義完全基於呢啲信仰,大家從來冇諗過人生為乜野而活,因為跟本無咁既時間去諗。或者咁講,因為大家個腦已經容納左一大套信仰,再冇思考空間,就算有人開始思考開始提出問題都會被嘲笑被喝止。”一路都係咁架啦!” “從來行之有效你諗咁多做乜?”

過去幾個星期,香港人都進化左,大家開始知道一直相信既事情唔係真相,父母老師唔係絕對正確,唔係全部警察都係好人,政府唔係處處為人民着想,專業人士都好多蠢人衰人,而身為專業人士都唔代表會備受尊重等等⋯

大家開始對呢套行之有效既信仰懷疑,亦即開始清醒。但係同時又好似失去緊人生嘅意義。有人會崩潰想結束生命以死控訴,有人會死守信仰排斥覺醒嘅人,有人會清醒過來開始思考人生意義開始思考到底乜嘢係岩乜野錯。

最後我想講,即使你已經崩潰千祈唔好結束生命,我地要不斷思考,不斷追求真理,進化過程係痛苦但總係會有同路人一齊進化,你不孤單!


\chapter{勇武出動的目的係咩?}

勇武出動的目的係咩?

阻止班狗濫捕市民?

破壞黨鐵監控?

打狗?

禍及家人呢?講左成年冇人做?

有冇諗過咩行動可以令政府回應五大訴求?

勇武出動,完成目標就好撤退,咪又磨爛蓆,話咩前線保護後排,後排等埋前線一起走

三個月了,一啲進步都冇。起完火、起完rb就走啦,搏拉咩 次次起完rb就企哂係到,屌!班狗衝的時候咪又係走,點解唔走左先呀?走=輸?

你又唔係想同班狗埋身肉搏,係既話就唔洗整rb啦。根本整rb只係提供一個虛幻心理暗示,令你覺得安全。就算你用火魔法,咁遠距離你點掉得中班狗?

依家勇武出動越嚟越頻密,依件係好事,但無腦衝就一定仆街,因為根本啲狗就無受過任何傷。

老豆搵仔啲資料咁多但無人識用敵人在明我地在暗都唔識好好利用自身優勢。放火判終身又敢做,接放學又冇人做。大家都係想班狗驚啫,點解唔做?過唔到自己心理關口唔洗諗住贏

你地做得勇武唔係上哂裝就叫勇,心態上見到狗就跑有咩用?

你地成日話撚狗,有冇諗過班狗根本唔care?佢地會驚一首歌?訓醒未呀?成日自high!點解唔播班狗啲資料呀?點解唔預告接邊隻狗仔放學,製造白色恐怖呀?

撚狗唔係用激光筆、圍狗屋佢地就會驚。佢哋坐係狗屋涼冷氣等你地慢慢嘈,再出來打人不知幾爽。

再繼續送頭唔洗一年,前線就會拉哂。

好好諗清楚勇武的意義係邊。


\chapter{大家成日鬧廢老娶大陸女人,有冇試過自己老竇就係呢個廢老?}

\#1無能用為力•5 年前

大家成日鬧廢老娶大陸女人,有冇試過自己老竇就係呢個廢老?

事發於8年前,我爸突然同我講佢就嚟結婚,我會有個妹,係佢返深圳工作識嘅廿幾歲四川鄉下女人,好純品對佢好好。生母已過身好多年,開心老竇搵到人陪佢。

頭幾年,個女人喺香港生完個女之後一直同同佢前夫生個女深圳住,我爸一半時間香港,一半時間大陸,大家相安無事。

2014,佢哋落曬嚟。本來話個大女唔落嚟。個女人家姐都係嫁香港男人,香港男人要求佢擺個仔喺四川鄉下先娶佢,但我爸份人心軟,最後養埋呢個契女。

嗰陣時,我都接受個女人,斯斯文文,大家都客客氣氣。嗰兩年我讀緊書好chur,我爸時不時鬧我間房執得唔整齊,成日鬧我係咪當屋企酒店,返嚟淨係食同訓,冇參與家庭活動。

幾年後我先發現係個大陸女人陰啲陰啲喺小事上煩到我老竇好躁,小小事都講到我唔為家庭付出咁。老竇仲話我搵到嘢做就可能要搬走,因為個女人老母都可能會落來,唔係冇人照顧。我當時內疚,覺得真係冇花時間喺屋企,又阻到人幸福,應承返工後會搬走。

2016,我哥買樓,借我老竇少少錢。我老竇因為其他事借過佢錢,所以借返畀佢係合理事。但係突然個女人現形,話我老竇畀曬啲錢人,佢就乜都冇(其實佢已搶咗好多嘢)。我聽到佢為錢為樓鳩嗌鳩喊,忍唔住走去鬧佢「屌你老母」,自此正式關係破裂。

屌佢老母後開始,佢好多小動作。試過明知我未返無啦啦鎖大門,差啲冇得入。我出嚟做嘢第一個月未出糧,佢阻止新請嘅工人姐姐煮飯畀我食,因為我冇(未)畀家用。夜晚知我訓就會出廳睇片或傾電話滋擾我,知我趕就塞住門口,諸如此類。我同我爸去旅行,佢就喪We7我爸個細女啲可愛相,講到個女冇人陪好陰公咁,又話每次我爸出遠門都好驚,點解佢連樓都唔加佢名令佢咁冇安全感,佢只係想要安全感。

最近跟足劇本發展,個女人四川呀爸死咗,佢老母冇人照顧好慘,想借宿幾日。慢慢唔知點解又會變多咗日子,講講吓又話長住。我每一次都強烈反對,講曬粗口鬧我老竇以前唔守條線,今日都仲唔守,以後點?我老竇好嬲我咁煩,話佢有咩辦法,控制唔到呀婆嚟唔嚟。我脾氣差,好易畀個女人每日啲小動作激親,積積埋埋咁上下就又會爆一次,同時個女人都係隔一排就大爆發,搞到我爸好心煩。

我:成個屋企畀個女人搞到咁,個個都嬲你你開心咩?呀婆同你住你開心咩?

父:關你咩事呀?

我:層樓我媽有份幫你供,所以關我事。

父:你問吓法律關唔關你事!

我:香港就係畀你呢啲人搞到咁!明知做錯都唔堅守,成班蝗蟲入曬來!

父:關你咩事呀?你好憂心香港咩?吓?咁唔鍾意你唔走?咁入境處依家確實冇阻止呀婆入境喎!

我聽到呢啲,已知道佢冇撚得救。我同佢關係依家都如個女人所願,變得好差。

女人背景:四川人,到深圳打工,來到香港做主婦,但其實都係交曬畀我個已退休老竇湊,日日喺度鳩揈。最近一年去做揼骨。我返工後出錢請咗個工人姐姐畀屋企,更加唔湊細路。大女:剛升上名英中;細女:普通小學。

佢衛生好差,佢M到嗰期廁所會有血,地下有時都會有。佢兩個女都經常唔沖廁、食完嘢唔執,牆上仲有鼻屎,連我工人姐姐都話佢哋喺床度食麥記好恐怖,床仲有污糟底褲,連我工人都驚惹曱甴。成堆蝗唔肯講廣東話,只講蝗語或鄉下話。

我爸取態:知個女人貪錢,話佢「唯一」做得最錯係娶咗個女人,其他嘅錯佢唔會認!覺得對我哋冇拖冇欠,已分好層樓嘅去向,話個女人冇我哋諗得咁差,對佢好好,想老咗有另一半照顧,亦都唔想細女(8歲)冇完整家庭,所以有諗過離婚但決定唔離,離婚仲要分身家。

我覺得我屋企已冇得救。我老竇都70歲,第時重病或老人痴呆,個女人實有方法再分隔我同我爸,可能直頭冇人通知我。放低我老竇搬出去,係對自己心理健康最好,但我走咗呢個屋企就真係冇,同埋走咗都應該好難再修好,以後都會唔想再理佢。同時為咗一堆冇付出嘅蝗蟲要捱貴租,好唔順氣。

想請教各位,我有啲咩實際事可以做


\chapter{好後悔(長文)}

好後悔(長文)

\#1你牙媽個bra•5 年前

識咗一年半 同你日日傾計真係好開心

由一開始就已經對你有好感 但一路無講出口

一直都傾得好開心 對你嘅感情都一日比一日深

互動都開始多 會一齊追劇 睇片

但始終一路都係朋友

直到近期 開始變得冷淡

以前嘅message經常都會秒回

依加有時一日嘅message數量一隻手指都數得完

你變左

我都變左 變得愈來愈鍾意你

終於鼓起勇氣同你講心底話

然後你同我講 你曾經鍾意過我 曾經...

但依加已經...

It Hurt me so bad

我個心好似俾無數條橡筋綁住咁

咁大個人 從來未試過有呢種感覺

好辛苦

點解我會咁蠢 點解啲時間會錯得咁緊要

如果一早同你講心底話 個結果會點

一路以嚟 發生咩事或者有咩心事都係同你講

依加我好驚 好驚你會消失係我世界入面

個腦無時無刻都係你 做乜都係諗起你

做咩都無心機

我認我廢 男人不風流枉少年

所有道理我都明 但又點

感情就係咁

我真係好鍾意你

你可唔可以死灰復燃 俾次機會我

就算唔得 求你唔好消失係我嘅世界

I need you...


\chapter{媽媽揼咗妹妹啲遺物}

媽媽揼咗妹妹啲遺物

\#1平凡投訴師

•1 年前

妹妹十年前與世長辭,我非常難過

妹妹生前好鍾意一啲公仔(大約30隻),佢哋都係動物嚟,例如bear bear熊、大笨象、白象仔、狗仔等等

我都好鍾意佢哋,細個時會拎佢哋嚟教妹妹做功課同埋氹佢,亦成日同妹妹一齊講啲公仔嘅嘢

佢哋每一隻都有自己嘅名同埋性格

喺我同妹妹心目中,佢哋唔係公仔,而係動物

妹妹過身後,我就將佢哋放喺我床邊,望住佢哋時就會諗番起我同妹妹嘅開心回憶

但係媽媽就一直都唔多鍾意佢哋,話佢哋已經好舊喇,成日話要揼咗佢哋

我同過媽媽講過我好鍾意佢哋,但媽媽就唔太明白,淨係覺得佢哋係普通公仔,舊咗就要揼咗佢哋

因為媽媽成日都話要揼咗佢哋,又話幫我換床單時要將佢哋拎嚟拎去好麻煩 (但其實平時都係我自己換床單),我唯有暫時將佢哋收埋

我用白色透明膠袋將佢哋裝住,但當時屋企唔夠白色透明膠袋,我就落管理處問仲有無白色透明膠袋,但管理員話無,只有黑色,咁有十幾隻公仔我就會黑色膠袋裝住咗

當時我都喺度諗,媽媽會唔會唔小心揼咗黑色膠袋架,但又諗,佢拎啲膠袋出嚟時應該會打開嚟睇吓啩

然後就將啲公仔放入櫃裏面

前兩、三年,我諗起妹妹時,都好想拎番啲公仔出嚟

但一諗起媽媽又會喺度嘈生晒,大大聲聲咁話「都咁舊啦,揼咗佢哋啦」

我費事同媽媽嗌交,就無拎佢哋出嚟

上個月我拎啲公仔出嚟時,發覺有十幾隻唔見咗,搵咗好耐都無

媽媽就話可能佢幾個月前拎妹妹啲舊衫去揼時唔小心揼埋啲公仔(黑色膠袋裝住嗰啲)

之此我就一直好抑鬱,好嬲媽媽點解咁唔小心

媽媽同我道歉咗幾次,我原諒咗佢

但一諗起唔見咗啲公仔,就覺得好傷心、好對唔住妹妹

妹妹以前問過我,如果火燭嘅話會唔會帶埋啲公仔走,我嗰時就會

但竟然親手將佢哋放落膠袋,然後幾年後媽媽掉走佢哋

咁啱上年我又唔見咗手機,連啲公仔嘅相都無埋

我覺得好對唔住妹妹,哩排一直都好自責

近兩個星期放大假,我乜都做唔到,每日都好抑鬱、自責

由一瞓醒就覺得好難過,直至瞓覺

妹妹生前嘅片段亦一直浮現出嚟

我好自責

1. 如果當初去買白色透明膠袋返嚟,啲公仔就唔會有事

2. 如果當初將佢哋收喺床下底,媽媽就攞唔到佢哋出嚟 (因為妹妹生前問過我火燭嘅話會唔會帶啲公仔走,我驚火燭嘅話拎佢哋出嚟趕唔切,所以無將佢哋放床下底)

3. 如果前兩、三年我拎番啲公仔出嚟,佢哋就唔會有事

我一諗起以後可能再見唔番啲公仔,就覺得好難過

前幾日喺相簿搵到其中兩隻公仔嘅相,諗住搵公司做番佢哋出嚟

搵過幾間報價,佢哋都話最少要做500隻

錢唔係問題,最緊要係可以整番佢哋出嚟

咁到我老咗嘅時候,望住啲公仔,都可以諗起妹妹

但係,仲有唔少係無相嘅

所以,我想搵3D(或者2D)繪圖高手,幫我憑我嘅記憶畫番啲公仔出嚟,咁我就可以搵公司做番佢哋出嚟

酬勞大概\$2000,亦都可以留contact報價

希望我可以快啲整番啲公仔出嚟,咁我就可以重過新生活


\chapter{成日覺得廣東歌有種死唔斷氣既感覺}

\#1arkk暨tsm苦主

•2 年前

早兩日同同事傾開聽歌,講到對廣東歌(特別係近呢個世紀)有種死唔斷氣既感覺,好多假聲好多氣音,好慘好慘,我好慘我好慘,我天下間最慘,慘到想死,我要死啦,我死啦我死啦 .......呀!!!

無失戀既時候聽廣東歌會頹,失戀緊個陣聽廣東歌會頹上加頹。咁傾開呢個話題,我就研究廣東歌同歐美歌既差異,發現有2個原因導致有我好慘我好慘既感覺

第一個原因係旋律,廣東歌既旋律有9成都係單戀失戀慘慘慘情歌,而呢堆慘情歌全部都係同一個TONE既抒情歌,旋律慢,整體音樂大部份時間都處於高音既位置。而歐美既失戀歌,有抒情,有控訴,有快歌有慢歌,有高音有低音,會係唔同角度演繹失戀

舉個例,抒情既有 Passenger - Let Her Go

https://www.youtube.com/watch?v=RBumgq5yVrA

控訴既有 Lewis Capaldi - Someone You Loved

https://www.youtube.com/watch?v=zABLecsR5UE

輕快既有 Charlie Puth - We Don't Talk Anymore (feat. Selena Gomez)

https://www.youtube.com/watch?v=3AtDnEC4zak

第二個原因係歌手既表達,因為隻歌大部份時間高音,令到歌手長期要用假音氣音,而呢個高音既位置係大部份既歌手都唔係好處理到,每次都好似「鏈」住喉嚨唱咁,唱到好虛,唱到有D斷氣FEEL。特別係男歌手,幾乎無低音既部份,成首歌用晒假音唱。大佬,你男人黎架,比D雄性魅力黎聽下

以上係唔識音樂既細佬以聽眾既角度分析,想問下大家意見,順便介紹D歌比我聽


\chapter{我屋企係典型嘅一個窮人家庭}

我屋企係典型嘅一個窮人家庭

我屋企係典型嘅一個窮人家庭

父母都係從事服務業

兩個英文都唔識多隻

佢地從細對我灌輸嘅知識係「抵」

好似抵食同抵玩

啲嘢食一超過佢地心中嘅數

就會話唔抵食

轉買返啲抵食嘅

好似麵包舖啲pizza麵包/貴啲嘅包

佢地會話有咩特別

一啲都唔抵食

食餐包/雞尾包仲好過 又好味

然後佢地會臨收鋪先去買

因為平啲 攞嚟做我第二朝早餐啱晒

仲有一個例子係去酒樓飲茶時嘅

兩老平時好鍾意飲茶

所以我次次都比佢地拖去

大家都知蝦餃係酒樓算特點

價錢比較昂貴 但我細個又好鍾意食

但次次都唔比我叫

因為佢地話唔抵食

甚至唔健康都夠膽講

但轉頭又叫燒賣鳯爪呢啲更唔健康嘅

最心痛嘅有年親戚朋友由外國嚟香港玩

兩老請佢地飲茶 佢地個仔話想食蝦餃

我父母即刻叫咗兩籠

當時仲係小學嘅我係全部人面前講咗句

嘩 我平時想食都無得食嘅蝦餃啊

當晚返屋企比我老母用藤條打到雙腳紅晒

再講抵玩

同學問一唔一齊去迪士尼/海洋公園玩呢啲一定無份

因為係佢地眼中又係貴到仆街

去樓下公園玩仲好過

出國旅行又係 從來無帶過我出去

我大學畢業先第一次出國

中小學無零用錢過

得食飯同搭車

仲要係比到啱啱好嗰隻

多啲都無

一有咩想買就會扯到其他野

你成績咁差都好意思想買嘢?

但屋企一個人都幫唔到我

問英文唔識 問數學唔識

補習又嫌貴 唔比錢去上

另一邊就屌點解得6,70分

如果我有書讀一定叻過你

自己文化水平低但又希望望子成龍

咁嘅學習環境點會成到材?

過年利是錢比完我之後 又話幫我保管

話就話我嘅 實情一啲都唔用得

有年我有個模型好想買 就拎咗利是錢

比老母知道後又係打到我仆街

我駁咗句我嘅利是錢點解唔用得

結果打得仲西利

打下打下發覺都打咗好多

之後得閒再打啦

多謝睇到呢度嘅大家


\chapter{我老豆係個大仆街。。。}

\#1你屎窟三個窿•6 日前

話說我老豆3年前已經俾我老母發現咗壞事做盡,叫雞,揼邪骨,包養洗頭妹,帶朋友唱淫K,搞公司伙計個老婆,什至係搞大屋企賓妹個肚。

因為我老豆有一次無lock手機,我老母好奇睇吓先發現所有嘢。當時已經嗌曬大交,嘈到話要離婚嫁啦,我老豆俾老母踢爆所有嘢都唔認衰,好似完全未驚過。後尾係透過WhatsApp應承我老母以後會生性、唔再做壞事,我老母都心軟咗,又為咗一個家庭好似完整啲,唔想我哋做仔女嘅要係個破碎家庭中長大,雖然我同我兩個家姐同細妹都已經成年。又唔想聽三姑六婆們嘅閑言閑語,所以我老母忍氣吞聲,忍辱負重咁繼續對住我老豆呢個大仆街渣男。。。

到上個禮拜,我老豆又唔記得lock Screen, 又俾我老母發現到呢幾年都係死性不改,繼續翻大陸叫雞,包養個細過自己26歲嘅大陸妹,仲偷偷地帶上嚟屋企扑嘢。到今日我老豆突然問我手機點樣改密碼,我就知佢又俾我老母發現到佢啲衰嘢,因為三年前佢都係咁樣問我點改密碼,但其實係因為佢無Lock Screen就咁放低部電話所以先會東窗事發。再加上,我有一次見到佢手機個noti有個匯款紀錄係匯錢過去菲律賓俾以前個賓妹,一路俾錢養住我半個細佬。我就知道佢一路無變。

衣家我老母好唔開心,覺得呢個渣男死性不改,60幾歲都仲咁咸濕,玩到咁猖狂,佢問我意見,要唔要同佢離婚,我一定係撐我老母嫁啦,但我應該做啲咩?我可以點做?有冇人可以教下我點面對?香港法例同大陸法例對呢種情況而離婚一般係會點判㗎?


\chapter{我覺得好對唔住細佬(長文}

我覺得好對唔住細佬(長文

老豆出軌離咗婚6年,依家同老母細佬3個人住公屋

老豆基本上每個禮拜都會同我哋食一次飯

細佬依家14歲,越大覺得佢越孤癖

直到早2日老豆同細佬大吵,打架。

我覺得好對細佬唔住,

嗰時老豆執嘢走咗,我只係顧住自己情緒

完全唔知細佬人仔細細都有好多情緒

直到媽咪同我講,

佢嗰晚成晚冇瞓過,初頭嗰幾個月

一聽到門口有聲就以為老豆返嚟。

我聽完好心痛

嗰時我好嬲老豆,我唔比老豆帶細佬同個小三同小三個仔一齊去海洋公園玩,我唔想細佬佢哋變咗一家人。

依家嘅我放低咗,但原來細佬由唔捨得老豆變做憎佢。中間發生好多事,我一直都冇上心。

依家我唔知點好,唔知點樣可以修復到佢哋之間嘅關係。我對於老豆嘅感情好複雜,我冇得單純用恨嚟面對佢。


\chapter{掛住你。(長文)}

掛住你。(長文)

\#1條野太詳•6 年前

睇完調位po有感而發 純粹叫下春

2年喇 我仍然好掛住你 好多無關痛癢既小事 我第一時間都係諗起你 係到諗 如果可以同你一齊做 究竟會有幾好 會有幾開心呢

當年一齊坐 大家由互相唔認識去到每堂糖痴豆咁傾計 唱歌 講心事 過完個晏就 大家買野食交換 開心share 所有野真係好幸福架

你訓覺起身 個臉豬dum 紅卜卜個樣 你測驗唔合格 扁哂嘴仔 眼紅紅既傷心樣 同埋你睇書睇到喊既樣 到呢家我仲歷歷在目 真係好得意

你個身體仲會散發出一種好特別既香味 唔知幾時開始 我呢個傻仔就比你深深吸引住喇

哈 我仲記得以前你問過我一個好傻既問題 「你究竟鐘意邊個女仔!」其實我一早就講左啦 果個咪係你囉 你又唔信

開心既事唔會長 個天總係會搞破壞

記得班主任當時公佈調位名單 我地將會調去一東一西 睇完結果 我地大家成堂都冇出過一粒聲 可能一樣係到諗 終於都要完喇 十分之唔捨得。

呢個係我人生沉默得最長既時間 我好想時間永遠停止係呢刻。 然後,你用就黎喊既聲線同我講左句「哈哈 係咪好掛住我呢 唔好呀你」我笑住咁答「勁掛住你呀!」  我地就咁破冰 開心咁上埋最後一齊坐既一堂

可能有人會問 明明大家有feel 點解最後你地冇一齊? 我真係唔知 當時我冇咁既勇氣 下一次講野既時候感覺已經唔同以前 最後亦不了了之。但係 即使咁樣 呢件事可以話係係我中學生涯入面最開心

此情可待成追憶,只是當時已惘然。的確,雖然已經返唔到舊時,但係呢段回憶我會一直好好保存落去。 就算你知 都係唔好相認喇


\chapter{放負post 放負入}

第 1 頁

1 頁

\#1西環夏薩特•5 年前

今日得閒係屋企坐冗思下

回憶起由一開始全部都認為有希望但到左有義士輕生警察開槍黑社會斬人再到近期嘅集會我發現我地失去左好多

而家有一大班藍出來發表我哋先係亂港元兇我真係覺得好痛苦到底呢班被洗腦嘅人有無良心同批判思想

我認為可能本身有不過被人已經被人洗左腦

而此刻我想香港可能就會成為下一個目標

唔單止佢哋仲有一班自私港豬認為我哋阻到佢做生意我唔敢相信有人會咁蠢只係留意自己生活賺錢只追求物質唔關心其他野 就連小小的犧牲都唔付出任由自由社會墮落同香港青少年嘅未來受破壞

去到721 820黑社會斬人我真係覺得好可怕好恐怖本來我地盡量和平用道理去溝通就算有和警察有激烈沖突但都唔會有無辜巿民受傷,現家?就連普通人都會被人斬仲要斬人條友到依家仲未落案

我唔係怕黑社會,我只怕香港連最基本言論自由都失去

我都有諗過既然係咁我點解仲要去表達祈求不如乜都唔好理或者移民算9數

但我知我不可以因為香港是我的家鄉我唔想被洗腦被逼放棄我有嘅自由民主

所以只要一日仲有希望我都不可以放棄唔係會對義士唔住

sorry for1999 我希望大家睇完可以撐落去放完負就出返因為我唔想再有人自殺


\chapter{欺凌+控制型父母(長文慎入)}

欺凌+控制型父母(長文慎入)

樓主十年十八,女,讀緊大學。一直以來受到情緒問題困擾,最近臨近考試好想死。我成日懷疑自己嘅唔開心係咪正常,定係係度無病呻吟。希望有巴絲可以幫我分析下。

先講下個人經歷啦。從小學到初中,我一直受到唔同程度嘅欺凌。我承認自己以前的確幾乞人憎,所以我並唔怪d欺凌我嘅人。我覺得佢地嘅行爲其實情有可原。我一直被話蠢,被人叫過豬扒、雞、豬、垃圾婆等等。細個因爲好曳所以成日比老師閙,連老師都會欺凌我,特別對我衰d。成日被各種形式嘅公開羞辱、孤立、語言暴力等等。都試過被人用鐵凳打,其他肢體上嘅傷害已經唔太記得,總之成個小學階段都係不斷被各種欺凌。中學嘅時候,性格依然比較古怪,同埋人品亦唔好,所以繼續被孤立,但冇好嚴重嘅欺凌。有人幫我整假fb account扮係我散播假消息,都有人將我d柒相/片擺上class group或者YouTube。到高中我慢慢改變,所以朋友漸漸多左,同以前傷害過我嘅人都做返朋友。雖然而家係大學都有一班\lr{亻}{},但係我依然會覺得自己好令人討厭,成日覺得我d朋友其實暗地裏唔中意我。我唔知點樣先可以擺脫呢種陰影,有番自信甘同人相處。

另一方面,屋企人嘅控制令我覺得好困擾。我媽一心一意想我嫁比有錢人,熱衷于改變我個樣去所謂取悅其他人。佢會管我笑嘅角度、身材、儀態、動靜,甚至最近佢覺得我笑得好難聼,想我改下自己嘅笑聲。凡此種種,我實在不堪其擾。每次我覺得好煩嘅時候,佢又會話自己好愛我,只係想我更加完美。佢日日ff我同有錢佬拍拖,但我明明有男朋友,雖然唔算有錢但好愛我。佢同我阿爸會成日叫我分手或者出軌,試下其他男仔。我阿爸甚至話,佢覺得我同呢個男仔一齊係簁左自己嘅學歷。總之我屋企人就不停羞辱我男朋友,有一次仲閙我閙左幾個鐘,我不停甘喊,但佢地根本唔會心軟。佢地好中意錢,想我返大陸發展。我明明考到英國一間全英top 10嘅大學,佢地都唔比我走,要我留係香港。

一直以來,我父母比我嘅關懷好少,净係會叫我讀書。我而家讀嘅科都係佢地幫我揀嘅。我媽會輕微語言暴力我,一唔順佢意就鬧我鬧得好犀利。我其實唔算係一個冇主見嘅人,但次次都會被佢逼到無路可退。佢真係一個冇同情心同無法説服嘅人,佢只會不停發癲甘鬧我。同埋佢有潔癖,對細節好執著,好小事就會被無限放大同挑剔,小小錯就會被人鬧。

我覺得自己嘅心理無可厚非受到左一定程度嘅影響。我個腦會好亂,唔知自己係邊個,唔知自己應該點左。有時候會想死,因爲覺得呢個世界嘅人都好唔中意我。我媽都親口講過好多次,佢支持我去死。我已經有無數次下定決心去死,但係次次都被我男朋友阻止。我唔知點算,有時好想分手,之後我就可以去死。我唔知自己算唔算抑鬱,我亦都唔知佢地算唔算傷害左我。其實佢地都冇錯,我父母都有好多優點,佢地係物質上會盡量滿足我,同埋花好多時間陪伴我。至於欺凌我嘅人,其實我都唔覺得佢地係衰人。有時候我覺得自己真係好應該去死,明明自己又廢又唔努力又乞人憎,點解仲要活著。有時候覺得我就應該聼阿媽話,唔好幻想自己會有咩事業,嫁個有錢人算啦。加上我大學成績好差,令我更加覺得自己好廢。我好唔開心,但父母覺得我唔開心只系因爲想逃避讀書同考試。我媽覺得我根本冇抑鬱症,只係懶同埋綠茶婊。

好想有人點醒我,比下建議比我。其實我係咪真係應該去死?或者向父母投降?我覺得自己係一個冇價值嘅人。唔知以上文字會唔會令人覺得我好煩好討厭。如果係嘅話,我想同你真誠地講聲對唔住。生而爲人,我真係好抱歉,令甘多人唔中意我。其實我真係唔介意去死,但我唔想傷害到愛我嘅男朋友。其實我係咪真係又懶又蠢又討厭?我係咪應該去死算?或許我死左,我男朋友好快會清醒返,發現我根本係個Bitch,慶幸我死左?

我打呢段文字嘅時候,都有懷疑自己其實係咪無病呻吟緊。其實我都唔算好痛苦,我生命中都有唔少覺得幸福,笑得好開心嘅時候。我成日好唔開心,但又會問自己何必如此。或許我都唔係真係想死?其實我好迷惘,個腦好亂。我好似分左三部分甘。第一部分係我自己本身,會有情緒,會渴望被愛,會有夢想。第二部分係我嘅superego,覺得自己好廢,好應該去死。當我想死嘅時候,又會同自己講何必如此,不過係無病呻吟。我會同自己講,我受嘅傷害其實一d都唔嚴重,我父母好愛我,我不過係曲解佢地對我嘅愛。第三部分嘅我被稱爲doll character,佢會説服我聼父母嘅意見,因爲我好蠢好冇能力,根本冇資格成爲一個獨立嘅個體。

打打下呢段文字,我又覺得自己其實根本冇事,我腦海中冇三個人,我根本冇受到傷害,只係某種孟喬森症候群。

我真係唔知邊個先係真嘅我啊。


\chapter{發唔發覺,喺香港冇嘢慘得過做毒撚?(長文,為毒撚發聲)}

見今日公司冇咩野做,作為一個a0毒撚,係時候出嚟講一下呢個問題

首先以下講嘅野唔適用於有能力但因為唔想社交而主動做毒撚,覺得毒撚好enjoy,甚至根本就係偽毒嘅人

我只係想講,喺香港,幾乎冇嘢慘得過做一個毒撚

先唔好講毒撚冇朋友,一世都唔會溝到女,冇人陪,注定孤獨終老呢d 廢話

毒撚,基本上喺邊到,由其香港,都係俾人笑,俾人嫌棄,人見人憎,人見人踩的類型

其他人有咩痛苦的事,身邊都起碼會有人陪,有人安慰,有心靈支柱

但毒撚,真係乜都冇,有痛苦,想搵個人呻下都冇,同埋幾乎做乜都只係會被人屌

每當毒撚做錯少少嘢,明明成件事同「毒」呢樣嘢根本就冇關係,但都會被無啦啦俾人話:你咁樣,唔怪得之你係毒撚/冇朋友/溝唔到女啦

就連本應最親的屋企人都會屌你,屌你點解30歲都仲係毒撚,搵唔到女朋友,係唔係你個人有咩問題?

真係唔知由幾時開始,毒撚會由一個表達人身份狀態的偏負面的形容詞,遂漸變成一個分辯對方係咪一個正常人的標籤

喺香港,由其連登,最常聽到人講:正正常常嘅人點會冇朋友/正正常常嘅人點會溝唔到女之類。

真係諗極都唔明,點解毒撚會同係咪一個正常人掛勾?

咁呢個世界,梗係有d人喺某方面叻d,某方面冇咁叻。咁毒撚純粹係喺社交方面冇咁叻,點解就一定就係屬於有問題嘅人?即係人一定就要識social?唔識就喺弱智有問題有精神病?

明明毒撚冇人陪,自己要永遠孤獨一個,本身成件事已經夠撚哂慘

但就係要日日夜夜繼續承受俾人罵,被人笑,被人標籤,被人踩的痛苦

平時成日都會見有人話唔好縱容網上暴力,唔好網上bully人,但偏偏個個見到毒撚就走去踩人一腳

平常d人有咩唔開心出post呻,大家都會就事件分析給予幫助同安慰,但偏偏毒撚出post就個個都笑撚番佢轉頭,叫毒撚照下塊鏡,睇下本銀行存摺,笑毒撚話鬼叫你係毒撚咩

喺香港,仲有咩係慘得過毒撚?

我可以話,出面感情台幾乎90%嘅感情問題都唔夠一個毒撚承受嘅嘢慘

你被人帽又好,被女被仔飛都好,你起碼曾經擁有過,起碼有過一段幸福的時光,你先會痛

但毒撚,由其a0,真係連呢d少少嘅幸福都未曾擁有過

同埋成日見有d人講,話你可以改,可以脫毒,唔再做毒撚

但講真,有d野,真係唔係你話改就可以輕易改到,就算有d野對你黎講好易,但對毒撚黎講可能真係可以係好難同做唔到,或者咁講,如果個個咁輕易就脫到毒,呢個世界就唔使有毒撚啦

好似個個都想做有錢人,唔通就個個都有能力做到?

但喺咪就係因為做唔到,因為喺毒撚,咁樣就值得俾你哋笑?值得俾你哋標籤?

Anyway,講咁多嘢都係希望大家唔好對毒撚咁hard,唔好一有嘢就笑鳩人毒撚同屌鳩毒撚,更加唔好再因為毒就標籤一個毒撚

毒撚,本身已經夠哂慘,甚至屬於全港最慘個批,唔需要你哋再加多腳黎踩底

最後,希望大家可以對毒撚好d,共建和諧氣氛


\chapter{移民英國一周年}

移民英國一周年,之前一篇分享對於收支平衡既擔心,今次講下心態上既調整。

第一,天氣方面,我開頭以為自己唔怕凍應該會好易適應,但黎左一年,發覺始終未算,見人地十度淨係著短袖背心,我真係唔得。識我既人都知我係中意夏天,勁討厭冬天既人,加上係中意上山下海,陽光與海灘既人,而英國有半年時間都算係冬天,絶對唔易過。我特登揀wales 主要中意佢有山行有海灘,可惜世事就唔係想像咁美好,9月天氣就已經得10度,去咩行山海灘呢?冬天真係好漫長,尤其十一月開始較慢一個鐘,4點就天黑都算,最慘係冬天會基本上日日落雨,隔個星期就打風,成日橫風橫雨,莫講話去玩,出街都未必想。之前9月頭滯留係西班牙旅行,感覺好舒服,日日好好太陽,一返到英國就變左另一個世界,我終於明白點解咁多英國人中意去西班牙。

第二,食物方面,真係好少選擇。開頭黎,見到超市好大好興奮,但其實食物選擇唔多,肉類的確平過香港,質素唔錯,但海鮮、蔬菜、生果好少選擇,超市雖然係好大,啱買既野唔多。出街食仲攞命,Cardiff 依度既餐廳仲少選擇。其實我已經算中意食西餐,開始黎到覺得好新鮮,都想試吓唔同嘅餐廳。可惜食黎食去,都係fish and fries , burger 之類,頂籠rib、steak 好悶,連有paster嘅餐廳都唔多,有時都會食下泰國野日本野,但依度真係得好少選擇。

第三就係要接受依度既人處事態度。有時候打去customer service hotline ,等好耐不特止,有時仲傾得唔like ,會cut 你線,甚至鬧翻你轉頭,香港係無可能發生。另外送貨,遲到無通知好正常,完全無到都好正常,啲人做嘢真係好無交帶,唯有再追多幾次。約GP 睇醫生,難過登天,成日無得比你睇,有時要將個病情講得嚴重啲,或者唯有自己買藥食算啦。住得耐慢慢就要接受,呢度啲人嘅處事手法,有時係會好得人驚,完全唔可以用香港果套,感覺氣餒係好正常。

黎到一年,心理調整好重要。我地要接受好多野唔係想像中咁美好,我絕對唔會報喜不報憂。開頭黎到當然覺得好新鮮,啲屋好靚,啲公園好大,但之後就會覺得無乜特別,個個英國城市都差唔多。之前會成日揸車去附近城市玩,依家都少左。新鮮感無左,就要好好調整心理,用唔同方法去適應生活遇到既落差問題。另外多啲搵朋友傾訴下都好,睇下有咩大家都遇到既問題同解決方法。

最後,你問我會唔會後悔過黎,我會話一定唔會。因為英國有既最重要係自由既空氣,小朋友讀書最開心既,唔需要好似香港咁大壓力,當然無所謂紅色思想洗腦已經好好。而鄰里關係亦係香港比唔到,大家會互相幫助,互相分享美食,係開心既。而依度既居住環境當然比香港好好多,香港啲屋豆腐潤咁細,真係好侷促,周圍高樓大廈,依度就另一個世界。點都好,大家努力啦,要係度生活絕對唔容易。


\chapter{警世長文 咪再同我鼓吹 我哋係最後一代 呢啲厭世廢話}

預咗比你地班戇鳩負皮上一頁,總之我豬自稱講埋今次,唔啱聽柒就

我由深入淺講

我同啲唔生嘅朋友都係咁傾,佢哋係2019之前已經話唔生,佢問我點解要生,生咗無得享受生活,要照顧小朋友,個小朋友有得享受,自己無得享受,同埋自己根本唔鐘意呢個地方,自己都覺得係捱完十幾廿年,就快三十歲先叫可以叫做識嘆下,點解要生個仔出黎受苦。

我無講太多,我淨係問佢,過去嗰三十年係咪淨係有唔開心?我段估佢,甚至大家第一次學識行路第一次畫畫油顏色第一次游水第一次飲可樂⋯都係開心卦?咁點樣覺得過去嗰三十年係淨係捱淨係唔開心?其實佢甚至大家,覺得呢個世界唔好,覺得佢好好差,唔係自己心目中咁,係因為 不如意事 十常八九 。大家連登仔都好憎大陸文化,偏偏呢樣又咁入心入肺。我無講任何政治野,淨係問佢細個係咪淨係得唔開心。我相信大部分嘅答案都係否定。我唔排除有人,好似印度嗰啲一百萬零一夜嗰啲,但係喺香港成長嘅童年,會有幾唔開心呀?我當你真係有個好差嘅童年,你係咪都相信生命嘅本質係咁悲劇先,你係咪都唔相信生命係五十五十先?你唔相信new born baby會開心?你話捱,好辛苦,係因為你唔相信生命會開心,我舉例,你唔鐘意港共,唔覺意生咗,你好唔鍾意個仔好鍾意港共,佢返大灣區搵好多錢做到好多佢想做嘅野好開心,你係咪都會好後悔生仔?如果係,你只係唔鍾意政權唔鐘意個在位者,唔係唔鍾意生命。你因為自己活喺唔如意嘅生活而放棄生命,係為咗局部而放棄全部。生活係局部、生命係全部。你個仔會有無限可能性,你偏偏悲觀到只係相信佢得一種可能。

咁ban完之後我嘅suggestion係咩。當然你唔鍾意個當權者無問題,非常好添,因為我都唔鍾意。唔鐘意呢個咪揀第二個囉 - 移民。你去揀過第二個管治你嘅人係咪比起你去放棄全部為好呀?除非你話自己係打從心底𥚃直頭討厭小朋友,聽到扭計會暈,咁另計。如果唔係,你好撚型咁對住鏡頭講自己係最後一代,係代表咩?你想同你唔鍾意嘅野玉石俱焚咁浪漫?Batman and joker?你唔鍾意佢,仲點會想同佢有個咁浪漫嘅connection 吖,咪玩啦。

我成個對話一句都無講過政治,只係要大家問自己,鐘意嘅係乜,唔鐘意嘅係乜。

所以,我嘅suggestion係,你唔鍾意嘅野,你咪避開囉。放棄生命係最後手段,我哋喺香港成長、經歷2019嘅一代人,真係未至於要行呢一步。我假設你係唔鍾意個政權,你咪移民囉,之後生仔囉。你憎佢,你係應該 我憎你所以我活得比你差,從而令你難受,哼。 咁樣?我唔同意,我認為係應該 你憎我,我都憎你,fine,我唔留喺呢到啦,我走,ok?但你睇下我,我一家大細好撚開心呀屌你老母。

最後,我講下 我哋係最後一代有咩唔好。就係如果你真係想做最後一代,你係同時地討厭緊生命,如果係嘅點解要用唔生呢啲咁被動嘅手段?你不如主動啲去了結自己嘅生命啦。自己嘅出生又被動,結束生命都咁被動,成世人根本就毫無立場,乜都係煮到埋黎就食,然後仲大大聲同人講喺到積極對抗緊。你睇下對家,吹我唔漲啦掛,哈哈。

老實,香港人咁醒,去邊到會搞唔掂呀?你話熱門地方搞唔掂而你又咁憎呢到,中歐西歐做難民都得卦,德國匈牙利奧地利⋯你睇下德國妹幾撚正。

你溝唔到可能你個仔溝到呢可能你個孫溝到呢?屌你為咗香港個局面搞到自己無曬可能性有乜為?無人會覺得難受,除咗你老母。你哋最鐘意個子華神都咁講啦 你抗議出黎絕食,好激進,其實係好有用嫁,如果你要對抗嘅人係你阿媽。

Nolan嘅interstallar 咪又係,人類命運係咪淨係得種田or等死?最後積極去解決重力問題,柳暗花明又一村。

屌你哋又當子華路蘭神咁拜,佢哋啲野又唔記。

簡單講,我唔同意 我哋係最後一代 呢個七傷拳;解決方法係,當你已經nothing to lose嘅時候應該積極移民,活得比對家好,生命有無限可能,咪比眼前嘅困境困到自己得返二元對立。

你有無諗過自己個仔可以代表加拿大田徑隊玩奧運

你有無諗過自己個仔可以喺東歐種田釀紅酒

你有無諗過自己個仔可以去袐魯做考古研究

你有無諗過自己個仔可以去瑞士從軍

你有無諗過自己個仔可以去nasa做太空人訓練

你有無諗過自己個仔可以去印度做工程師

無,因為你忙住消極死亡抗共。

我講嘅所有野都無情感勒索道德勒索過一句,只係問大家鐘意啲乜唔鐘意啲乜,幾時開心幾時唔開心,同埋想大家認清返自己嘅反抗目標,從而去做合適嘅反應。


\chapter{長文慎入 有個發錢寒嘅老母真係好撚慘}

交代下背景先

我屋企住公屋

老豆地盤佬一個 無乜學歷

老母結婚之後一路無做野,無學識無技能嘅師奶

由細到大都同佢地關係唔係特別親

相對上同老豆關係會好少少

佢地對我嘅了解

都係去到知我讀邊間中學邊間大學

至於讀咩科 我都無同佢地講過

由細到大都無咩資源俾過我

完全係放養式咁養大我

零用錢呢啲奢侈品當然係無啦

記得小學有一次,我見同學學琴

我同我老母講我都想學

佢一句:學咩呀 有咩好學

就完咗呢段對話

甚麼興趣班補習去exchange去旅行咩都無份

其實仲有好多好多呢啲野 唔喺度詳講了

金錢上嘅資源無俾,連精神上嘅支持都無俾過

只係識得拎我同人比,人地個女點點點

小學中學默書考試考到95分

就問點解無100分

考到第二名

就會問點解唔係第一名

喺佢眼中永遠唔會有滿足呢兩隻字

有少少成績/成就 美其名為我自豪

實際上就拎我出去同人打飛機

滿足自己嘅虛榮心

自己另一個仔啲野就收埋唔講

大學畢業後出黎做野

佢曾經有問過我攞家用

我同佢講 要我交家用我就搬出去

我一直以黎都好想搬出去住

我從來無喺佢地身上感受到正常人嘅關懷同愛

本身平時喺屋企大家都無兩句

反而咁樣有返啲距離感,似乎係對大家最好

最近佢間唔中會問我有無拍拖

我平時對感情事一概不提

因為唔想俾佢地批評

喺佢地眼中我係一個A0

而佢最近嘅態度都好似幾想我拍拖,幾正面咁

咁我就一時大意鬆懈咗同佢講咗我拍緊拖

佢第一樣關心嘅就係 對方有無樓

講埋啲咩千祈唔好幫人供樓,最好就層樓都寫埋我個名

我男朋友托賴有層私樓供緊

去到呢度佢嘅態度都幾正面嘅(因為人地有樓)

佢就係咁話要約埋我男朋友出黎食個飯見個面

我男朋友都無咩所謂嘅

終於去到嗰一日

成程氣氛尷尬到 我都唔想再回想起

審犯咁 問人返咩工、之前返過咩工、層樓月供幾錢 etc.

然後又黑口黑面咁

最後知道佢嫌人無錢 因為層樓唔係full pay

我返屋企食飯又俾說話我聽

講埋啲唔知乜水個仔喺邊度有層full pay嘅樓

問我做咩唔揀(我連相都無睇過)

佢已經講到人地非我不娶咁,我係聽到笑咗出聲

又講埋人地係獨生子

全程都只係識講人地有幾多層樓有幾多錢

無提過人地性格學歷興趣(因為佢都唔知)

佢唔在乎個人適唔適合我,我會唔會鍾意

只在乎佢個「未來女婿」有幾多層樓,佢可以拎出去同人打飛機

最好笑係佢叫我唔好再錯落去 要止蝕離場 (講緊同我男朋友一齊)

我覺得好委屈我男朋友

因為我同佢屋企人食飯個陣 大家氣氛都好好

會傾下啲日常生活野、講下未來

亦無人會問我人工幾多 屋企住公屋定私樓

佢阿媽仲成日話要買機票book酒店請我地去旅行

因為呢件事我都同咗兩老嘈交

我同佢地講要求人地有樓有車之前都要睇下自己屋企咩條件

而家係我地屋企條件差過人

但佢地偏偏睇唔到呢一點

唔講仲以為佢地係愛新覺羅汪啲親戚

佢地嘅取態係佢地無錯 佢地想「我」好

仲強烈否認自己淨係講錢

下下都提住人地啲樓,唔係講緊錢唔通講心?

佢仲癲到私底下走去搵我朋友講

我男朋友無錢 層樓要供

又幻想我幫人供樓

順便數臭我無良心無禮貌唔交家用之類啦

總之就受害者上晒身

仲話第時我地結婚係唔會得佢兩老嘅祝福

其實我想講

佢嘅唔出現,就係對我最大嘅祝福

終於呻完 多謝大家睇到呢度


\chapter{(長文慎入)好撚大壓力就頂唔順 想jump}

(長文慎入)好撚大壓力就頂唔順 想jump

\#1好大壓力就爆煲•7 個月前

首先,我認衰,完全明白自己犯嘅錯應該自己承擔,但我真喺覺得好大壓力,好想講出嚟抒發下,因為我連身邊嘅人都唔敢同佢地講,驚自己再屈屈埋埋會jump。

事源喺2年前,個時啱啱失業,本身已經坐食山崩,仲要唔懂事染上賭癮,越賭越大,借錢賭個隻,乜野一線二線、信用卡透支咩都借過曬,最高峰差緊啲數加加埋埋差唔多一球。而家唸返轉頭真喺抵自己死。

日日喺屋企撻喺到同埋去賭錢嘅日子大概維持左半年,借借下真喺借到盡又開始冇錢還之後發覺唔可以再喺咁,個人醒左,搵返野做,去做紮鐵,就喺唸住搵多啲,可以慢慢找返曬啲數。

返呢份工真喺好辛苦,做左呢行咁耐到今時今日都依然有時會攰到返屋企就好似斷電咁訓著,日日辛辛苦苦返工搵埋搵埋就喺拎去還,還完都已經冇咩剩,我明呢個情況喺我自己攞嚟嘅,我亦都想將來可以有返正常人嘅生活,唔洗日日為還錢煩惱,所以我好比心機做。由差緊接近一球數,還左年半到而家還左大概40萬,仲差60萬左右。

本身覺得keep 住落去,應該捱多2年就可以重見天日,不過2個月前突然開始少左工開,本身都擔心會唔會搵唔夠數,找唔到數,然後原來當人黑起上嚟真喺會有好多事接二連三埋身,突然電話收到好多我爸比人追數嘅WhatsApp,屋企喺咁比人上門追數,淋紅油,個段時間我媽日日喺屋企喊,從我媽口中得知原來佢幾個月前已經知呢件事,仲走去問財仔借左10幾萬比我爸,當時我知道左咁嘅情況,我完全唔知點應對,我唔敢比我媽知我都差人好多錢,我驚佢真喺一時間接受唔到會唸唔開,同埋我2年嚟一路都仲搵到夠錢找到數,唸住自己搞得掂冇話過比屋企人知,冇唸過冇啦啦要面對咁嘅情況,當然我都仲喺冇同我屋企人講,但日日比人上門咁樣追真喺唔喺辦法,我就傻更更自己走去再借,去還埋我爸條數。

借多左加上唔夠工開,慢慢開始入不敷支,搵埋唔夠還不突只,仲要數冚數,日日就為錢煩惱,食唔安訓唔落。我屋企2老、女朋友、朋友、同事都唔知我嘅情況,冇同過任何人講。壓力越嚟越大,早排真喺有一刻唸住不如試下去做犯法野搵快錢搞掂佢,但喺一唸到如果唔好彩比人拉左,我女朋友點算,我媽點算,最後都喺唔夠膽。而家日日就唸住啲錢,我覺得自己就爆煲,有萌生過jump 嘅念頭,我驚自己頂唔順,所以上嚟發下牢騷,希望抒發下。 同埋想睇下有冇巴絲可以比到意見我可以點,除左破產我可以做啲咩?


\chapter{(長文)有冇人完全唔知自己人生做緊咩?}

(長文)有冇人完全唔知自己人生做緊咩?

覺得自己人生完全冇目標 冇嘢想去做 靜係想平平穩穩咁過咗佢 見到身邊啲人好似好有目標咁有時會好羨慕 都好想自己有啲事令自己有衝勁去做 但真係搵唔到 無論咩方面都俾唔到動力自己

愛情方面 雖然有個拍咗兩年拖嘅女朋友 但真係愈嚟愈唔想拍拖 覺得好攰 佢發脾氣要氹 唔係唔鍾意佢 但真係覺得要氹好麻煩好大壓力 明明我自己心情都唔好點解仲要做小丑去氹人 有好幾次想分手但最後都因為好愛大家同唔捨得而冇分

友情就得好少真係熟嘅朋友 喺大學識嘅都係學業上有需要先搵大家/hi bye friend 中學個班都係幾個月先約一次 而自己又好少會冇啦啦同人講心事 所以約出嚟都係講吓近況 寒暄幾句就散水

其他人表面睇我好似成日好樂觀 好講得笑咁 但其實我內心好多嘢好抑壓 會成日冇啦啦好頹 對自己將來想做咩完全冇憧憬 學業成績一路差落去但完全冇動力去追 靜係想快啲讀完 但讀完想做咩都係冇方向 諗到可能之後會屈喺office成日 收工返屋企食個飯就訓 訓醒第二日又要返工就更加頹

有時真係覺得好辛苦 有冇人明


\chapter{[長文慎入] 練大隻!全部同我練大隻!}

[長文慎入] 練大隻!全部同我練大隻!

[長文慎入] 練大隻!全部同我練大隻!

回帶

追蹤

發送匣

名已留

吹水台自選台創意台熱 門講故台最 新學術台

新聞

時事台World政事台財經台娛樂台房屋台

科技

手機台電器台Apps台硬件台電訊台軟件台

生活

創意台感情台健康台家庭台飲食台上班台旅遊台校園台活動台

興趣

體育台學術台講故台遊戲台影視台動漫台攝影台音樂台汽車台寵物台玩具台潮流台直播台美容台成人台

其他

站務台黑 洞排序

最相關

主題新至舊

回覆新至舊

全部

全部吹水台手機台未分類突發新聞社評政事台World體育台娛樂台動漫台Apps台遊戲台影視台講故台健康台感情台家庭台潮流台美容台上班台財經台房屋台飲食台旅遊台未分類神秘校園台汽車台未分類本地華語歐美日本樂器韓國未分類改歌藝術設計硬件台未分類家電音響攝影台玩具台寵物台未分類程式開發AI活動台電訊台直播台站務台成人台

主題

HumanGetCrazy

3 年前

3714

41 頁

選擇頁數1 頁2 頁3 頁4 頁5 頁6 頁7 頁8 頁9 頁10 頁11 頁12 頁13 頁14 頁15 頁16 頁17 頁18 頁19 頁20 頁21 頁22 頁23 頁24 頁25 頁26 頁27 頁28 頁29 頁30 頁31 頁32 頁33 頁34 頁35 頁36 頁37 頁38 頁39 頁40 頁41 頁

youtuber 無句.真 (長文慎入)

吹水台

士多啤梨蒸魚

1 年前

-336

26 頁

選擇頁數1 頁2 頁3 頁4 頁5 頁6 頁7 頁8 頁9 頁10 頁11 頁12 頁13 頁14 頁15 頁16 頁17 頁18 頁19 頁20 頁21 頁22 頁23 頁24 頁25 頁26 頁

[長文]另一半好忙好忙

感情台

牙東囝

2 年前

271

26 頁

選擇頁數1 頁2 頁3 頁4 頁5 頁6 頁7 頁8 頁9 頁10 頁11 頁12 頁13 頁14 頁15 頁16 頁17 頁18 頁19 頁20 頁21 頁22 頁23 頁24 頁25 頁26 頁

俄羅斯的眾沙皇(長文)

學術台

職業雜工隊

6 年前

177

26 頁

選擇頁數1 頁2 頁3 頁4 頁5 頁6 頁7 頁8 頁9 頁10 頁11 頁12 頁13 頁14 頁15 頁16 頁17 頁18 頁19 頁20 頁21 頁22 頁23 頁24 頁25 頁26 頁

(長文) 24歲仔失業日記

上班台

我係打手

5 年前

47

6 頁

選擇頁數1 頁2 頁3 頁4 頁5 頁6 頁

何以戰? (長文)

時事台

東亜の子

2 年前

253

25 頁

選擇頁數1 頁2 頁3 頁4 頁5 頁6 頁7 頁8 頁9 頁10 頁11 頁12 頁13 頁14 頁15 頁16 頁17 頁18 頁19 頁20 頁21 頁22 頁23 頁24 頁25 頁

奧匈帝國的生前死後(長文)

學術台

大白兔奶糖\_瑤

3 年前

606

27 頁

選擇頁數1 頁2 頁3 頁4 頁5 頁6 頁7 頁8 頁9 頁10 頁11 頁12 頁13 頁14 頁15 頁16 頁17 頁18 頁19 頁20 頁21 頁22 頁23 頁24 頁25 頁26 頁27 頁

兒童病房失竊加欺凌。。(長文)

吹水台

碌鳩叫做jer

5 年前

3922

34 頁

選擇頁數1 頁2 頁3 頁4 頁5 頁6 頁7 頁8 頁9 頁10 頁11 頁12 頁13 頁14 頁15 頁16 頁17 頁18 頁19 頁20 頁21 頁22 頁23 頁24 頁25 頁26 頁27 頁28 頁29 頁30 頁31 頁32 頁33 頁34 頁

長文 男朋友真係好撚柒蠢

感情台

唔啱feel乜都無用

3 年前

-2

1 頁

選擇頁數1 頁

長文

感情台

你的媽媽

8 個月前

13549

40 頁

選擇頁數1 頁2 頁3 頁4 頁5 頁6 頁7 頁8 頁9 頁10 頁11 頁12 頁13 頁14 頁15 頁16 頁17 頁18 頁19 頁20 頁21 頁22 頁23 頁24 頁25 頁26 頁27 頁28 頁29 頁30 頁31 頁32 頁33 頁34 頁35 頁36 頁37 頁38 頁39 頁40 頁

[長文] 原來自己係咁依賴老母

吹水台

你睇唔睇到我

3 年前

-3069

41 頁

選擇頁數1 頁2 頁3 頁4 頁5 頁6 頁7 頁8 頁9 頁10 頁11 頁12 頁13 頁14 頁15 頁16 頁17 頁18 頁19 頁20 頁21 頁22 頁23 頁24 頁25 頁26 頁27 頁28 頁29 頁30 頁31 頁32 頁33 頁34 頁35 頁36 頁37 頁38 頁39 頁40 頁41 頁

長文慎入,啲男人咁鍾意打J?

感情台

蔡英文講中文

4 年前

2375

62 頁

選擇頁數1 頁2 頁3 頁4 頁5 頁6 頁7 頁8 頁9 頁10 頁11 頁12 頁13 頁14 頁15 頁16 頁17 頁18 頁19 頁20 頁21 頁22 頁23 頁24 頁25 頁26 頁27 頁28 頁29 頁30 頁31 頁32 頁33 頁34 頁35 頁36 頁37 頁38 頁39 頁40 頁41 頁42 頁43 頁44 頁45 頁46 頁47 頁48 頁49 頁50 頁51 頁52 頁53 頁54 頁55 頁56 頁57 頁58 頁59 頁60 頁61 頁62 頁

抗爭出現疲態(長文),附有建議

時事台

大力抽插

4 年前

2249

29 頁

選擇頁數1 頁2 頁3 頁4 頁5 頁6 頁7 頁8 頁9 頁10 頁11 頁12 頁13 頁14 頁15 頁16 頁17 頁18 頁19 頁20 頁21 頁22 頁23 頁24 頁25 頁26 頁27 頁28 頁29 頁

[長文慎入] 香港要有方舟計劃

時事台

咖啡揸流攤

2 年前

1378

20 頁

選擇頁數1 頁2 頁3 頁4 頁5 頁6 頁7 頁8 頁9 頁10 頁11 頁12 頁13 頁14 頁15 頁16 頁17 頁18 頁19 頁20 頁

家暴 - 長文慎入

上班台

超大碌男神

6 年前

-5

2 頁

選擇頁數1 頁2 頁

ocamp(長文)

感情台

友誼巴士

4 個月前

-622

23 頁

選擇頁數1 頁2 頁3 頁4 頁5 頁6 頁7 頁8 頁9 頁10 頁11 頁12 頁13 頁14 頁15 頁16 頁17 頁18 頁19 頁20 頁21 頁22 頁23 頁

[長文]我一家適唔適合移民英國?

World

共產黨總書記

2 年前

634

28 頁

選擇頁數1 頁2 頁3 頁4 頁5 頁6 頁7 頁8 頁9 頁10 頁11 頁12 頁13 頁14 頁15 頁16 頁17 頁18 頁19 頁20 頁21 頁22 頁23 頁24 頁25 頁26 頁27 頁28 頁

[長文]工程界之墜落及何去何從

上班台

中辣走酸走芫茜

5 年前

244

41 頁

選擇頁數1 頁2 頁3 頁4 頁5 頁6 頁7 頁8 頁9 頁10 頁11 頁12 頁13 頁14 頁15 頁16 頁17 頁18 頁19 頁20 頁21 頁22 頁23 頁24 頁25 頁26 頁27 頁28 頁29 頁30 頁31 頁32 頁33 頁34 頁35 頁36 頁37 頁38 頁39 頁40 頁41 頁

[長文連載]分享一年交友App經歷

感情台

馮老蛋

7 年前

1425

23 頁

選擇頁數1 頁2 頁3 頁4 頁5 頁6 頁7 頁8 頁9 頁10 頁11 頁12 頁13 頁14 頁15 頁16 頁17 頁18 頁19 頁20 頁21 頁22 頁23 頁

[多圖+長文]我今日生日,竟然收到⋯⋯

創意台

真名叫陳真

5 年前

325

10 頁

選擇頁數1 頁2 頁3 頁4 頁5 頁6 頁7 頁8 頁9 頁10 頁

[長文]感情價值觀

感情台

重岡大毅

5 年前

-861

8 頁

選擇頁數1 頁2 頁3 頁4 頁5 頁6 頁7 頁8 頁

(長文) 茶木收皮未

飲食台

Walaaaaaaa

7 年前

-4

6 頁

選擇頁數1 頁2 頁3 頁4 頁5 頁6 頁

長文慎入. 求意見

感情台

張天褲

4 個月前

-723

22 頁

選擇頁數1 頁2 頁3 頁4 頁5 頁6 頁7 頁8 頁9 頁10 頁11 頁12 頁13 頁14 頁15 頁16 頁17 頁18 頁19 頁20 頁21 頁22 頁

27歲⋯屋企人仲係咁管我【長文慎入】

家庭台

神勇爆炸哥

2 年前

3838

41 頁

選擇頁數1 頁2 頁3 頁4 頁5 頁6 頁7 頁8 頁9 頁10 頁11 頁12 頁13 頁14 頁15 頁16 頁17 頁18 頁19 頁20 頁21 頁22 頁23 頁24 頁25 頁26 頁27 頁28 頁29 頁30 頁31 頁32 頁33 頁34 頁35 頁36 頁37 頁38 頁39 頁40 頁41 頁

[警世] 長文慎入!Apple delete左我萬幾張相!

吹水台

易事健身

5 年前

4189

41 頁

選擇頁數1 頁2 頁3 頁4 頁5 頁6 頁7 頁8 頁9 頁10 頁11 頁12 頁13 頁14 頁15 頁16 頁17 頁18 頁19 頁20 頁21 頁22 頁23 頁24 頁25 頁26 頁27 頁28 頁29 頁30 頁31 頁32 頁33 頁34 頁35 頁36 頁37 頁38 頁39 頁40 頁41 頁

[長文慎入] 練大隻!全部同我練大隻!

健康台

中辣走酸走芫茜

5 年前

54

41 頁

選擇頁數1 頁2 頁3 頁4 頁5 頁6 頁7 頁8 頁9 頁10 頁11 頁12 頁13 頁14 頁15 頁16 頁17 頁18 頁19 頁20 頁21 頁22 頁23 頁24 頁25 頁26 頁27 頁28 頁29 頁30 頁31 頁32 頁33 頁34 頁35 頁36 頁37 頁38 頁39 頁40 頁41 頁

[長文連載]分享一年交友App經歷 (2)

感情台

好大壓力就爆煲

7 個月前

-335

26 頁

選擇頁數1 頁2 頁3 頁4 頁5 頁6 頁7 頁8 頁9 頁10 頁11 頁12 頁13 頁14 頁15 頁16 頁17 頁18 頁19 頁20 頁21 頁22 頁23 頁24 頁25 頁26 頁

(長文慎入)好撚大壓力就頂唔順 想jump

吹水台

大麻誠信破產

8 個月前

798

23 頁

選擇頁數1 頁2 頁3 頁4 頁5 頁6 頁7 頁8 頁9 頁10 頁11 頁12 頁13 頁14 頁15 頁16 頁17 頁18 頁19 頁20 頁21 頁22 頁23 頁

玩具反斗城員工入一入黎(長文慎入)

吹水台

heoshcjalla

3 年前

1255

41 頁

選擇頁數1 頁2 頁3 頁4 頁5 頁6 頁7 頁8 頁9 頁10 頁11 頁12 頁13 頁14 頁15 頁16 頁17 頁18 頁19 頁20 頁21 頁22 頁23 頁24 頁25 頁26 頁27 頁28 頁29 頁30 頁31 頁32 頁33 頁34 頁35 頁36 頁37 頁38 頁39 頁40 頁41 頁

長文慎入,有無人對apple慢慢開始失望......

手機台

好想俾貓撚

3 年前

-1

1 頁

選擇頁數1 頁

長文慎入

吹水台

tidusss

5 年前

1

1 頁

選擇頁數1 頁

長文慎入

政事台

GAYBUI

5 年前

8

1 頁

選擇頁數1 頁

[長文]616後

時事台

看天上的星

5 年前

2

1 頁

選擇頁數1 頁

[長文]爪牙

講故台

羅特

6 年前

0

1 頁

選擇頁數1 頁

[長文] 後備

感情台

Yin0..0

6 年前

0

1 頁

選擇頁數1 頁

長文慎入

感情台

男兒膝下有腳趾

6 年前

0

1 頁

選擇頁數1 頁

機會?(長文)

感情台

鴨嘴小火龍

6 年前

0

1 頁

選擇頁數1 頁

困惑“長文”

感情台

咪撚過薄大

2 年前

3448

22 頁

選擇頁數1 頁2 頁3 頁4 頁5 頁6 頁7 頁8 頁9 頁10 頁11 頁12 頁13 頁14 頁15 頁16 頁17 頁18 頁19 頁20 頁21 頁22 頁

[文長]中文大約的確已經死了

學術台

黒咲芽亜

11 個月前

150

10 頁

選擇頁數1 頁2 頁3 頁4 頁5 頁6 頁7 頁8 頁9 頁10 頁

[長文] 做運動會變高?

學術台

樹枝少年

6 年前

112

6 頁

選擇頁數1 頁2 頁3 頁4 頁5 頁6 頁

[長文]我大左肚,點算。。。

感情台

上啦上啦快啦

2 年前

319

31 頁

選擇頁數1 頁2 頁3 頁4 頁5 頁6 頁7 頁8 頁9 頁10 頁11 頁12 頁13 頁14 頁15 頁16 頁17 頁18 頁19 頁20 頁21 頁22 頁23 頁24 頁25 頁26 頁27 頁28 頁29 頁30 頁31 頁

[長文警世post] ig 黃店soul.852 賣假鞋、淘寶野

潮流台

浮浪者

3 年前

-46

37 頁

選擇頁數1 頁2 頁3 頁4 頁5 頁6 頁7 頁8 頁9 頁10 頁11 頁12 頁13 頁14 頁15 頁16 頁17 頁18 頁19 頁20 頁21 頁22 頁23 頁24 頁25 頁26 頁27 頁28 頁29 頁30 頁31 頁32 頁33 頁34 頁35 頁36 頁37 頁

[長文]認真研究 一家四口英國生活使費

World

零下七度

3 年前

3026

41 頁

選擇頁數1 頁2 頁3 頁4 頁5 頁6 頁7 頁8 頁9 頁10 頁11 頁12 頁13 頁14 頁15 頁16 頁17 頁18 頁19 頁20 頁21 頁22 頁23 頁24 頁25 頁26 頁27 頁28 頁29 頁30 頁31 頁32 頁33 頁34 頁35 頁36 頁37 頁38 頁39 頁40 頁41 頁

[長文]係黃店「葉宗記」買水果 結果一肚氣

吹水台

蕭別離

4 年前

-109

23 頁

選擇頁數1 頁2 頁3 頁4 頁5 頁6 頁7 頁8 頁9 頁10 頁11 頁12 頁13 頁14 頁15 頁16 頁17 頁18 頁19 頁20 頁21 頁22 頁23 頁

(長文) 腎虧 = 氣質猥瑣,膽小、內向、不善交際

感情台

地球的最後一夜

4 年前

1556

43 頁

選擇頁數1 頁2 頁3 頁4 頁5 頁6 頁7 頁8 頁9 頁10 頁11 頁12 頁13 頁14 頁15 頁16 頁17 頁18 頁19 頁20 頁21 頁22 頁23 頁24 頁25 頁26 頁27 頁28 頁29 頁30 頁31 頁32 頁33 頁34 頁35 頁36 頁37 頁38 頁39 頁40 頁41 頁42 頁43 頁

[長文]我想趁5月10前提醒大家一啲野。

時事台

丸走睪飛(一粒)

4 年前

650

30 頁

選擇頁數1 頁2 頁3 頁4 頁5 頁6 頁7 頁8 頁9 頁10 頁11 頁12 頁13 頁14 頁15 頁16 頁17 頁18 頁19 頁20 頁21 頁22 頁23 頁24 頁25 頁26 頁27 頁28 頁29 頁30 頁

(長文)(置頂)(戰術)金鐘野餐攻略 沙盤推演

時事台

極速神驅

4 年前

557

41 頁

選擇頁數1 頁2 頁3 頁4 頁5 頁6 頁7 頁8 頁9 頁10 頁11 頁12 頁13 頁14 頁15 頁16 頁17 頁18 頁19 頁20 頁21 頁22 頁23 頁24 頁25 頁26 頁27 頁28 頁29 頁30 頁31 頁32 頁33 頁34 頁35 頁36 頁37 頁38 頁39 頁40 頁41 頁

[長文] 許穎婷: 拆解 「香港人=華裔血統」迷思

時事台

細清

5 年前

1060

36 頁

選擇頁數1 頁2 頁3 頁4 頁5 頁6 頁7 頁8 頁9 頁10 頁11 頁12 頁13 頁14 頁15 頁16 頁17 頁18 頁19 頁20 頁21 頁22 頁23 頁24 頁25 頁26 頁27 頁28 頁29 頁30 頁31 頁32 頁33 頁34 頁35 頁36 頁

[長文]得罪講句 你班MARVEL撚都真係幾撚煩

影視台

妄求王子

5 年前

531

35 頁

選擇頁數1 頁2 頁3 頁4 頁5 頁6 頁7 頁8 頁9 頁10 頁11 頁12 頁13 頁14 頁15 頁16 頁17 頁18 頁19 頁20 頁21 頁22 頁23 頁24 頁25 頁26 頁27 頁28 頁29 頁30 頁31 頁32 頁33 頁34 頁35 頁

[長文] Longd除夕被分手 新一年會好好的❤️

感情台

吃檸檬

2 年前

7079

15 頁

選擇頁數1 頁2 頁3 頁4 頁5 頁6 頁7 頁8 頁9 頁10 頁11 頁12 頁13 頁14 頁15 頁

「長文」女朋友有個呀哥

感情台

叉雞飯(多飯)

4 年前

-270

7 頁

選擇頁數1 頁2 頁3 頁4 頁5 頁6 頁7 頁

[長文]我係犯賤的港女

感情台

Merry-go-round

5 年前

134

5 頁

選擇頁數1 頁2 頁3 頁4 頁5 頁

[長文]為甚麼我不聯署?

時事台

丹比利記名宿

5 年前

1811

6 頁

選擇頁數1 頁2 頁3 頁4 頁5 頁6 頁

[長文]今晚 我老豆死左

政事台

第 1 頁

1 頁2 頁3 頁4 頁5 頁6 頁7 頁8 頁9 頁10 頁11 頁12 頁13 頁14 頁15 頁16 頁17 頁18 頁19 頁20 頁21 頁22 頁23 頁24 頁25 頁26 頁27 頁28 頁29 頁30 頁31 頁32 頁33 頁34 頁35 頁36 頁37 頁38 頁39 頁40 頁41 頁

下一頁

\#1易事健身•5 年前

大家好,我地係易事健身 Easy Fitness。

我地係一班業餘健身愛好者,有男有女,都發過夢。

有感好多一齊發夢嘅,無論係學生定係大人,好多一睇就知冇做開運動,一係好瘦,一係偏肥。大佬呀!每次講到ENDGAME咁,但係你平時唔好好鍛鍊自己,點保護自己同埋身邊嘅人呢?

我地覺得香港健身風氣真係太差!我地想試下實行一個全民健身計劃!我地要你地一齊練。大。隻!

Q. 點解要練大隻? 女仔都要?

A. 有力先發夢搬得郁物資架嘛;

有力先carry到啲裝備架

遇到正面衝突,至少有個勢,即使唔夠揪,走佬都要跑得快啦。

所以,養成運動健身習慣非常重要!

香港冇軍訓,先天上已經蝕底啲。

女仔都會一齊發夢,就算你唔敢企最前,而69, 612係好多女仔的),後勤都需要你撐住。

我地想各位會發夢嘅易事強身健體,鍛鍊體能,增強自信,比狗咬都唔怕!

Q. 咩人適合參加呢個健身計劃?

A. 我地目標係幫新手培養健身習慣,講緊完全冇接觸過健身,而平時又極少運動嘅人。

如果你係已經接觸健身一段時間但停濟不前,或者想操fit體能入紀律部隊(考狗免問,屌你老母),我地都會盡量幫你,但係優先次序會俾新手先。

Q. 駛唔駛收錢?

A. 唔收。完全免費。你想支持我地,可以加入我地幫手,或者課金去義士基金或傳媒。

Q. 俗語都有話免費嘅野最貴,點解你地唔收錢為其他人付出?

A. 如果你真係想問呢個問題,不如去問義士,不如去問一直默默付出嘅學生。

我地相信每個人喺各自嘅崗位努力,我地想有心人知道其實佢地並不孤單。

Q. 我乜都唔識,你地準備點教我?

A. 易事健身推廣健身風氣暫定有兩種方式

⠀⠀⠀⠀⠀⠀⠀⠀⠀

1. 推廣適合新手嘅健身資訊。

我地知道健身一大堆資訊新手好難消化,所以我地會預備由淺入深同埋有趣方式將有用嘅資訊講出黎。

⠀⠀⠀⠀⠀⠀⠀⠀

2. 一對一Remote Mentorship Programme

我地覺得健身冇一套必定正確嘅方法,每個人適合唔同方法同埋計劃。我地目標係一個Mentee配一個Mentor,幫你度身設計個人化嘅健身目標同計劃, 幫你由零開始慢慢學習同進步。萬事起頭難,我地覺得有個人一開始帶住你,可以走少好多冤枉路,洗少好多冤枉錢。

Q. 我係女仔,但係覺得問男mentor會怕醜。

A. 我地男仔教男仔,女仔教女仔。我地絕對唔會問你拎個人資料,除咗身高體重同職業等。你亦都絕對唔好向我地提供任何個人資料。我地希望Mentor同Mentee保持一個平等同互相尊重嘅關係。

Q. 如果我想參加你地嘅健身計劃,有乜野需要預備?

A. 刻苦,決心,毅力,意志,時間,精神同埋紀律。

你可以揀買啞鈴喺屋企練,喺街練,去康文,physical,或者clubhouse etc,點都好過你乜都唔做成碌木企喺度。

Q. 有巴絲問,熱狗以前都號召過人做類似野,你地同佢地有乜分別?

A. 我地只係一班小市民,連組織甚至團體都講唔上。我地只係想大家練大隻,保護自己同保護身邊嘅人,就係咁簡單。聽落好似微不足道,不過我地相信,每個人都係一個小齒輪,如果每粒小齒輪都可以更健康、更強壯,先可以推得郁大齒輪。

Q. 想幫手可以點?

A. 加入文宣組或者加入成為mentor。文宣組識普通製圖就可以;Mentor我地希望有健身經驗、知識同熱誠就得,唔需要好叻或者好專業,最重要係教到新手健身果陣點樣盡量避免受傷。如果你係健身老手,又想幫手一齊帶起健身風氣,歡迎留個TG傾傾。

(屌你班大隻佬淨係識嗌交,又唔出黎幫手,抵你俾人笑,係咪?想洗脫柒名就黎幫手操FIT各位巴絲)

最後,先小人後君子,希望你至少為自己定立一個目標。如果你就咁留低一句「我想減肥」或者「想練腹肌,食少啲有冇用?」我地好難幫你。

記住記住,我地只能從旁協助,你有疑問歡迎提出,我哋一定會盡所能去解答,但如果你期望我地比啲一式一樣嘅餐單你去跟,比幾個動作你去做,然後你本人唔經思考盲目咁跟,唔使點附出就冇啦啦變到好fit,sorry,你可以撳返上一頁,我哋幫唔到你。

任何嘅改變都係由你自己開始,堅持堅持再堅持,先會有結果。

我地依家啱啱開始冇耐,已經幫緊幾位巴絲設計咗Programme。原諒我地人手真係不足,暫不考慮落區健身。

如果你有決心作出改變,但又無從入手;或者你都覺得我地個概念好,想幫下手,不妨留低TG試下搵我地傾下,或者喺下面個IG DM我地,的起心肝去試下行出第一步。

https://www.instagram.com/easyfitness.easyfitness/

最後最後,天佑香港,共勉之。


\chapter{[長文]做一個好人係咪好on9\_}

[長文]做一個好人係咪好on9?

一直以來都覺得應該要對人友善,有禮貌,幫人,自己蝕底少少唔緊要,諗咗人哋先諗自己,覺得對其他人好好緊要,要有同埋心,會去迎合人,盡量唔會去得罪人,唔會特燈激嬲人,但係發現自己唔識嬲,係咁俾人鬧on9,kam….仲會當佢係friend

但係一直都有唔同既人同我講話我太單純,太傻,太善良,太乖,連個樣都似乖仔,第日一定會被人利用,我聽完之後心諗邊有咁多壞人吖

但係近排有幾個friend 話佢做任何嘢嘅原則係諗對佢自己有冇利益,話一切都係利益關係,友誼同埋愛係不切實際,全部都係建基利益之上,冇利益嘅話睬你都傻,例如佢親口同我出街係自己想搵開心,另一個friend 話我主動搵佢傾計,係想同一間U入面有多個人可以利用,佢哋仲話其實個個人都係當朋友工具,利用,只係唔講出嚟。另外有次同個女仔friend傾,佢講有人會因為為未來利益而approach佢,到底係我太單純諗唔到定佢太心機

但係我一直冇咁樣諗過當friend係工具,只係純粹想識啲真心朋友,其實係咪我個諗法太on9,完全唔適合喺香港社會,定呢個係世界嘅真相,個個都係諗自己利益

有冇巴打好似我咁


\chapter{[長文]心入面有條刺}

呢樣嘢一直收喺心入面好耐

男朋友其實好錫好愛我,咩都就曬我,完全係將全個人所有嘢奉獻俾我嗰種愛,我哋又住得近日日見咁滯,我係覺得好幸運

但同時佢都做過一樣我接受唔到嘅錯事,就係食咗我隻豬

因為我本身係處,一直都表明過唔俾。但有次佢襯我醉醉哋搞咗,仲要冇套。雖然只係幾下同冇射,最後亦冇事,但我介意嘅係佢無視我一直以嚟強調唔扑得呢樣嘢。我未至於想留到結婚後,但我好想我嘅第一次係心甘情願,因為好愛一個人好想再親近啲而俾。而且,佢唔係處,以前都成日做,令我更介意。

因為咁我哋鬧過分手,最後我原諒咗佢。

你可能會冷笑一聲,on9,嗰個係你男朋友,搞你都好正常有咩要嬲要唔開心。但在我角度,我係被人強姦,而嗰個仲要係自己男朋友。自此一聽到破處、扑嘢、強姦等等嘅話題,我內心都會好洶湧好激動。

我之後有再同佢講,就算搞過一次,都唔代表以後可以。佢都表示明白同肯忍肯等。我覺得係要好親密好穩定嘅關係、好愛一個人先可以做,我同佢一年都未有,我覺得自己未到嗰個點,即使佢真係好愛我。

講係咁講 有一兩次佢好想要,我都有俾。當刻都會開心嘅,畢竟係同鍾意嘅人親近。但係每次靜番落嚟,我都會好唔開心,因為我要嘅唔係咁,因為明明覺得未到可以扑嘅感情基礎。我覺得自己係抱住一種「反正都已經唔係處,做一次同做一百次都冇分別」嘅心態而俾,但咁樣就同我嘅原則,啫係覺得做愛係感情嘅頂峯、唔係一件隨便嘅事 背道而馳咗。會覺得自己係廢物,堅守唔到自己原則。

我都好想同佢享受性愛嘅歡愉,想自己唔好諗咁多,唔好每次靜落嚟都會變番唔開心,唔想自己又要俾又要sad咁麻q煩⋯但 有條刺就係有條刺

再做時 往往會有一種因為佢錯咗一次就將錯就錯,令佢可以如願以償咁嘅感覺。我唔想帶住屈服、無奈、認命嘅心態去做,咁樣好可悲亦好辛苦。

另一個失落嘅原因係 明明佢實牙實齒講到以後就算係我話想要,都唔會做,因為知我其實唔係想咁/之後會唔開心,講盡一切好好聽好冠冕堂皇嘅說話,但,佢其實做唔到⋯ 明明佢係認真咁講因爲我太吸引令佢忍唔住,喺我耳中難聽過粗口。

我知佢唔係為扑而對我好,而係真心愛我,但係發生過嘅事就係發生過,時隔半年我一樣好介意。呢啲感受我都會直接同佢講,佢除咗歉疚其實咩都做唔到。

我知我又要俾,俾完又要唔開心係好戇鳩,但我都控制唔到自己腦內大亂鬥

話唔可以再做又唔係,以後開綠燈又唔係,內心非常矛盾,心累⋯

究竟我可以點應該點







% \chapter{旺角行}
% 旺角行

% 旺角街巷人來往,指夾香煙孤遊蕩

% 遊蕩花園心等望,友來一刻喜若狂

% 未見三月萬緒谷,此聚別我英倫讀

% 澤少先來黃生遲,寒暄盡省慳冗辭

% 此晚共歡不欲止,始謦已悲離別時

% 本作餞行宴會延,卿我皆知還有事

% 穿耳之約討云云,再不成事恨此生

% 大學五載將終告,傘運至今未曾做

% 霓虹艷麗二極俍[1],旺角逆韓仍時尚

% 港男港女MK味,粵音悠悠香港地

% 潮特兆萬多奇舖,戒環款式任君數

% 問主何價心意亂,此結義深銀包損

% 須臾萬世針一穿,一穿百念友情存

% 彼左我右閃炩炩,再買兩三笑盈盈

% 創興廣場食肆多,捕飛邊爐部隊鍋

% 百層揀盡中日韓,卒選放題節制喪

% 七色魚生大喜屋,蒸揚燒煮滿口福

% 清酒暢飲杯杯乾,精饈維美啖啖肉

% 餞宴盛惠二百七,淢[2]意未耗仍有凸

% 黃生終參我哋倆,無星夜晚多下場

% 共知密地各自往,私聚好處市中藏

% 昔日驕鐵今賤銅,冤魂未雪太子封

% 彌敦大道的士惡,貼錢買難人墮落

% 優步寶馬位來定,把握時間偈謦謦

% 不語並哀時下局,不義蓋天心鬱郁

% 槍林彈雨催淚驚,苦吟救命新屋嶺

% 邪道得勢英雄烈,元朗閩民何時滅

% 犬聲載道人委屈,藍衣亮刀斬見骨

% 每每下手不留情,血碎皮包欲手甩

% 有人不做做曱甴,語出身定決心殺

% 自持凜義瞄頭發,心花怒放手狠辣

% 揮棍韰爽[3]血淋淋,警寇歹作笑淫淫

% 藞苴手段打嚇氹,高叫我名滅聲𢫏[4]

% 男軀壯健可圍毆,女有小穴屌過夠

% 變態天倫齊齊玩,中出輪姦心燦爛

% 魔警發洩狎謔戲,玩完即棄落海死

% 黑衣義士夜不歸,街坊對芒[5]嘆閉翳

% 權不講理政乏義,天直民直胡話兒[6]

% 遂令百年法治體,不求達義求受規

% 中西禮樂俱崩壞,吾民尊嚴唔嬲踩

% 酷楚施盡半屍骸,誰人叫你話不乖?

% 似是而非官話語,未到三句服港豬

% 華帝赤納天下知,榮光歸港待何時

% 義仇義憤心中燒,達義之法唯私了

% 警寇手下無完膚,私了殘疾尚餘辜

% 對此極惡刑孰效,文革批鬥可參考

% 辱盡罪身尚有仁,此時澤少有餘猶

% 我笑彼思或藍色,澤少不忿百辭逆

% 語出方來我驚恐,瘋測祈求永不中

% 聚會卒之黃生來,幽靜擁抱紅酒開

% 喜些酒蓮兼暗梅,相望無懫[7]呼聲悔

% 亮火一擦點百愁,呼出煙雨念浮浮

% 浮浮幽愁語語思,呼呼泣訴不得志

% 黃生本作律士徒,冥中作祟困楚途

% 自家生意半工讀,朝勞晚冊未閒哭

% 身在福中知其福,不敢牢騷怨路曲

% 澤少自細優才生,入讀醫系心漸仁

% 天生帥氣人人道,強顏歡笑匿頹魂

% 憶昔仍享逸樂時,科科皆甲話般易

% 十五初奪紫荊𧘹,為港爭光無辰止

% 牛津面試光輝晉,光輝歲月忽暴殉

% 父去家散母薨逝,與妹相依捱生計

% 學業荒廢情斷繫,欄杆哭笑命無稽

% 蓮梅諸花忘愁借,苦海洋洋萬事亾[8]

% 顧影自憐聲嗟嗟,唯恥有辱更慘者

% 貿辭易念求意達,有否靈犀恐有惑

% 念緒纏纏冗辭長,誰曉冗辭回心響

% 夜色漸薄日繼出,筵席有散天地律

% 最後一煙功效淺,只求相別得拖延

% 晃淚殘顏講再見,封印此時共懷緬

% 鐵鳥載千半港生,粵音英語官話䘲

% 嘻哈閒語一身貴,貴族何曾履義為

% 脊斷哀號不留人,眼盲未聞飛霄雲

% 遠走高飛念重重,倫敦天氣灰濛濛

% 掟磚縱火手欲發,躲內對芒求刺殺

% 心在前線無獻奉,冷氣軍師匿懦容

% 煲底相聚待何日,凱旋歸來道莫窮

% 勸君明哲莫上前,背後發功效倍千

% 對斯雨絲似問道,避世怕死定正見?

% 越想慚漸愧流淚,書此奉微港萬歲

% [1] 俍,leong1,力丈切陰平。「老土」、「過氣」、「缺乏美感」之意,一般用於形容他人衣著打扮,或事物的外表形態及其裝飾等。用例:「佢著衫勁俍囉。」「個新建成嘅市政大樓嘅樣好俍呀。「俍」本義為「善於,擅長」,又解作「行走緩慢」,本音為「loeng4」,除了曾於古書和部分生僻詞中使用之外,基本上非常用漢字。採納「俍」是因為「俍」乃形聲字,且「俍」本用於形容他人衣著,故適合。

% [2] 淢,wek1,解「出去玩」、「出街蒲」之意,特別是指夜晚到煙酒風流之地花天酒地、男歡女愛。Wek1 據筆者理解並無所謂之本字,亦不見任何懷疑乃本字的選擇。愚見以為,可自行賜字。與其花費精力做字,不如死字復活,舊字新用,現取「淢」。「淢」乃死字,本義解作「急流」,粵語本音為 wik6。現正舊字新用且訓讀作罷。那取「淢」有沒有任何的內部邏輯和原因呢?理論上,按照傳統的漢字造字賜字理則,wek1 大概會以形聲字字來解決其有音無字的問題。而按照這套邏輯,不論其聲符是什麼,斷估其邊旁必定為「女」字旁。理論上應選用女字旁,無非是因為傳統漢字理則以女為樂,如「娛」、「耍」等。故此若然要承繼此邏輯,wek1 之字書符號理應像「女或」樣子的東西。我認為此案不可取。其一是因為沒有這個電腦字符,其麻煩不抵其得益。但最重要的是,斯字選延續著一種不要得的辱女思維和物化女性思維。取「淢」,是有見其字的水字旁。粵語人以水為財,斯理則可屢見於其詞彙措辭,茲不贅述。取「淢」,就是要強調和指出,「出去淢」的這個行為,是要花費很多錢(水)的。

% [3] 韰爽,haai1 song2,「韰」「爽」並列詞。「韰」則「high」,取死字「韰」以書之。茲創「韰爽」一詞,目的在於以粵語語素(茲則為英源粵語素)造詞,以鞏固其語素的生命力,防止死亡。

% [4] 𢫏,kam2。解「掩蓋」、「覆蓋」,例詞:「𢫏被」、「𢫏蓋」、「𢫏牌」。例句:「哎呀,個仔竟然怕生保怕到𢫏住塊面喎!」

% [5] 芒,mong1,「monitor」折取之後得「mon」繼而粵化之詞。

% [6] 直,嚴復對「right」的創譯。此創譯的選擇背後包含了非常深思熟慮的考慮,迴避了當今以「權」譯「right」而導致「might(權) is right(權)」的巨大文明性問題。 「天直」為「natural right」,「民直」則為「civil liberty」,「胡話兒」則為雙關語,因為「天直民直」是泰西的概念,也是西方最西方對中國指指點點的胡話兒,在某些人的眼中更是「胡話兒(屁話)」。

% [7] Judge 一詞,取「懫」字。其取字顯然為形聲字,但未嘗未有半絲會意字之意味。可考慮以「懫」來字書化「judge」一詞。「懫」本音 zi3,本義為「偏激、凶狠的怨恨」,又或「 阻止;塞滿。」基本上是死字。現正訓讀為 zat 6。取「懫」,一方面是因為其形聲結構合宜,其二是考慮到某程度上「心質為懫(Judge)」的結構有輕微的會意味道。

% [8] 亾,讀 he 3,去旡切陰去。意思豐富,《粵典》列「亾」有四解:一、敷衍、不認真、交行貨;二、無所事事,漫無目的地打發時間;三、求其、不認真、馬虎、得過且過;四、形容事情缺乏挑戰性,可以輕鬆完成。例句:「佢做野不嬲都係咁亾。」「我而家日日淨係喺屋企亾。」「份報告亾俾佢咪算囉。」「個測驗超亾,冇溫書都實合格。」坊間俗寫有「hea」、「迤」、「迆」等,最為普遍這為「hea」。茲取《學苑》中《提升香港話地位 — — 剝脫既定思想賦予口語詞彙書面寫法》一文之建議,取「亾」為其字。「亾」本為「亡」之異體字,但已乃死字一個。該文建議取「亾」為「hea」之賜字,是因為「亾」像一人躺臥在床上百無聊賴之形。茲實乃「形借」之為。「形借」類似「假借」,都是把已有之字重新註釋,注入新用法的技巧。「假借」借音借形不借義,「形借」則借形不借義,或借或不借音。形借字數量很少,基本上都只存在於網上語言,最為人知的例子莫過於「出獄」意思的「出冊」的「冊」。這裏的「冊」,意思不是「書冊」,而是取「冊」像監獄鐵欄之形,繼而從新註釋「冊」為監獄。斯「冊」借形借音不借義,而我們的「亾」則借形而義音均不借。其他的頗為流行的形借字包括「囧」、「厹」等等。
\printindex % Print the index










\end{document}
